In the course of the implementation and evaluation of the proof-of-concept accompanying this thesis, a number of possible improvement arose, which were not explored further but which can form the basis of future investigations. Specifically:

\begin{description}
	\item[Causality-based world simulation.] Presently, the agents create plans by taking hypothetical actions and simulating the world state as a result of these. As a consequence, the lengths of plans and the number of time-steps required to perform them correspond one-to-one.
	This schema is functional, but has apparent drawbacks when we compare it to the way in which humans plan actions. If, say, one wanted to go 100 steps in a straight line to get a glass of water, one would not consider each step. Rather, one would summarize the required 100 steps as the single action ``walk in a straight line towards the glass''. Similarly, if one had to wait ten minutes for a train, one would not consider what to do during each second of the wait; one would simply resolve to ``sit there''. Clearly, not all actions or series of actions are explicated to the same degree in the minds of humans when they make plans.
	
	 It thus stands to reason that, during the planning process, one ought to consider a sort of {\em causal distance} --- that is, the number of actions which the agent regards as qualitatively distinct. As soon as we begin to group actions together and distinguish temporal from causal distance, the question during planning ceases to be ``how long will it take to achieve X?'' and becomes ``how complicated is it to achieve X?''.
	 
	 \item[Goal-based planning.] The present planning scheme first selects an emotion to serve as the guiding one. The planning process then proceeds until the guiding emotion is either satisfied, leading the the plan's execution, or until a conflicting emotion overpowers it, leading to the plan's abortion. This is, once again, basically functional, but one could improve upon it by associating certain outcomes --- e.g. sating one's hunger or killing a Wumpus --- with certain emotions and selecting one of these as goals to reach. Agent would thus no longer seek to satisfy their dominant emotions by any means possibly, but by working towards specific goals. 
	 
	 \item[Emotional learning.] In conjunction with goal-based planning, one might also make the association of outcomes with emotions a dynamic one. Agents would be able to learn what constitutes a ``good'' or ``bad'', or a ``pleasurable'', ``painful'' outcome. 
	 
	 \item[Inference about world-states and forgetting.] The agents' memory is merely a perfunctory fact-storage which remembers past perceptions about the world. Importantly, it does not incorporate inferences about likely changes which an agent might reasonably learn, such as the fact that plants regrow or that an agent which was last seen surrounded by 10 Wumpuses is likely dead now. The learning and application of such inferences made about the likely, but not directly observed, changes in the world is an open-ended area of improvement, but it would certainly lead to much-optimized behaviour.
	 
	 \item[Concept synthesis.] Although agents are able to experience individual facts about their surrounding world, they do not create larger concepts from these facts to serve as cognitive shortcuts. An agent might perceive three Wumpuses in front of it, say, but it presently has not concept of ``three Wumpuses'' or ``a horde of Wumpuses''. One can think of many other macroconcepts which would directly aid in the creation of efficient plans and reduce cognitive load: ``a dangerous area'', ``an aggressive agent'', ``a gathering-place for Wumpuses'', etc.
	 
	 \item[Evolution of neural nets.] Emotional reactions are currently hand-crafted; the personalities of agents customized by inserting different nets for individual emotions. One might instead allow emotions to evolve by applying genetic algorithms to the neural nets, selecting the best-performing agents in each generation and creating the agents of the next one through recombination and mutation of their parents' neural nets.
\end{description}
