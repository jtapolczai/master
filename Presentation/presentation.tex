\documentclass[handout]{beamer}

% for themes, etc.
\mode<presentation>
{ \usetheme{boxes} }

\usepackage{times}  % fonts are up to you
\usepackage{graphicx}

% these will be used later in the title page
\title{Design and Implementation of an Agent Architecture combining Emotions and Reasoning}
\author{Janos Tapolczai \\
   Matrikelnummer 0825077 \\
}
\date{Wien, 09.05.2016}

% note: do NOT include a \maketitle line; also note that this title
% material goes BEFORE the \begin{document}

% have this if you'd like a recurring outline
\AtBeginSection[]  % "Beamer, do the following at the start of every section"
{
   \begin{frame}<beamer> 
      \frametitle{Content} % make a frame titled "Outline"
      \tableofcontents[currentsection]  % show TOC and highlight current section
   \end{frame}
}

\begin{document}
   
   % this prints title, author etc. info from above
   \begin{frame}
      \titlepage
   \end{frame}
   
   \section{Our Goal}
   
   \begin{frame}
	   	\frametitle{Our Goal}
	   	
	   	\begin{itemize}
	   		\item What is AI supposed to be able to do?
	   		\pause
	   		\item Many approaches based on search/logic exist:
		   		\begin{itemize}
		   			\item A* search,
		   			\item Iterated deepening DFS,
		   			\item Answer-set programming, etc.
		   		\end{itemize}
		   	\pause
		   	\item These do well, but world-ontology has to be encoded in explicit rules!
	   	\end{itemize}
   \end{frame}
   
   \begin{frame}
   	\frametitle{Our Goal}
   	
   	\begin{itemize}
   		\item Neural networks are another approach:
   		\begin{itemize}
   			\item supervised learning,
   			\item unsupervised learning,
   			\item hierarchical learning, etc.
   		\end{itemize}
   		\pause
   		\item The network must approximate some function with the help of training data.
   		\pause
   		\item We must have labelled examples/cost functions available.
   		\pause
   		\item It's not obvious what a neural network ``understands''.
   	\end{itemize}
   \end{frame}
   
   \begin{frame}
      \frametitle{Our Goal}
      
   	\begin{itemize}
   		\item Neural networks are another approach:
   		\begin{itemize}
   			\item supervised learning,
   			\item unsupervised learning,
   			\item hierarchical learning, etc.
   		\end{itemize}
   		\pause
   		\item The network must approximate some function with the help of training data.
   		\pause
   		\item We must have labelled examples/cost functions available.
   		\pause
   		\item It's not obvious what a good neural network ``understands''.
   	\end{itemize}
   \end{frame}
   
   \section{Biological Considerations}
   \section{Architecture}
   \section{Results}
   \section{Conclusion}
   
%   \begin{frame}
%      \frametitle{A Geometry Proof}
%      
%      (Illustrating {\sc beamer}'s $\backslash$uncover command.)
%      \vskip 0.5in
%      
%      \begin{theorem}
%         The angles in a triangle sum to $180^{\circ}$.
%      \end{theorem}
%      
%      \pause
%      
%      Plan:  Extend AC past C to D.  Draw CE parallel to AB.
%      
%   \end{frame}
   
%   \begin{frame}
%      \begin{proof}
%         \begin{tabular}{ll}
%            % uncover makes advanced overlay
%            \uncover<1->{1. u = y} & \uncover<2->{Alternate angles of a
%               transveral.} \\ 
%            \uncover<3->{2. v = x} & \uncover<4->{Consecutive interior angles of a
%               transveral} \\ 
%            \uncover<5->{3. z+u+v = $180^{\circ}$} & \uncover<6->{ACD is a straight
%               line.} \\ 
%            \uncover<7->{4. z+y+x = $180^{\circ}$} & \uncover<8->{Substitution
%               from Steps 1 and 2.} \\
%         \end{tabular}
%      \end{proof}
%   \end{frame}
\end{document}
