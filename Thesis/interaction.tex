Section~\ref{sec:worldSimulation} outlined what could be called {\em deliberate action} in the from of a planner-world-simulator loop. Section~\ref{sec:affect} described the structure and components of affect. These systems are of course not isolated from each other; emotional states influence both the planner's chosen heuristics and the world simulator's creation of worlds. In addition, attention, also influenced by affect, controls the allocation of cognitive resources. We now explore these relationships in further detail.

\paragraph{Planning as search} In the AI literature, search algorithms are of great importance. In this context, we can view the loop between planner and world simulator as a greedy search: the planner chooses the nodes which are to be expanded and sends them to the world simulator. It, in turn, performs the expansion by simulating the appropriate worlds. These simulated worlds are sent back to the planner for evaluation regarding desirability (i.e.\ cost). This presents an obvious problem: since greedy search is not complete, our planner-world-simulator loop can't be complete either. In fact, the situation is worse --- greedy search computes the cost of all candidates for expansion and chooses the cheapest, whereas our planner, being heuristic, might not consider certain nodes at all.

This might seem damning, but we must also consider the interaction with attention and memory. First, planned steps are committed to memory and thus, we gain access to past costs. An agent does not plan blindly, but can recall how long its plans are and what costs past planned steps entail. Given this information, we can turn the greedy algorithm into an A$^*$ search, with the qualification that the planner might not consider certain nodes. The mechanism of attention can further be used to enhance the search: if planning along a certain path takes too long, the agent might decide to abandon it altogether and start afresh with a different strategy. This failure too is stored in memory and can influence the planner in the new planning process by making the proposing of steps of the previously pursued path  unlikely.