\documentclass[]{scrartcl}

\usepackage[utf8]{inputenc}
\usepackage[top=1.5in, bottom=1.5in, left=1in, right=1in]{geometry}

%opening
\title{Umriss: Masterarbeit zu Affective Computing im Kontext von \texttt{dlvhex}}
\author{Janos Tapolczai}

\begin{document}

\maketitle

\begin{abstract}

\end{abstract}

\section {Literaturrecherche}

\begin{itemize}
	\item Grundsätzlich 2 Ansätze: ingenieurmäßige Konstruktion (mit Hilfe von Fuzzy Logic, Bayesschen Netzwerken u.ä.) vs. Modellierung der physischen Vorgänge (neurale Netze).
	\item Vielzahl von konkurrierenden Modellen: cell assemblies, neurale Netze, OCC model, Emotion Model von Gallanho \& Hallam.
\end{itemize}

\section{Stichworte: geplante Schwerpunkte}

\begin{itemize}
	\item Konstruktion/Implementierung eines parametrierbaren Modells von Emotionen \& Integration dieses Modells in dlv-basierte, rationale Agenten.\\
	Vorzugsweise auf Basis der {\em somatic markers hypothesis} \& {\em cell assemblies}.
	\item Durchführung eines genetischen Algorithmus: Agenten treten gegeneinander an, erfolgreiche Agenten werden miteinander kombiniert und bilden die Basis für die nächste Generation.
	\item Szenarien: typische Situationen für soziale Wesen, aber abstrakt gehalten - Kampf um Ressourcen, Aufteilung von Beute, einfache Verhandlungen, etc.
\end{itemize}

\section{Unklarheiten}

\begin{itemize}
	\item Inwiefern soll das Modell parametrierbar sein? Sollen nur einige numerische Parameter angepasst werden können, oder soll auch der Programmablauf verändert werden können?
	\item Wie soll die Effektivität der Agenten beurteilt werden? Anhand von festgelegten, händisch erstellten Szenarien, oder automatisch?
	\item Wie wird die Welt simuliert? Ist der dlv-Solver Teil der Agenten, oder wird die Welt selbst mit dlv modelliert?
\end{itemize}

\end{document}
