\documentclass[draft,final]{vutinfth}

\usepackage{graphicx}
\usepackage{soul}
\usepackage{caption}
\usepackage{amsmath}
\usepackage{amssymb}
\usepackage{amsthm}
\usepackage{subcaption}
\usepackage[utf8]{inputenc}
\usepackage[top=1.5in, bottom=1.5in, left=1in, right=1in]{geometry}
\usepackage[usenames, dvipsnames]{xcolor}
\usepackage{framed}
\usepackage{booktabs}

% Tikz
\usepackage{tikz}
\usetikzlibrary{calc}
\usetikzlibrary{arrows.meta}

\makeatletter
\newtheoremstyle{break}
   {\topsep}{\topsep}%
   {\itshape}{}%
   {\bfseries}{}%
   { }
   {\thmname{#1}\thmnumber{\@ifnotempty{#1}{ }\@upn{#2}}%
    \thmnote{ {\bfseries(#3)}}.}% 
\makeatother
\theoremstyle{break}

\newtheorem{thm}{Theorem}
\newtheorem{definition}[thm]{Definition}
\newtheorem{invariant}[thm]{Invariant}
\newtheorem{example}[thm]{Example}
\newtheorem{notation}[thm]{Notation}

\newcommand{\mt}[1]{\textnormal{#1}}
\renewcommand{\tt}[1]{\texttt{#1}}
\colorlet{shadecolor}{yellow!60}
\newcommand{\highlight}[1]{\colorbox{yellow!70}{#1}}
\newcommand{\shade}[1]{\begin{shaded}#1\end{shaded}}

\newcommand{\N}{\mathbb{N}}
\newcommand{\Q}{\mathbb{Q}}
\newcommand{\R}{\mathbb{R}}
\newcommand{\B}{\mathbb{B}}
\newcommand{\C}{\mathbb{C}}
\newcommand{\allQ}[1]{\left[\forall #1 \right]}
\newcommand{\exQ}[1]{\left[\exists #1 \right]}

\newcommand{\co}{\mathbf{Co}}
\newcommand{\cansend}[2]{#1 \rightarrowtail \{#2\}}
\newcommand{\cantsend}[2]{#1 \stackrel{\times}{\rightarrowtail} \{#2\}}
\newcommand{\canrec}[2]{\{#1\} \rightarrowtail #2}
\newcommand{\cantrec}[2]{\{#1\} \stackrel{\times}{\rightarrowtail} #2}
\newcommand{\rec}[1]{\field{rec}(#1)}
\newcommand{\ft}[1]{\tt{ft}_{#1}}
\renewcommand{\int}[1]{\tt{int}_{#1}}
\newcommand{\proc}[1]{\tt{proc}_{#1}}
\newcommand{\sends}[3]{#1 \rightarrow [#2] \rightarrow #3}
\newcommand{\sendsm}[4]{#1 \rightarrow [#2 , #3] \rightarrow #4}
\newcommand{\sendsf}[6]{#1\langle #2 \rangle \rightarrow [#3,#4] \rightarrow \langle#5\rangle#6}

\newcommand{\compT}[2]{\tt{Comp}_{#1,#2}}

\DeclareMathOperator{\tint}{int}
\newcommand{\type}[1]{\mathtt{#1}}
\DeclareMathOperator{\cint}{cint}

% Symbol for the extended Wumpus world.
\newcommand{\wext}{\ensuremath{\mathcal{W}_{\mathrm{ext}}}}
\newcommand{\wjun}{\ensuremath{\mathcal{W}_{\mathrm{jun}}}}
\newcommand{\ssem}{\ensuremath{\mathrm{sem}}}

\newcommand{\dist}[2]{||{#1},{#2}||}

\newcommand{\tuple}[1]{\ensuremath{\langle #1 \rangle}}

\newcommand{\lp}{{\rm (}}
\newcommand{\rp}{{\rm )}}
\newcommand{\paren}[1]{\lp{#1}\rp}
% Graph Cartesian product
\newcommand{\gcp}{\Box}
% A field in a record (ADT)
\newcommand{\field}[1]{\mathtt{#1}}

\newcommand{\floor}[1]{\left\lfloor{#1}\right\rfloor}
\newcommand{\ceil}[1]{\left\lceil{#1}\right\rceil}

\newcommand{\action}[1]{\ensuremath{\mathsf{#1}}}

\newcommand{\degs}{{^\circ}}

\definecolor{PaleRed}{RGB}{202,61,58}
\definecolor{DeepRed}{RGB}{158,46,43}

\newcommand{\figDataContainer}[2]{%
\coordinate (a) at #1;
\coordinate (b) at ($(a) + (0.15,0.15)$);
\coordinate (c) at ($(b) + (0.15,0.15)$);

\shade [top color=PaleRed,bottom color=DeepRed]
($(c) + (0,0)$) -- ($(c) + (0,-2)$)
                to [out=340,in=180] ($(c) + (1,-2.2)$)
                to [out=0,in=170] ($(c) + (4,-1.5)$)
                -- ($(c) + (4,0)$);
                
\draw [very thick, opacity=0.1]
($(c) + (0,0)$) -- ($(c) + (0,-2)$)
                to [out=340,in=180] ($(c) + (1,-2.2)$)
                to [out=0,in=170] ($(c) + (4,-1.5)$)
                -- ($(c) + (4,0)$) 
                -- ($(c) + (0,0)$); 
                
\shade [top color=PaleRed,bottom color=DeepRed]
($(b) + (0,0)$) -- ($(b) + (0,-2)$)
                to [out=340,in=180] ($(b) + (1,-2.2)$)
                to [out=0,in=170] ($(b) + (4,-1.5)$)
                -- ($(b) + (4,0)$);
                
\draw [very thick, opacity=0.1]
($(b) + (0,0)$) -- ($(b) + (0,-2)$)
                to [out=340,in=180] ($(b) + (1,-2.2)$)
                to [out=0,in=170] ($(b) + (4,-1.5)$)
                -- ($(b) + (4,0)$) 
                -- ($(b) + (0,0)$); 
                
\shade [top color=PaleRed,bottom color=DeepRed]
($(a) + (0,0)$) -- ($(a) + (0,-2)$)
                to [out=340,in=180] ($(a) + (1,-2.2)$)
                to [out=0,in=170] ($(a) + (4,-1.5)$)
                -- ($(a) + (4,0)$);
                
\draw [thick, opacity=0.3]
($(a) + (0,0)$) -- ($(a) + (0,-2)$)
                to [out=340,in=180] ($(a) + (1,-2.2)$)
                to [out=0,in=170] ($(a) + (4,-1.5)$)
                -- ($(a) + (4,0)$) 
                -- ($(a) + (0,0)$); 
                
\node at ($(a) + (2,-1)$) {\color{white}{#2}};
}

\newcommand{\figProcessingComponent}[2]{%
\coordinate (a) at #1;
\shade [top color=NavyBlue,bottom color=RoyalBlue]
($(a) + (0,0)$) -- ($(a) + (0,-1)$)
                -- ($(a) + (1,-1)$)
                -- ($(a) + (1,0)$);
\draw [thick, opacity=0.4]
($(a) + (0,0)$) -- ($(a) + (0,-1)$)
                -- ($(a) + (1,-1)$)
                -- ($(a) + (1,0)$)
                -- ($(a) + (0,0)$);
                
\node at ($(a) + (0.5,-0.5)$) {\color{white}{#2}};
}

\newcommand{\figFilter}[2]{%
\coordinate (a) at #1;


\coordinate (a) at #1;

\shade [top color=LimeGreen,bottom color=OliveGreen]
($(a) + ({0.5+sin(45)/2}, {-0.5+(sin(45)/2)})$) arc [radius=0.5,start angle=45,end angle=315] -- ++($(0,{cos(45)})$);

\draw [thick, opacity=0.4]
($(a) + ({0.5+sin(45)/2}, {-0.5+(sin(45)/2)})$) arc [radius=0.5,start angle=45,end angle=315] -- ++($(0,{cos(45)})$);

\node at ($(a) + (0.5,-0.5)$) {\color{white}{#2}};
}
\newcommand{\co}{\mathbf{Co}}
\newcommand{\cansend}[2]{#1 \rightarrowtail \{#2\}}
\newcommand{\cantsend}[2]{#1 \multimap \{#2\}}
\newcommand{\canrec}[2]{\{#1\} \rightarrowtail #2}
\newcommand{\cantrec}[2]{\{#1\} \multimap #2}
\newcommand{\rec}[1]{\field{rec}(#1)}
\newcommand{\ft}[1]{\tt{ft}_{#1}}
\renewcommand{\int}[1]{\tt{int}_{#1}}
\newcommand{\proc}[1]{\tt{proc}_{#1}}
\newcommand{\sends}[3]{#1 \rightarrow [#2] \rightarrow #3}
\newcommand{\sendsm}[4]{#1 \rightarrow [#2 , #3] \rightarrow #4}
\newcommand{\sendsf}[6]{#1\langle #2 \rangle \rightarrow [#3,#4] \rightarrow \langle#5\rangle#6}

\newcommand{\compT}[2]{\tt{Comp}_{#1,#2}}

\newcommand{\field}[1]{\mathtt{#1}}

\DeclareMathOperator{\tint}{int}
\newcommand{\type}[1]{\mathtt{#1}}
\DeclareMathOperator{\cint}{cint}

\newcommand{\action}[1]{\ensuremath{\mathsf{#1}}}
\definecolor{PaleRed}{RGB}{202,61,58}
\definecolor{DeepRed}{RGB}{158,46,43}
\definecolor{PaleBlue}{RGB}{100,180,250}
\definecolor{DeepBlue}{RGB}{30,100,255}
\definecolor{DeepGreen}{RGB}{60,150,30}
\definecolor{LightGray}{RGB}{220,220,220}

\newcommand{\figDataContainer}[2]{%
\coordinate (a) at #1;
\coordinate (b) at ($(a) + (0.15,0.15)$);
\coordinate (c) at ($(b) + (0.15,0.15)$);

\shade [top color=PaleRed,bottom color=DeepRed]
($(c) + (0,0)$) -- ($(c) + (0,-2)$)
                to [out=340,in=180] ($(c) + (1,-2.2)$)
                to [out=0,in=170] ($(c) + (4,-1.5)$)
                -- ($(c) + (4,0)$);
                
\draw [very thick, opacity=0.1]
($(c) + (0,0)$) -- ($(c) + (0,-2)$)
                to [out=340,in=180] ($(c) + (1,-2.2)$)
                to [out=0,in=170] ($(c) + (4,-1.5)$)
                -- ($(c) + (4,0)$) 
                -- ($(c) + (0,0)$); 
                
\shade [top color=PaleRed,bottom color=DeepRed]
($(b) + (0,0)$) -- ($(b) + (0,-2)$)
                to [out=340,in=180] ($(b) + (1,-2.2)$)
                to [out=0,in=170] ($(b) + (4,-1.5)$)
                -- ($(b) + (4,0)$);
                
\draw [very thick, opacity=0.1]
($(b) + (0,0)$) -- ($(b) + (0,-2)$)
                to [out=340,in=180] ($(b) + (1,-2.2)$)
                to [out=0,in=170] ($(b) + (4,-1.5)$)
                -- ($(b) + (4,0)$) 
                -- ($(b) + (0,0)$); 
                
\shade [top color=PaleRed,bottom color=DeepRed]
($(a) + (0,0)$) -- ($(a) + (0,-2)$)
                to [out=340,in=180] ($(a) + (1,-2.2)$)
                to [out=0,in=170] ($(a) + (4,-1.5)$)
                -- ($(a) + (4,0)$);
                
\draw [thick, opacity=0.3]
($(a) + (0,0)$) -- ($(a) + (0,-2)$)
                to [out=340,in=180] ($(a) + (1,-2.2)$)
                to [out=0,in=170] ($(a) + (4,-1.5)$)
                -- ($(a) + (4,0)$) 
                -- ($(a) + (0,0)$); 
                
\node at ($(a) + (2,-1)$) {\color{white}{#2}};
}

\newcommand{\figProcessingComponent}[2]{%
\coordinate (a) at #1;
\shade [top color=PaleBlue,bottom color=RoyalBlue]
($(a) + (0,0)$) -- ($(a) + (0,-1)$)
                -- ($(a) + (1,-1)$)
                -- ($(a) + (1,0)$);
\draw [thick, opacity=0.4]
($(a) + (0,0)$) -- ($(a) + (0,-1)$)
                -- ($(a) + (1,-1)$)
                -- ($(a) + (1,0)$)
                -- ($(a) + (0,0)$);
                
\node at ($(a) + (0.5,-0.5)$) {\color{white}{#2}};
}

\newcommand{\figFilter}[2]{%
\coordinate (a) at #1;


\coordinate (a) at #1;

\shade [top color=LimeGreen,bottom color=OliveGreen]
($(a) + ({0.5+sin(45)/2}, {-0.5+(sin(45)/2)})$) arc [radius=0.5,start angle=45,end angle=315] -- ++($(0,{cos(45)})$);

\draw [thick, opacity=0.4]
($(a) + ({0.5+sin(45)/2}, {-0.5+(sin(45)/2)})$) arc [radius=0.5,start angle=45,end angle=315] -- ++($(0,{cos(45)})$);

\node at ($(a) + (0.5,-0.5)$) {\color{white}{#2}};
}

\newcommand{\figProcessingComponentBent}[2]{%
\coordinate (a) at #1;

\shade [top color=PaleBlue,bottom color=RoyalBlue]
($(a) + (0.1,0)$) -- ($(a) + (1.35,0)$)
                  -- ($(a) + (1.5,-0.5)$)
                  -- ($(a) + (1.35,-1)$)
                  -- ($(a) + (0.1,-1)$)
                  to [out=115,in=270] ($(a) + (0.0,-0.5)$)
                  to [out=90,in=245] ($(a) + (0.1,0)$);
                  
\draw [thick, opacity=0.4]
($(a) + (0.1,0)$) -- ($(a) + (1.35,0)$)
                  -- ($(a) + (1.5,-0.5)$)
                  -- ($(a) + (1.35,-1)$)
                  -- ($(a) + (0.1,-1)$)
                  to [out=115,in=270] ($(a) + (0.0,-0.5)$)
                  to [out=90,in=245] ($(a) + (0.1,0)$);

\node at ($(a) + (0.75,-0.5)$) {\color{white}{#2}};
}

\newcommand{\figProcessingComponentBentUp}[2]{%
\coordinate (a) at #1;

\shade [top color=PaleBlue,bottom color=RoyalBlue]
($(a) + (0,-1.4)$) -- ($(a) + (0,-0.15)$)
                  -- ($(a) + (0.5,0)$)
                  -- ($(a) + (1,-0.15)$)
                  -- ($(a) + (1,-1.4)$)
                  to [out=205,in=0] ($(a) + (0.5,-1.5)$)
                  to [out=180,in=335] ($(a) + (0,-1.4)$);
                  
\draw [thick, opacity=0.4]
($(a) + (0,-1.4)$) -- ($(a) + (0,-0.15)$)
                  -- ($(a) + (0.5,0)$)
                  -- ($(a) + (1,-0.15)$)
                  -- ($(a) + (1,-1.4)$)
                  to [out=205,in=0] ($(a) + (0.5,-1.5)$)
                  to [out=180,in=335] ($(a) + (0,-1.4)$);

\node at ($(a) + (0.5,-0.75)$) {\color{white}{#2}};
}

\newcommand{\cubeShape}[2]{%
		\coordinate (a) at #1;
	
		\fill [color=#2, opacity=0.4] (a) -- ++(0,1)
								     -- ++($({sqrt(2)/2},{sqrt(2)/2})$)
								     -- ++(1,0)
								     -- ++(0,-1)
								     -- ++($({-sqrt(2)/2},{-sqrt(2)/2})$);
		
		\draw [opacity=0.3] (a) -- ++(0,1)
								-- ++($({sqrt(2)/2},{sqrt(2)/2})$)
								-- ++(1,0)
								-- ++(0,-1)
								-- ++($({-sqrt(2)/2},{-sqrt(2)/2})$)
								-- ++(-1,0);
		\draw [opacity=0.3] (a) -- ++($({sqrt(2)/2},{sqrt(2)/2})$)
								-- ++(1,0);
		\draw [opacity=0.3] ($(a) + (0,1)$) -- ++(1,0)
							           	    -- ++($({sqrt(2)/2},{sqrt(2)/2})$);
		\draw [opacity=0.3] ($(a) + ({sqrt(2)/2},{sqrt(2)/2})$) -- ++(0,1);
		\draw [opacity=0.3] ($(a) + (1,0)$) -- ++(0,1);
}

% Simple cubes with a label

\newcommand{\figSimpleCube}[3]{%
	\coordinate (a) at #1;
	\shade [top color=#3,bottom color=#3,fill opacity=0.5]
	($(a) + (0,0)$) -- ($(a) + (0,-1)$)
	-- ($(a) + (1,-1)$)
	-- ($(a) + (1,0)$);
	%\draw [thick, opacity=0.4]
	%($(a) + (0,0)$) -- ($(a) + (0,-1)$)
	%-- ($(a) + (1,-1)$)
	%-- ($(a) + (1,0)$)
	%-- ($(a) + (0,0)$);
	
	\node at ($(a) + (0.5,-0.5)$) {\color{black}{#2}};
}

\newcommand{\figBlueCube}[2]{\figSimpleCube{#1}{#2}{PaleBlue}}
\newcommand{\figDarkGreenCube}[2]{\figSimpleCube{#1}{#2}{DeepGreen}}
\newcommand{\figRedCube}[2]{\figSimpleCube{#1}{#2}{PaleRed}}

\newcommand{\cubeLabel}[2]{\begin{tabular}{c}\textbf{``#1'':}\\\begin{tabular}{l}#2\end{tabular}\end{tabular}}
%\newcommand{\stdlisting}{%
\lstset{
			mathescape,							%Damit der Math-Modus funktioniert
			language={},							%Sprache
			keywordstyle=\color{KeywordColor},	%Farbe f. Keywords
			commentstyle=\color{CommentColor},	%Farbe f. Kommentare
			stringstyle=\color{StringColor},	%Farbe f. Stringliterale
			morekeywords={procedure},			%Zus. Keywords (zu language)
			otherkeywords={},
			basicstyle=\textnormal\ttfamily\footnotesize,	%Textstil
			breaklines=true,					%Umbruch langer Zeilen
												%(sonst hängen sie über)
			escapechar=´,						%Alles zwischen 2 ´ ist
												%außerhalb der lstlistings-Umgebung
												%Kommentare im Code sind möglich mit:
												%´\comment{blabla}´
			frame=none,						%Art des Rahmens
			captionpos=t,						%Position des Titels
			tabsize=4,							%Tabs durch Spaces ersetzen
			showtabs=false,						%Bei true werden Tabs im Code mit kleinen
												%Futzeln angezeigt
			showspaces=false,					%Dasselbe mit Spaces
			showstringspaces=false				%Dasselbe mit Spaces in Stringliteralen
			numberstyle=\footnotesize,
			numbersep=5pt,
			%backgroundcolor = \color{white},
			stepnumber=1
			}%
}

\newcommand{\code}[1]{\Fcolorbox{Gray}{LightGray}{#1}}




% Define convenience functions to use the author name and the thesis title in the PDF document properties.
\newcommand{\authorname}{Janos Tapolczai} % The author name without titles.
\newcommand{\thesistitle}{Title of the Thesis} % The title of the thesis. The English version should be used, if it exists.

% Set PDF document properties	
\hypersetup{
	pdfpagelayout   = TwoPageRight,           % How the document is shown in PDF viewers (optional).
	linkbordercolor = {Melon},                % The color of the borders of boxes around crosslinks (optional).
	pdfauthor       = {\authorname},          % The author's name in the document properties (optional).
	pdftitle        = {\thesistitle},         % The document's title in the document properties (optional).
	pdfsubject      = {Subject},              % The document's subject in the document properties (optional).
	pdfkeywords     = {a, list, of, keywords} % The document's keywords in the document properties (optional).
}

\setsecnumdepth{subsection} % Enumerate subsections.

\nonzeroparskip             % Create space between paragraphs (optional).
\setlength{\parindent}{0pt} % Remove paragraph identation (optional).

\makeindex      % Use an optional index.
% \makeglossaries % Use an optional glossary.
%\glstocfalse   % Remove the glossaries from the table of contents.

% Set persons with 4 arguments:
%  {title before name}{name}{title after name}{gender}
%  where both titles are optional (i.e. can be given as empty brackets {}).
\setauthor{BSc.}{\authorname}{}{male}
\setadvisor{a.o. Univ.-Prof. Dr.}{Hans Tompits}{}{male}

% For bachelor and master theses:
\setfirstassistant{Pretitle}{Forename Surname}{Posttitle}{male}
\setsecondassistant{Pretitle}{Forename Surname}{Posttitle}{male}
\setthirdassistant{Pretitle}{Forename Surname}{Posttitle}{male}

% Required data.
\setaddress{Address}
\setregnumber{0825077}
\setdate{01}{01}{2016} % Set date with 3 arguments: {day}{month}{year}.
\settitle{\thesistitle}{Design and Implementation of an Agent Architecture combining Emotions and Reasoning}
% \setsubtitle{Optional Subtitle of the Thesis}{Optionaler Untertitel der Arbeit} % Sets English and German version of the subtitle (both can be English or German).

% Select the thesis type: bachelor / master / doctor / phd-school.
% Bachelor:
\setthesis{master}
\setmasterdegree{dipl.} % dipl. / rer.nat. / rer.soc.oec. / master

% For bachelor and master:
\setcurriculum{Computational Intelligence}{Computational Intelligence} % Sets the English and German name of the curriculum.

\begin{document}
	
	\frontmatter % Switches to roman numbering.
	% The structure of the thesis has to conform to
	%  http://www.informatik.tuwien.ac.at/dekanat
	
	\addtitlepage{naustrian} % German title page (not for dissertations at the PhD School).
	% \addtitlepage{english} % English title page.
	\addstatementpage
	
	\begin{acknowledgements*}
		This thesis is the result of two years of work.
		
		
		I would like to thank my mother for tirelessly pursuing my education from my early years on, instilling in me the value of bettering oneself. Without her tremendous efforts, I would doubtlessly not be where I am today.
		
		matthias \"{O}sterreicher
		I would also like my good friends J\"{u}rgen Cito, Manuel Mertl, Bernd Rathmanner, Gerald Schermann for their support and friendship through these years.
		
		Furthermore, I would like to sincerely thank my professors Hans Tompits and Rudolf Freund for their advice, and their forbearance in discussing my admittedly idiosyncratic ideas at length.
	\end{acknowledgements*}
	
	\begin{abstract}
		We design and implement a hybrid AI that combines emotional reactions to stimuli with reasoning about the consequences of actions as a means of developing effective behaviour in a toy world. The fundamental task of our agents is to survive by periodically finding and eating food and by dealing with hostile predators, either through flight or through flight. Agents are free to cooperate, antagonize, or ignore each other and agents with different emotional profiles will pursue different strategies.
		
		The thesis begins with theoretical investigations of different models of computations and their relation to the biological brain. Our proposition is that the brain's function is, to a degree, analogous to a collection of white boxes, observable to each other. Accordingly, our agents are modeled as a collection of loosely coupled components which communicate with each other through messages. Any component is free to read any message and components have no knowledge which others will read messages they insert into the agent's message space.
		
		A relatively simple affective evaluation forms the basis of each agent's decision-making, though it is complemented by a belief generator which makes inferences about future world-states resulting from certain choices. Though evaluation of these future states as well, agents can optimize their behavior, as they can foresee likely positive or negative consequences of their actions.
		
		The thesis ends will a qualitative and a quantitative evaluation. We qualitatively evaluate agents qualitatively by placing agents in a number of simple scenarios and observing whether they perform tasks like collecting food or avoiding predators. We then evaluate them quantitatively by placing various populations into larger scenarios and recording the survival of different personality types over time.
	\end{abstract}
	
	\begin{kurzfassung}
		Wir entwerfen und implementieren eine hybride AI, welche emotionale Reaktionen auf Stimuli mit dem Schließen über den Folgen von Aktionen zum Zwecke der Ausarbeitung effektiven Verhaltens in einer einfachen Welt kombiniert. Die fundamentale Aufgabe unserer Agenten ist es, durch das Finden und Essen von Nahrung, sowie der Handhabung von Raubtieren durch Kampf oder Flucht, zu überleben.
		
		Die Arbeit beginnt mit einer theoretischen Untersuchung mehrerer komputationaler Modelle und ihrem Verhältnus zum biologischen Gehirn. Unserer Behauptung ist, dass das Gehirn zu einem gewissen Grad analog zu einer Ansammlung von \emph{White Boxes} ist, die die Arbeit der jeweils anderen beobachten können. Demgemäß sind unsere Agenten als lose gekoppelte Komponenten modelliert, die über Nachrichten miteinander kommunizieren. Jede Komponente kann jede Nachricht auslesen und keine Komponente weiß, welche anderen die Nachrichten lesen werden, welche sie in den zentralen \emph{Message Space} des Agenten einfügt.
		
		Eine relativ einfache affektive Evaluation ist die Grundlage der Entscheidungsfindung der Agenten, sie wird aber ergänzt durch einen \emph{Belief Generator}, der zukünftige Zustände der Welt als Folge bestimmter Aktionen simuliert. Durch die Evaluation dieser zukünftigen Zustände können Agenten ihr Verhalten anpassen und sowohl die negativen als auch die positiven Folgen ihrer Aktionen absehen.
		
		Die Arbeit endet mit einer qualitativen sowie einer quantitativen Evaluierung. Wir evaluieren die Agenten qualitativ, indem wir sie in einer reihe simpler Beispielszenarien ausprobieren und testen, ob sie einfache Aufgaben, wie das Sammeln von Essen oder die Flucht von Feinden, ausführen. Danach evaluieren wir sie quantativ mittels verschiedener Populationen in eine größeren Szenario, in dem wir die langfristige Überlebensrate verschiedener Persönlichkeitstypen festhalten.
	\end{kurzfassung}
	
	% Select the language of the thesis, e.g., english or naustrian.
	\selectlanguage{english}
	
	% Add a table of contents (toc).
	\tableofcontents % Starred version, i.e., \tableofcontents*, removes the self-entry.
	
	% Switch to arabic numbering and start the enumeration of chapters in the table of content.
	\mainmatter
	
	\chapter{Introduction}
	
	The history of AI is marked by vacillations between two paradigms: the biological and the ideal one. These sometimes move closer to each other, and then apart again, due to the fact that they are naturally in tension, yet also inextricably bound together. The biological model, which subsumes the connectionist and the cybernetic ones, seeks to imitate biological organisms on the lowest level; cybernetics, now unpopular, tried to emulate intelligence via feedback loops; connectionism wants to  build complex cognition out of interconnected, simple parts. Its most famous instance --- neural networks --- and the approach in general, have, time and time again, failed to produce even the rudimentary intelligence we seen in animals. The high hopes of the pioneers of the field that the mathematical modelling of some primitive, biologically inspired computing machinery, cleverly pieced together, could mimic the cognition of humans, were sorely disappointed. Then, after years of lacklustre results, Marvin Minsky published his devastating proof of the theoretical weakness of perceptrons --- a then popular type of neural network --- in his book {\em Perceptrons: An Introduction to Computational Geometry} \cite{Minsky1988}. This led to the abandonment of the approach among AI researchers and, arguably, to the AI winter of the 1970s. In hindsight, we can say that, despite their unimpeachable brilliance, their position in history imposed upon them a fatal naivete and an optimism born from the ignorance of the harrowing complexity of an organ that had been, in one form or another, a good few hundred million years in the making. The idea that neural networks mimicking the smallest-scale structures in the brain and consisting of few thousand nodes, trained over the course of minutes or days, could hope to emulate human intelligence, turned out to be false.

Though it had been developed coterminously with the connectionist approach, the ideal approach came to prominence after the former's disappointments. We can identify it largely with {\em symbolic} computation, whose proponents falls into the two camps (allegedly first identified by Roger Schank): the {\em neats}, who favour provably correct methods, and the {\em scruffies}, who are willing to use whatever works \cite[pp. 421–424]{McCorduck2004}. Both, to some degree, kept a penchant for mathematical formalisms, but they abandoned the goal of re-tracing the workings of biological brains. The neats especially, in an Aristotelian manner, tried to automate how one {\em ought to think}. How humans accomplished their tasks was no longer of importance; only what they accomplished was --- and preferably, that {\em what} was abstract reasoning. Out of this school thought came many applications that have proved very useful: automated planning, in the forward and backward variety; default and common-sense reasoning; answer-set and logic programming; Prolog; knowledge-based systems that assist experts in their decision-making, in a way, informed an uninformed search algorithms. Many of these can now outperform the best humans in specialized tasks. Knowledge-based in medicine now rival doctors in the quality of their diagnoses \cite{bla}. Automated driving system can navigate through traffic accident-free \cite{googleCar}. In chess, the greatest human players have trouble keeping up with computers, as Gary Kasparov famous loss to Deep Blue showed in 1997 \cite{deepBlue}. Yet, impressive as they are --- and they are impressive ---, these {\em weak AIs} have not allowed us to piece together the big picture.\footnote{As a side note, it should be said that that the term ``weak AI'' may be unfairly denigrating. Even if we came into possession of human-level strong AI, so-called weak  AIs would still easily outperform it in special areas, just as they outperform humans today.} As good as they are, one would be hard-pressed to call them intelligent in the colloquial sense. As good as any chess program or automated car is, they are still just machines (again in the colloquial sense). It would indeed be absurd to propose that one could exchange one for the other. While Gary Kasparov was able to drive home after his match against IBM's Deep Blue, the notion of ``car'' was not even part of his opponent's conceptual universe. Equally, the control system in Google's self-driving car would not be able to play chess, or assemble a shopping list, read a book, smell, interpret emotions, or do any number of things of which humans and animals are capable.

In recent years, the schism between the biological/connectionist and the ideal/symbolic schools of thought has grown quite pronounced, they are, on a deeper level, joined at the hip. The simple reason for this is that the brain is both a ``messy'' biological organ, and a high-level computation device. It is neither an amorphous mass of neurons, as connectionism models it, nor is it merely a neutral hardware for idealized reasoning, to which the symbolic approach is wont to reduce it. It is both, and in this thesis, I submit that any strong AI must find find a bridge between the two views and integrate them into one whole.

This thought is not new; such a bridge already exists in the form of the {\em integrated approaches}. This vague category comprises things like James Albus's Hierarchical Control Systems \cite{albusHCS}, Rodney Brook's Subsumption Architectures \cite{brooksSubsumption}, Carnegie-Mellon's 4CAPS \cite{4caps}, and, in theory, the architecture laid out in Minsky's The Emotion Machine \cite{emotionMachine}. The chief commonality of these systems is that they proceed from an engineering perspective, not from the mathematical one of symbolic computation, or the biological one of connectionism. Yet, they attempt to combine the high-quality results of the mathematical methods with the dynamism of the biologically inspired systems. It is to this category to which I hope to make a small contribution by proposing that one must not only consider the mind not just as messy, but something worse: as {\em evolved}. I submit that we can best make sense of the seemingly random jumble of features, defects, idiosyncrasies, and quirks of brains by tracing their history. By going through the developments step-by-step, by proposing simple components that came about one by one and grew over time, what is seemingly random and senseless can begin to make sense. By re-creating them, we can hope to create {\em authentic} intelligences --- ones that match biological ones not just in raw processing power, but also in kind.

\paragraph{Structure of thesis.} After Section~\ref{sec:relatedWork} \textbf{Related work}, the thesis consists of two large segments. The first one is the theoretical argument and empirical data supporting it. It deals with the origin and purpose of neural systems, their evolution (insofar as is known), the components of which they are likely comprised, and how these could have come about. The constituent sections are
	\begin{itemize}
		\item Section~\ref{sec:preliminaries} \textbf{Preliminary considerations}, containing the general evolutionary story,
		\item and Section~\ref{sec:schemaOfCognition} \textbf{Schema of cognition}, which hypothesizes about the cognitive structure of humans.
		\item After that, Section~\ref{sec:mathematicalModel} \textbf{Mathematical notation}, in which we introduce a bit of notation to ease talking about the systems, and
		\item Section~\ref{sec:selectedSubsystems} \textbf{Selected subsystems}, in which a number of proposed subsystems are sketched.
	\end{itemize}
	
The second segment puts the material of the first into practice and consists of Section~\ref{sec:proposedArchitecture} \textbf{Proposed architecture} and Section~\ref{sec:implementation} \textbf{Implementation}. Therein, I describe a kind of affective agent architecture that utilizes both emotions and reasoning to navigate a toy world populated by others like it. The last section contains the result of that experiment.

I would like to remark that everything in this work is, at best, a rough outline;: though (conjectured to be) basically correct, and, I hope, useful, we do simply not have the huge amount of data about real brains to construct truly faithful models of them, or to verify even hypothesis in detail.
	
	\chapter{Preliminary Considerations}\label{ch:preliminaryConsiderations}
	
	We begin this chapter with a look at related work in the field of artificial intelligence: approaches, architectures, implementations, and philosophies on which later parts of the thesis draw. After that, in Section~\ref{sec:preliminaryBiologicalConsiderations}, we will go through biological and neurological considerations that shall ground the models and architectures proposed herein. Lastly, Section~\ref{sec:whiteBoxModel1} will outline a white-box model of cognition which assumes various parts of a central nervous system observing and influencing each other relatively freely. We specifically contrast this with the black-box model of structured programming, where the inner workings of functions and procedures are oblique to the caller.

\section{Related Work}\label{sec:relatedWork}

\paragraph{Cognitive architectures.} This thesis falls into the category of cognitive architectures and the integrated approach to AI, pioneered by people like Rodney Brooks and his subsumption architecture, and \cite{brooksSubsumption}, Douglas Hofstadter, who famously wrote about many aspects of AI in \emph{Gödel, Escher, Bach} \cite{geb}, and who created the \emph{Copycat} analogy-making program \cite{copycat}. Another important work is the \emph{Hierarchical Control System} of James Albus \cite{albusHCS}, in which cognitive tasks are organised hierarchically and delegated by nodes on higher levels to those on lower ones (this is similar to the mesh-like organisation of components described in Section~\ref{sec:mathematicalModel}, and to the layered structure of Minsky's {\em The Emotion Machine} \cite{emotionMachine}). The organisation described by Albus \cite{albus93areference} is, moreover, very similar to the one in Section~\ref{ch:implementation}, with world simulator, belief generator, sensory perception, and knowledge base (herein called ``memory'') modules being mostly analogous. Another large and conceptually similar system is Carnegie-Mellon's 4CAPS \cite{4caps}, which posits small, relatively simple components, individually doing simple tasks, and having only limited computational resources. Most of 4CAPS's stated principles can be recognised in the coming sections \cite[Operating Principles of 4CAPS]{4caps}:
\begin{emquote}
	\begin{enumerate}
		\setcounter{enumi}{-1}
		\item Thinking is the product of the concurrent activity of multiple brain areas that collaborate in a large-scale cortical network. \ellipses
		\item Each cortical area can perform multiple cognitive functions, and conversely, many cognitive functions can be performed by more than one area.
		\item Each cortical area has a limited capacity of computational resources, constraining its activity.
		\item The topology of a large-scale cortical network changes dynamically during cognition, adapting itself to the resource limitations of different cortical areas and to the functional demands of the task at hand.
		\item The communications infrastructure that supports collaborative processing is also subject to resource constraints, construed here as bandwidth limitations.
	\end{enumerate}
	
	\quad\quad\ \ellipses
\end{emquote}

The probably earliest example of a cognitive architecture was Allen Newell's and Herbert~A.~Simon's {\em Logic Theorist}, created in 1955 \cite[p.\ 44]{crevier93}. Simon's theory of bounded rationality \cite{Gigerenzer2001} --- the idea of finding a merely satisfactory solution instead of a (provably) optimal one --- is very similar to the loop between belief generation and evaluation described in Section~\ref{ch:implementation}. In both cases, agents with limited information search heuristically for the first solution that they find acceptable. Unlike exhaustive search methods (e.g. A*), this does not guarantee the best possible results, but it is much more cost-effective and closer to the way real humans solve problems. In spirit, this is also similar to the {\em Procedural Reasoning System} of Michael Georgeff et al. \cite{pcs}, which is based on the belief-desire-intention (BDI) model\cite{Rao95bdiagents, Bratman87}. Much theoretical work has been done on BDI, but it is only tangentially related to this thesis.

\paragraph{Sloman.} Many of the fundamental ideas in this thesis can be found in Alan Sloman's works \cite{sloman1993, sloman1997, sloman1999, sloman2000, slomanSimAgent}, especially in {\em Beyond shallow models of emotion} \cite{sloman2000}. Therein, he formulated the criterion of evolvability in the context of cognitive architectures and postulated the possibility that nervous systems may be chaotic (but not unorganised). The agent architecture in Section~\ref{ch:implementation} substantially resembles his, though it was not taken from there. The similarity is, however, indicative of a great deal of shared thought.

\paragraph{Implementation.} In terms of software engineering, our model has similarities, both to the Actor model developed by Carl Hewitt et al.\ \cite{hewittActor}, and to publish/subscribe architectures \cite{publishSubscribe} --- although more as a concession to practicality and less because of a similarity to their theories. The theoretical basis of our implementation is the postulate that the components of the brain function as white boxes and that other components may listen in on their activity, so to speak. Since this is diametrically opposed to the traditional idea of the procedure/function as a black box, which nigh every programming language follows, we compromise and model the cognitive structure as a mesh of loosely coupled components communicating via passing. This description is reminiscent to the Actor model, although there are differences\footnote{Note that we do not describe the implementation in the language of the Actor model, but that a translation into it would be quite easy. Such a translation would require using only very rudimentary features of the model, however, and, as that is not the focus, we forego the task.}: in the Actor model, the topology of the network may change through the creation of new actors, and messages are always passed from one source to known targets (via addresses). In our model, on the other hand, there is no topology in a strict sense; messages are put into a global message storage and every component is free to consume any message it deems relevant. Senders do not know who will read their output, and consumers do not know the sources. This arrangement can be seen as a particularly loose variant of a publish/subscribe architecture, in which the source and the target of a message are completely unaware of each other, and in which there are no specific channels to which one may subscribe. The only criterion by which messages may be accepted or rejected is their content.

% We also make use of already existing solutions --- specifically answer-set programming and the \acthex\ solver \dlvhex. The internal world simulation of our agents makes use of the non-monotonic reasoning provided by ASP and \acthex.  Answer-set programming was created by Gelfond and Lifschitz \cite{asp1}. Soon after them, Subrahmanian made the connection between ASP and planning \cite{Subrahmanian95relatingstable}. Together with Eiter and others, he later developed the \acthex\ language which allowed provided a framework for decision making in logic programming via external input and output atoms \cite{heterogeneous1, heterogeneous2, heterogeneous3}.

\paragraph{Nouvelle AI.} Lastly, the overall goal, if not the method, of this thesis echoes that of the {\em nouvelle AI} of, again, Brooks \cite{Brooks91intelligencewithout}, who claims that
\begin{emquote}
	the Von Neumann model of computation has lead Artificial Intelligence in particular directions. Intelligence in biological systems is completely different. 
\end{emquote}
The nouvelle AI approach stands in contrast to traditional AI in that it does not aim for human-level performance at specific tasks, but rather for the faithful reproduction of the behaviour of lower animals like dogs \cite{nouvelleAI}. Brooks might be closer to the biological realities in his desire to abandon the von Neumann model in favour of biologically modelled computation, though we will take only general inspiration from his approach, not follow it closely. As our goal is merely a proof-of-concept implementation, and since the realization of truly novel programming, biologically oriented, paradigms is quite laborious, we opted for a compromise position and only tried to imitate biological computation in general spirit rather than in every detail.

\section{Preliminary Biological Considerations}\label{sec:preliminaryBiologicalConsiderations}

In this section, we will go over the foundational ideas that, while serviceable on their own, will underlie the work in the second part of this work. The information will primarily concern biology, computational models, and the brain as a product of evolution. Biologists will find all of it terribly basic, but this document is not intended for them; it is intended for computer scientists --- who, we feel, have not truly taken to heart the consequences of the routes our nervous systems have taken through history for their present state. Sure enough, we have things {\em called} ``evolutionary algorithms'' and ``machine learning'', but names such as these invite us to a perilous confusion of labels with the real things. Although such mathematical abstractions may be inspired by biological processes, they are not the equivalents of these processes. Recreating the end-products of biology demands an understanding of biology on its own terms, not through the lens of misguidingly named mathematical abstractions. Providing the basis of such an understanding will be our aim for the next couple of pages.

\paragraph{Historical and designed artefacts.} In order to understand how our brain works or could work, we must possess conceptual clarity --- we must conceive of it, not as a product of one-time engineering, but as a historical artefact. Unlike ``perfect'' systems, like Peano arithmetic and the $\lambda$-calculus, those which grew historically does not make sense if one only looks at their current snapshot. One will find nonsensical solutions, and attempts to mitigate the consequences of earlier designs that have now become disadvantageous. The system as a whole might, at first glance, appear incomprehensibly and needlessly complicated. Of all such systems, the human brain might well be the most complex one; the task of understanding it correspondingly harrowing. Sloman asked whether the brain might have no architecture at all  \cite[p.\ 5]{sloman1997}:

\begin{emquote}
	Another question on which there is disagreement is whether the provision of a large set of capabilities, such as those listed above, necessarily involves the creation of an {\em intelligible} design, with identifiable components performing separate tasks, or whether the functionality could sometimes \paren{or always?} emerge only in a very complex and incomprehensible fashion from myriad interacting components.
	
	For example, experimenters using genetic algorithms to evolve neural sets to control a robot sometimes create networks that work, but which seem to be impossible to understand \paren{not unlike some legacy software which has grown over many years of undisciplined developments}.
\end{emquote}

It the classical sense, it probably does not, but we ought to be cognisant that ``the classical sense'' was induced by tradition and the limits of human cognitive ability. We might dismissively describe the brain as a jumbled, tangled chaos of neurons, but the fact that we do not recognise a structure by no means implies that one does not exist. Things, contrary to what is often espoused, do not ``just work''; if they reliably produce complex results, they must have an architecture inside them, independent of our ability to recognise or understand it as such. We merely need to relax the notion of ``architecture'' to include structures that result from incremental change and the creative combination of pre-existing parts. While the results of such processes are often extremely unintuitive and often even incomprehensible to us, we at least have a way of understanding them by re-tracing their evolution. Doing so is laborious and requires a huge amount of data (which we currently do not have), but this approach of regarding brain functions as through-and-through Darwinian (as opposed to having been pieced together) might bear results that have, so far, eluded the other schools of thought in the field.

What, one might now ask, is the consequence of such a view? The first is that each new feature in the developmental history had to have been useful on its own. The second is that it allows the distinction between what we will herein call \emph{efficient} systems and \emph{clean} systems. Since, at each stage of its evolution, the organism that carried the brain had to be viable, the end product is by definition guaranteed to be ``efficient''. Because of that same fact, however, it is all but guaranteed not to be ``clean'': for one, it was not possible to snap whole new components into the system; it would have also been impossible to combine old components in the elaborate and precise ways in which a human engineer might use parts. Worse, old components were almost certainly not discarded when new and better ones came into being. A good exposition of this process in humans can be found in Paul MacLean's seminal work {\em The Triune Brain in Evolution} \cite{maclean1990}.

\begin{figure}[t]
	\centering
	\includegraphics[width=0.4\textwidth]{Figs/noNervousSystem.png}
	\caption{Relationship between the components of an organism without a nervous system.}
	\label{fig:noNervousSystem}
\end{figure}
%\vspace{-0.18cm}
\subsection{Origin of Nervous Systems} The evolution of nervous systems in organisms dates back to the development of primitive electrical signalling in eukaryotes, using calcium action potentials\footnote{See any textbook on evolutionary biology.} and sodium channels \cite{Liebeskind31052011}:

\begin{emquote}
	Voltage-dependent sodium channels are believed to have evolved from calcium channels at the origin of the nervous system.
\end{emquote}

These sodium channels predated modern-day neurons, but served the same fundamental purpose of acting as control systems. We can readily conceive the benefits of imparting a control system onto an organism with the following thought experiment: let us imagine a microscopic organism without any sort of nervous system --- all of its behaviour is hard-coded and mechanical. It can take in nutrients through its cell walls or through an opening; parts of it can contract or expand in response to stimuli like light or pressure; homeostatic conditions can influence its chemistry. Figure~\ref{fig:noNervousSystem} shows this schema: if we enumerate the constituent parts or {\em components} of an organism as $\{C_1,\dots,C_n\}$, the organism's behaviour is caused by signals being sent between $C_i$ and $C_j$ (the case $i=j$ is possible). Such an organism suffers from three disadvantages: (a) reactions are localised, as two of its components might be too far apart to communicate in a timely manner or at all; (b) its repertoire of behaviours is necessarily simple and (c) it is not very adaptable.

Precursors to nervous systems communicated (a) first via action potentials, which were intracellular electrical signals \cite{Liebeskind31052011} (emphasis ours):
\begin{emquote}
Another key animal innovation was the nervous system, which is present in all but a few animals \paren{i.e., sponges and placozoans}. {\em Rapid, specific, long-distance communication among excitable cells} is achieved in bilaterian animals and a few jellyfish \paren{cnidarians} through the use of action potentials \paren{APs} in neurons generated by voltage-dependent sodium \paren{Na$_\mathrm{v}$} channels. Voltage dependent calcium \paren{Ca$_\mathrm{v}$} channels evolved in single-celled eukaryotes and were used for intracellular signalling. {\em It has been hypothesised that Na$_v$ channels were derived from Ca$_v$ channels at the origin of the nervous system} \textsl{{\upshape[}the results in the paper support the hypothesis{\upshape]}}, thereby conferring the ability to conduct action potentials without interfering with intracellular calcium. This view was reinforced by the apparent lack of sodium currents in sponges.
\end{emquote}

The introduction of dedicated, long-distance\footnote{The term ``long-distance'' may very well mean ``long-distance within a single cell''. Balu\v{s}ka and Mancuso argue in {\em Deep evolutionary origins of neurobiology: Turning the essence of ``neural'' upside-down} \cite{frantisek} that neural analogues already existed in prokaryotes (bacteria and archaea; organisms without cell walls and nuclei) and unicellular eukaryotes.} signalling cells between parts of an organism created the possibility of not only transmitting, but also modifying information. The moment an organism's parts do not communicate directly biochemically/mechanically, but over transmissions lines, evolutionary processes acting upon these lines are able to mutate them so that they change the signals. The first changes might consist of amplifying, diminishing, or distributing signals. Over time, the nerves may come to act as transducers on the stream of signals; in some rudimentary sense, they may begin to compute functions. Schematically, we see this in Figure~\ref{fig:nervousSystem}, where a function $F$ is interposed between two components. Not all components of an organism are created equal, of course. The first and most important use of nerve cells was the communication between sensory organs and the movement apparatus of the organism, and the bulk of nerve cells were located close to the sensory organs, where they processed information. A mere handful of neurons are not able to compute much, but they must have conferred considerable advantage to their owners.

\begin{figure}[t]
	\centering
	\includegraphics[width=0.4\textwidth]{Figs/nervousSystem.png}
	\caption{Relationship between the components of an organism possessing a nervous system. $F$ can be understood as a simple signal transformer or a central coordinating mechanism.}
	\label{fig:nervousSystem}
\end{figure}

The history of these developments is not entirely clear, but action potentials are present in all animals (with the exception of sponges) and in plants \cite{Leys01051999, PCE:PCE1614}. 
A step up from mere stream transducers are the nerve nets that permeate the entire bodies of cnedaria (jellyfish) and the nerve cords that run along the bodies of bilateria (animals with left and right sides). In Figures~\ref{fig:animalia2} and \ref{fig:animalia} we see them in the phylogeny of the kingdom animalia. Both can process signals in a sophisticated way, and enable the performing of varieties of complex tasks, although the sets vary widely from species to species.

\begin{figure}
	\centering
	\begin{subfigure}[t]{0.30\textwidth}
		\includegraphics[width=\textwidth]{Figs/neuralNet.png}
		\caption{Nerve nets in ctenaphora.}
		\label{fig:nerveNets}
	\end{subfigure}
	\begin{subfigure}[t]{0.65\textwidth}
		\includegraphics[width=\textwidth]{Figs/animalia3.png}
		\caption{Phylogeny of the animalia. Even Ctenophores, macroscopic marine invertebrates which predate both jellyfish and bilateria, have nervous systems in the form of distributed nerve nets.}
		\label{fig:animalia2}
	\end{subfigure}
	\caption{Nerve nets and phylogeny of animalia. From {\em {T}he ctenophore genome and the evolutionary origins of neural systems} \cite[p.\ 100]{animalia2}.}
\end{figure}

\begin{figure}
	\centering
	\includegraphics[width=0.8\textwidth]{Figs/animalia.jpg}
	\caption{Phylogeny of the animalia. Note the cnidaria and bilateria; both of these have types of nervous systems. From {\em Animals I -- An Overview of Phylogeny and Diversity} \cite{animalia}.}
	\label{fig:animalia}
\end{figure}

\begin{figure}
	\centering
	\includegraphics[width=0.6\textwidth]{Figs/chordata.jpg}
	\caption{Body plans for metazoans. The bottom three items are all bilateria and all have nerve cords of some kinds, but only the bottommost (chordates) have a dorsal (upper) nerve cord. Vertebrates are a subphylum of the chordata. From {\em Evolution of bilaterian central nervous systems: a single origin?} \cite[p.\ 3]{chordata}.}
	\label{fig:chordata}
\end{figure}

\paragraph{Central nerve cord and cephalisation.} Nerve nets, while interesting, are not our aim. Unlike jellyfish, bilateria have a central nerve cord which runs from their front to their back. At various points alongside the cord, we find ganglions --- thickenings containing larger amounts of nerve bundles. In all animals but worms, the frontal ganglion further thickened until it came to contain the overwhelming majority of the organism's neurons --- forming the head. While substantial neural activity was occurring before this time, it is only here that it becomes to proper to speak of brains, and where we can begin to analyse macroscopic structures like lobes.

\begin{figure}
	\centering
	\includegraphics[width=0.5\textwidth]{Figs/brain.png}
	\caption{Illustration of the cerebrum with lobes shown. Hidden: limbic lobe, insular cortex. From {\em Anatomy of the human body} \cite[Fig. 728]{graysAnatomy}.}
	\label{fig:brain}
\end{figure}

Vertebrate brains are subdivided into hinbrain, midbrain, and forebrain, having evolved in this order. The forebrain (cerebrum) is responsible for all higher functions and is again divided into six lobes, which we see in Figure~\ref{fig:brain}. The functions performed by these lobes are not precisely understood, but a number can be clearly associated to one lobe. The frontal lobe, for instance, is responsible of conscious thought; the temporal lobe processes auditory and olfactory signals; the occipital lobe deals with sights.\footnote{Interestingly, the occipital lobe is at the {\em back} of the head.} While some functions, like memory, are not neatly localisable, we can nonetheless see in the anatomy of vertebrate and mammalian brains the accruing of large groups of functions: motor control, smell, hearing, sight, reason, emotions. The question of organisation remains, however: it is one thing to say that we have hearing and smell, but what, if anything, ties these experiences together? We, after all, perceive the data from all our senses as one integrated experience. Here, views diverge. The common-sense belief is that we simply have one, indivisible consciousness. Such a view would implicate the frontal lobe as an central organising unit, without which an organism, even if it could smell or see, would not consciously do so. Minsky, Sloman, and Dennett argue persuasively, but speculatively, against this view in their works \cite{emotionMachine,societyOfMind,sloman1991, dennett1991}. They differ on the details, but all agree that the unified consciousness is an illusion; that it is not a single ``I am''-thing gathering raw data, but a dispersed locus of experiences that we merely perceive as immediate.\footnote{Dennett criticises the idea of a \emph{consciousness-thing} with the concept of the ``Cartesian theater'' \cite{dennett1991}. According to him, positing that there is such a thing in our brains, and that it observes all other brain functions, is fundamentally problematic: if there is such a sort of homunculus in our heads that, say, sees the result the output of visual processing in the manner in which one would see a film, then how its visual perception work? Is there yet another a homunculus inside the homunculus that interprets visual information? Such a view implies either an infinite regress, or the algorithmic inexplicability of some part of the brain.} In this view, the frontal lobe, while still instrumental, would not be the only contributor. All other regions of at least the cerebrum would contribute in some way to the organism's conscious experience. An animal without a frontal lobe would not be conscious in the same way as we are, but it would not be utterly blind either --- it would already have some dim awareness of its existence; some rudimentary ``I am'' that we can hardly imagine would already be present in it. To quote Sloman (emphasis ours):\footnote{The quotation appears in {\em The Emotion Machine} \cite[p.\ 97]{emotionMachine} and Minsky attributes it to a post made by Aaron Sloman in the \texttt{comp.ai.philosophy} newsgroup, but we have been unable to find the original.}
\begin{emquote}
It is not worth asking how to define consciousness, how to explain it, how it evolved, what its function is, etc., {\em because there's no one thing for which all the answers would be the same. Instead, we have many sub-capabilities, for which the answers are different}: e.g., different kinds of perception, learning, knowledge, attention control, self-monitoring, self-control, etc.
\end{emquote}

\paragraph{Implications.} The point in all this is not to give an detailed summary of evolutionary neurobiology; it is to show that nervous systems are ancient, gradually developed things. They have been shaped by the vicissitudes of hundreds of millions of years, and they could have developed in other ways. They were not planned, as a human would understand the word. If we are to gain headway in piecing together the ``big picture'', we must take these facts to heart, and choose our modelling methods accordingly.

In the abstract of this work, we described the biological and the idealistic approaches as being polar opposites, and this is true as far as engineering is concerned, but in terms of their assumptions, false. They are both idealistic. Neural networks, insofar as their users want to re-create human behaviour, implicitly presuppose an intelligence in neurons that is not there. The comparatively small network is taught to compute some desired function, the hope being that it might thereby come to perform some complex, real-world function like common-sense reasoning. In principle, this strategy could work, but in practice, it is unrealistic --- the environment in which real organisms had to succeed was the planet's ecosphere; billions upon billions of nervous systems of all complexities were run over millions of years; nervous systems died off and were re-created from scratch by genes. It is therefore entirely unreasonable to assume that neural networks, trained against an objective function over a period of hours or days could re-create the function a biological organism, unless one were to suppose that there is some inherent quality in neurons that strives for such; that groups of cells somehow {\em wish} to organise themselves into specific configurations in which they are able to perform activities we would call ``cognition''. 

All this being said, we should not confuse criticism of the suitability of a method for a specific purpose with criticism of its suitability for any purpose. Neural networks have proven useful in understanding mental activity at small scales; both they and the symbolic/logic-based approaches have had a myriad of industrial applications. From this, however, it does not follow that we can build genuinely intelligent agents with them. Our only means of doing that (the only means that remain) is to laboriously unravel the developmental history of animal brains, step by step, making sense of each development in context. Where empirical data are not available, we at least have to hypothesise how things could plausibly have happened. To this day, structural and genetic analyses have been done (via genetic sequencing and MRI), but they do not deliver sufficiently detailed data. Such methods are rather akin to measuring voltages and task time in a PC --- they do tell us something, but an observer would never infer the existence of, e.g., compilers, call stacks, or type systems from such observations. For an understanding of the brain so specific that we can re-implement it in a computer, we will need currently non-existent and not-conceived-of technology. Until that day, guesswork, like the assumptions made in this thesis, will have to suffice.

\subsection{Ways of Adaptation}

After the philosophical groundwork and biological basics, let us describe possible means by which nervous systems can change and acquire new features. We begin with the observation that the existence of neuron bundles between parts of an organism is analogous to a loose coupling of components in a software systems. By having intermediaries that take over the task of communication, selection pressure can produce more and more complex functions, since it no longer has to act upon the body parts that send various signals, but change the nervous system that processes these signals instead. As example: pain receptors, muscles fibres, and the optical nerve have been unchanged for quite some time, long pre-dating the human species, but more recent brain developments have given us the ability to utilise them in novel ways --- by providing a rich mental experience of suffering, playing instruments, and mentally rotating objects, respectively. Manipulating the software is far easier and more quickly done than doing so with the hardware, so to speak.

Having said that, the changes still have to have occurred incrementally. Even if a nervous system can change quickly (for evolutionary timescales), it still has change in tiny steps. We shall leave the matter that for the time being, but, as we will discuss later, this simple fact has profound computational consequences that are seldom thematised in discourse on this matter. 

Let us return to the consideration of primitive life forms. We can imagine the malleable neuron bundles of such ancient organisms changing in a variety of ways in the face of selection pressure: when the environment required it, they could, after several generations, start to compute different or more elaborate functions. An organism which had had developed in an environment where food was abundant in bright places and which had now found itself in darkness would have benefited from a variety of plausible changes, such as
\begin{itemize}
	\item an inversion of its light-seeking behaviour,
	\item switching off its metabolism in light places to conserve energy,
	\item accelerating its metabolism in dark places to make better use of the food there.
\end{itemize}

Of course, other changes would have also been possible, such as the metabolisation of different food sources,\footnote{A current-day example is given by nylon-eating bacteria, which have developed in the last century and which now have an abundant food source and no competition.} but we can see how the aforementioned three could have been effected through mutations in a simple nervous system. For a system to permit such mutations, it must be far more robust than most products of human engineering, however. If one were to take out a piston in a car or replace a cogwheel in a mechanical clock with a differently sized one, the machine would, in most cases, simply break. In all others, it would catastrophically malfunction. Machines are designed to fit together perfectly and their complexity tends to be irreducible. Even software, which is more readily changed, is easily broken by small-scale tinkering.

When discussing how they can evolve and, in particular, {\em evolve to perform new tasks} and not just variations on old ones, explanations are again constrained by two criteria: (a)~the change has to be small, or at least have a small cause\footnote{The effect does not have to be small --- changes in single genes can switch entire components on or off. The MYH16 gene, which is present in non-human primates but has been switched off in humans, is an example. In us, its disabling lead to a drastic reduction in the size of jaw muscles and a corresponding increase in brain size~\cite{carroll2005}. Nonetheless, such events are rare and not the main drivers of evolution.} and (b)~each change must be beneficial in the short term.\footnote{Caveats apply: if the selection pressure on a group of organisms is not too strong, changes which may be suboptimal but perhaps beneficial at some later point may spread, and non-selective processes like genetic drift can also play a role.} Something that we would conventionally recognise as a program, something which has precise notions like ``instruction'' and ``call structure'' is probably not suited to this pattern of changes.\footnote{Cf.\ evolutionary program generation, in which expression trees mutated. We charge that such algorithms are not adequate models of what happened in the evolution of our brains.} Instead, we ought to imagine the brain as a mesh of computation in which functions are computed cumulatively, so that small changes in neural structure only lead to small changes in output.

To illustrate this, we can look at a simple neural network in Figure~\ref{fig:neuralNetwork}, with a marked node $N_x$. Figure~\ref{fig:unlikelyEvolution} shows an unlikely change scenario in which some new component/function is cleanly grafted onto the system. Figures~\ref{fig:likelyEvolution} and \ref{fig:copyEvolution} then show two more likely scenarios: in the first a mutation causes $N_x$ to be split and the new nodes take over some of its connections. In the second, a larger component is accidentally copied as-is and, over time, is moulded to do something useful.\footnote{Such copies can be caused by mutations and are known to happen with some frequency in nature.} In time, new functions can thus grow into the system, but never in the manner in which, say, an engineer would implement a new feature. 

\begin{figure}
	\centering
	\begin{subfigure}[t]{0.41\textwidth}
		\includegraphics[width=\textwidth]{Figs/neuralNetwork.png}
		\caption{A simple neural network.}
		\label{fig:neuralNetwork}
	\end{subfigure}
	\begin{subfigure}[t]{0.45\textwidth}
		\includegraphics[width=\textwidth]{Figs/unlikelyEvolution.png}
		\caption{An unlikely change scenario in which new, discernible components are grafted on from whole cloth.}
		\label{fig:unlikelyEvolution}
	\end{subfigure}
	\begin{subfigure}[t]{0.45\textwidth}
		\includegraphics[width=\textwidth]{Figs/likelyEvolution.png}
		\caption{A more likely change scenario in which one part is split into three but where the overall shape of the network is not appreciably altered.}
		\label{fig:likelyEvolution}
	\end{subfigure}
	\hspace{2mm}
	\begin{subfigure}[t]{0.45\textwidth}
		\includegraphics[width=\textwidth]{Figs/copyEvolution.png}
		\caption{A second change scenario in which an entire component is accidentally copied. While the resultant change is large, a small genetic mutation can cause it.}
		\label{fig:copyEvolution}
	\end{subfigure}
	\caption{Means of change in neural networks.}
\end{figure}

\paragraph{Sloman's brain.} One might ask what the relationship between the gradual growth of neural bundles and the observed, large-scale functions in the brain is. We have now supposed at some length that the organisation is not neat, but the question remains whether we can speak of an organisation at all (even a messy one). In {\em Beyond shallow models of emotion} \cite[p.\ 8]{sloman2000}, Sloman illustrates the possible chaotic organisation of the brain with Figure~\ref{fig:slomanBrain}, conjecturing that it might be a jumble of parts that just happen to work together:

\begin{emquote}
	Any observed behaviour might be produced by an unintelligibly tangled and non-modular architecture. \paren{Rectangles represent information stores and buffers, ovals represent processing units, and arrows represent flow of information, including control signals.}
\end{emquote}

\begin{figure}[t]
	\centering
	\includegraphics[width=0.4\textwidth]{Figs/slomanBrain.png}
	\caption{Sloman's illustration of the brain as an ``unstructured mess''.}
	\label{fig:slomanBrain}
\end{figure}
It sounds somewhat like cheery optimism to presuppose that there even are information stores and control signals. The actual situation is likely a far worse one: it is not just different programs that are run in the brain, but entire different models of computation, with the same pattern of activity being interpreted simultaneously in more than one way.
Today, we can scarcely imagine how such an ``architecture'' would work, let alone how one would program it --- but if we really want to create genuinely animal intelligences, we will have to find out. Sloman himself admits to this difficulty in {\em What Sort of Control System Is Able to Have a Personality?} \cite[p.\ 6, Section 9 ``Is the task too hard?'']{sloman1997}:
\begin{emquote}
Given the enormous diversity in both design space and niche space and our limited understanding of both, one reaction is extreme pessimism regarding our ability to gain significant insights.
\end{emquote}
The following remedy is offered:
\begin{emquote}
My own attitude is cautious optimism: let us approach the study from many different directions and with many different methodologies and see what we can learn. \ellipses\\
In particular, the Cognition and Affect group at Birmingham has been trying to use a combination of philosophical analysis, critical reflection on shared common sense knowledge about human capabilities, analysis of strength and especially weaknesses in current AI systems, and where appropriate hints from biology, psychology, psychiatry and brain science, to guide a combination of speculation and exploratory implementation. \ellipses
\end{emquote}

The methods listed all have their applications, but computational analysis is missing among them. When we talk of psychology, philosophy, and critical reflection, we have already supposed too much; we want to replicate the high-level output of the brain without having explored the mechanism that produces it. In a manner of speaking, we have seen the forest, but do not understand what trees are. If we are to gain the sort of knowledge of brains that we can implement into an AI, we must, empirically, find out about their method of computation. Barring that, we must at least approximate it as far as is practical, and accept that the result will necessarily be an inferior simulacrum.

What, then, is the computational model used in the brain? So far, nobody knows, and, presumably, that will stay that way for the foreseeable future. It is very much a guess, but from the concept of slowly growing neural networks (seen in Figures~\ref{fig:neuralNetwork}-\ref{fig:copyEvolution}), one might infer something like the ``active symbol'' hypothesis in Douglas Hofstadter's {\em G\"{o}del Escher Bach}: that patterns of activity form little programs and pieces of data at once; that manipulate other patterns of activation and are manipulated yet other patterns during their lifetime. These are only imperfect analogies, of course. On the coming pages, we shall outline a conceptually compatible white-box model of computation as another, imperfect analogy that will serve as the basis for the model in Chapter~\ref{ch:componentModel}, and for the implementation of the toy agents in Chapter~\ref{ch:implementation}.

\section{The Brain as a Collection of White Boxes}\label{sec:whiteBoxModel1}

We now leave the realm of established fact and venture into conjecture. What has been said up to this point has been good, general fact, but it does not suffice for building actual programs. Data on the computational structure of the brain is scarce, thus we will limit myself to positing general, plausible hypotheses about what sorts of structures and loci of computation could have plausibly arisen in it over the course of its evolution. 

A plausible case has been made by Minsky, Sloman, and others (especially in {\em The Emotion Machine} \cite{emotionMachine}) that the brain must possess components in some form. Were it not so, the organ would have long ago succumbed to the inefficiencies of its design. As more and more functions are grafted onto a system, the number of interactions between its parts or regions, and therefore the bugs in it, increases. Worse yet, the system becomes brittle: even if, like in a neural network, some accidentally working configuration would have been able to be reached, small changes would surely have upset it again. The part-less system is an evolutionary dead-end from which no improvement is possible, and given how far along our cognition is, it is quite clear that we are not dealing such when we look at our brains.

\begin{figure}[t!]
	\centering
	\includegraphics[width=0.5\textwidth]{Figs/slomanSystem.png}
	\caption{Architecture for motivated agents. From \cite[p.\ 10]{sloman1997}}
	\label{fig:slomanSystem}
\end{figure}

If we concede that we are dealing with identifiable parts, a second question arises: how do these parts communicate? We would like to deal with this question in some detail. In the literature, this issue is often glossed over --- in diagrams, one frequently finds unannotated arrows going between functions; the accompanying texts mention concepts like ``selection'', ''message'', and ``sending'' under the implicit assumption that these are merely primitives in no need of further explanation. When we consider the workings of neurons, however, it is not at all clear how groups of them could put together any sort of complex message, and, once put together, how it would travel, and how another group of neurons could receive and interpret it. Are there dedicated interpreters, akin to compilers and runtimes in computer systems? This is not known. We cautiously propose that it is not so, but we can present plausibly-sounding scenarios for both outcomes:
\begin{itemize}
	\item On the one hand, we may image that, early on, some simple message format developed through, allowing more efficient communication between not quite differentiated regions of a nervous system. Over time, this was extended as more components came into play; these new components would have found it easier to make use of the pre-existing protocol. Larger clusters of parts might have even repeated the process and developed simple, internal message formats for communication among themselves. As an orthogonal development, newly developed components might have performed more abstract duties, using older ones as subsidiaries, if at all. To solve conflicts whenever these new and old parts proposed different solutions to whatever issue the organism faced, some other component could have received inputs from both, and adjudicated. In such way, a hierarchical and layered structure could have come into being --- different layers working at different levels of abstraction, and each component only communicating on an on-basis with others. All in all, the whole system would come to resemble a human-developed program.
	
	\item On the other hand, we could imagine quite a different scenario: suppose that the basic scheme of neurons sitting as growths on the communication lines between components never changed. Their basic task was the modulation of signals, and if some new function was to be grafted into the system, then this would have been achieved by growing more neurons that modulated the signals of their fellow neurons. They would not have opened a communication channel with the existing components, but would have listened in on their activity in the manner of interlopers surreptitiously modifying messages. Since neurons would have had no reason to hide their activity (as a black box does), this would have been quite easy and straightforward to do. New bundles of functionality could have inserted themselves into the middle of the information flow (enhancing existing functionality, or adding administrative features), before it (providing pre-processing), or after it (providing post-processing, or usage of the output for higher-level tasks). In contrast to above, we concentrate on the {\em process} of software development instead of its product: a human programmer adds functions one at a time, here and there, extending and refining functionality where fancy strikes or necessity requires.
\end{itemize}

One could thus call the first scenario the \emph{product-oriented view} and the second the \emph{ process-oriented view}: the first looks at the end product of a development, the second posits that the very process of that development, fossilised, is present in the end product.

Evidence is scarce and equivocal for both. In fact, it need not even be the case that they form a dichotomy: we might just as well speculate, for instance, that the second is the low-level reality, but that the first emerged from over time due to the efficiency of its design. The components could be fuzzy, to some degree. We could also posit that the first one ``degenerated'' into the second one; that a formal system is emulating an informal one because of the latter's greater versatility and dynamism.

For the rest of the thesis, we will explore the second of the above two hypotheses, not necessarily because we firmly believe it to be true, but rather because the first one has been tried for some time, and has so far not produced a general AI.

\paragraph{Practical abstraction.} While such a white-box model, and the hypothesising that preceded it, are conceptually useful, a mesh of gradually grown patterns does not lend itself to implementation in a program. We do not have the capability of faithfully pouring the structure of the human brain into a computerised mould just yet, but, for the time being, we may opt for the next best thing and take cues from it in the hopes of improving our imitations.

Therefore, we will present a simplified model which, while attempting to remain true to the conceptual view, will, pragmatically, contain discrete functions and components. The white-box nature of brain activity will be emulated by a message-passing scheme in which messages model the internal activity of components. Instead of each component blindly acting in some fashion on the activity of another, components will have explicit parsers and interpreters and later, these will be further simplified into localised message formats and tagging, for the sake of easy implementation.
This effort is guided by the same thought as Sloman's cognitive architecture depicted in Figure~\ref{fig:slomanSystem}. It is not a truly accurate representation of the brain and it does not claim to be, but it is {\em something like it}; something that is close, and good, enough. We will meet this cognitive architecture again later, but for now, we move on to the description of neural components.

%While some form interpretation surely must happen in one or more places, it is implausible to suppose that there is one agreed-upon message format, or one programming language that all parts use. 
%
%
%The facts and processes of the previous subsection are relatively uncontroversial and can be found reiterated in any textbook on the evolution of nervous systems. The functional structure and the model of computation used in the brain, however, are not well understood. FMRI and similar brain imagining techniques, while invaluable, give only rough impressions about the neural correlates of certain forms of cognition and do not give fine-grained insight into its structure. As such, the model we shall describe in the following paragraphs is a conjecture. The implication of such an evolutionary viewpoint, we conjecture in this document, is that brain functions don't ``just appear'', but are rather the result of small changes and the recombination of pre-existing parts. This, in turn, informs the plausibility of various possible brain architectures. It becomes unlikely that the brain should be a collection of neatly delineated functions, or that it should have certain coordinating units or universal message formats for communication between components. The reason for this is that administrative mechanism confer little evolutionary benefit on their own, and do not confer it gradually: the imposition of a central coordinating mechanism on a pre-existing mesh of neurons would necessitate the complete reorganisation of such, and the abandonment of the previous communication channels in favour centralised coordination. The same objections can be raised against a universal or even a local message format. Moreover, such mechanisms require substantial changes in the organism with no obvious or immediate advantage.
%
%These objections do not contradict the existence of macroscopic structures in the brain, dedicated to certain tasks. The development and adaptation of such remains entirely plausible. They do, however, give insight into the pattern of processing inside such structures, which is often simply regarded as atomic or replicated in computers as if it were a conventional engineering product.
%
%Instead of a rigidly ordered brain with central organisation and large, discrete, and highly complex features like ``sight'' or ``reason'' which function like ready-made black boxes, to be plugged in at will, we propose a decentralised white-box architecture composed of simple parts: first, every component, while perhaps sophisticated, is conceptually simple. Second, communication between different components is not performed in the function-call pattern of computer programs, but rather by one component listening in on the activity of another. Since there is, inherently, no mechanism of function abstraction in neural systems, it stands to reason that the most likely way for new functions to develop is for additional neurons to modulate the activity of others. In such a scheme, a visual perception component does not have to know which other components will consume its output (or rather, listen on its activity); changes which affect agent activity in useful ways based on the visual data can occur gradually and, over time, become large enough to count as components in their own right.

	
	\chapter{Overall Component Model}\label{ch:componentModel}
	
	Having gone through a number of biological considerations, we now translate these into a mathematical model of a component system. These components can send and receive messages, filter them out, and interpret them in various ways. This model does not yet specify our architecture, but our artificial intelligence, as described in Chapter~\ref{ch:implementation}, will be comprised of such components.

\section{Components as White Boxes}\label{sec:whiteBoxModel2}

We can imagine the components of the mind as white boxes which inform other components by their very functioning --- however, this does not lend itself to easy implementation. Instead, we can emulate this behaviour via a \emph{message space}, from which individual components take their input and into which they put their output. A \emph{component} is then a local processing unit which continuously scans the message space, running messages through its \emph{filter}. If the filter detects a relevant message, it is then passed to the \emph{interpreter}, which parses the message into the needed format and hands it over to the \emph{processor}. The processor, after having finished, puts its output back into the message space for other components to read. Figure~\ref{fig:global} illustrates this scheme. Note the lack of explicit hierarchical structure and central organising units.

\begin{figure}
	\begin{center}
		\begin{tabular}{l l}
			\toprule
			Symbol & Description\\
			\midrule
			
			\begin{minipage}[t]{0.2\textwidth}
				\includegraphics[width=60pt]{Figs/legend_proc.png}
			\end{minipage}
			& Processing component\\
			\includegraphics[width=60pt]{Figs/legend_choice.png} & Choice\\
			\includegraphics[width=40pt]{Figs/legend_container.png} & Data container (Queue, List, etc.)\\
			\includegraphics[width=40pt]{Figs/legend_data.png} & Data\\
			\includegraphics[width=20pt]{Figs/legend_generator.png} & Stream generator\\
			\includegraphics[width=40pt]{Figs/legend_imaginary.png} & Counterfactual (imaginary) data\\
			\bottomrule
		\end{tabular}
	\end{center}
	\caption{Notation for the diagrams in this and the following sections.}
	\label{fig:diagramNotation}
\end{figure}
%

However, as we will show in the next section, this model is generic enough to accommodate special-purpose structures like a message space. Figure~\ref{fig:global} shows the message-passing scheme, but it also specifies a graph in which the nodes are the components and fixed, while the edges are the accepted messages and are determined by the nodes; through their filters, components control the shape of the graph. By imposing invariants on these filters, we can have the graph take any shape we desire. In particular, we can model the kinds of structures that occur in many other cognitive models and in empirical research: central organisers, sequences of components (``pipelines''), localized messages affecting only a small part of the mind, a component reading its own messages, loops and iterative messages between two or more components et cetera.

\paragraph{Messages.}

We may now ask how such messages between components are structured. Here, I make two empirical claims:
\begin{enumerate}
	\item messages have a priority and
	\item they are effectively unstructured.
\end{enumerate}

To the best of our knowledge, the veracity of either has thus far not been determined by neuroscience. For the first, Marvin Minsky's ``The Emotion Machine'' provides some circumstantial evidence \cite[p.\ 222]{emotionMachine}:

\begin{emquote}
	Of course, when one activates two or more Critics or Selectors, this is likely to cause some conflicts, because two different resources might try to turn on a third resource both {\em on} and {\em off}. To deal with this, we could design the system to use various policies like these:
	
	\begin{enumerate}
		\item Choose the resource with the highest priority.
		\item Choose the one that is most strongly aroused.
		\item Choose the one that gives the most specific advice.
		\item Have them all compete in some ``marketplace''.
	\end{enumerate}
\end{emquote}

\begin{figure}[t]
	\centering
	\includegraphics[width=0.5\textwidth]{Figs/global.png}
	\caption{Global neural architecture.}
	\label{fig:global}
\end{figure}

\begin{figure}[b]
	\centering
	\includegraphics[width=120pt]{Figs/message.png}
	\caption{Structure of a neural message.}
	\label{fig:message}
\end{figure}

The selection strategies Minsky lists imply that there is some mechanism in the brain to determine the urgency of a signal. While it is possible that higher brain functions like reasoning or affect make an additional, rational evaluation, sensations like intense pain, bright lights, or great sadness can likely be communicated most easily by the appropriate components causing a flood of activity which, by its very intensity, informs other components of the urgency of their messages.

The second claim --- that messages are essentially unstructured --- means that there is no common, agreed-upon format in which they are stored. In addition to the evolutionary implausibility of such a format being created, an unstructured message format is in line with the white-box nature of components: since components merely ``listen in'' on others, and since each components will have its own pattern of activity, a listener would simply have to try and make sense of this activity as best it could. The proposed structure of messages is thus shown in Figure~\ref{fig:message}: every message comprises a priority header, together with an unstructured body which, for our purposes, is simply a string of bits.

\paragraph{Filters.} Before a component can respond to a message by another, such a message must be assessed for the presence of relevant information. Conceptually, this happens via a \emph{filter} in each component, which pattern-matches incoming messages and, if a certain threshold is reached, signals relevance and hands the message over the \emph{interpreter} for parsing. Figure~\ref{fig:filter} shows such a filter: it is composed of a directed graph of nodes, and a node is activated if it detects some specific content in the message. Nodes, in turn, are connected via edges of strength within the interval $[0,1]$. When a node is activated, it sends a charge proportional to the strength of its link to its neighbours, contributing to their activation as well. Some nodes are marked as {\em output nodes}; if enough such output nodes become activated, the message is deemed to be sufficiently relevant. This model of filters is inspired by the {\em spiking neural P Systems} of Georghe Pa\u{u}n et al. (\cite[p.\ 337]{membraneComputing} and \cite{spikingNeural}), in which charges sent along directed graphs of neurons are used to compute functions.

\begin{figure}[t]
	\centering
	\includegraphics[width=168pt]{Figs/filter.png}
	\caption{A pattern-matching filter for a component $C_i$.}
	\label{fig:filter}
\end{figure}

\section{Mathematical Model}\label{sec:mathematicalModel}

We now create a mathematical model for the description of the architecture. This model will be split into two parts: the structural and the operational semantics. The structural semantics encode the static properties of neural systems, whereas the operational semantics describe the behaviour of such a system at runtime.

\subsection{Preliminaries}\label{sec:mathematicalPreliminaries}

Since the mathematical model is built with implementation in mind, we will use some basic type theory in the coming sections. The following notions are from the $\lambda$-calculus and its attendant type systems; anyone familiar with such can therefore freely skip this section. We will introduce types, type constructors, and their relation to functions, together with a few example types which will come in handy later on. The following can be found in any introduction to type theory and was taken (with simplification) from \cite{Mendler:1988:IDT:913822}, \cite{typeIntroduction2}, and \cite{Jacobs97atutorial}.

\begin{definition}[Syntax: Type]\label{def:type}
	For our purposes, types are defined inductively thus:
	\begin{enumerate}
		\item {\upshape Basic type:} $\R$ and $\emptyset$ are types.
		\item {\upshape Sum type:} If $\type{T_1},\type{T_2}$ are types, the {\em sum type} $\type{T_1 + T_2}$ is a type. 
		\item {\upshape Product type:} If $\type{s}$ is a string and $\type{T_1},\dots,\type{T_n}$ are types, the {\em product type} $\type{s}\ \type{T_1} \dots \type{T_n}$ is a type. A special case is the {\em anonymous product type} \paren{tuple}, where $s=``\langle\rangle"$. There, we just write $\langle\type{T_1},\dots,T_n \rangle$.
		\item {\upshape Full application:} If $\type{T_1},\dots,\type{T_n}$ are types and $\allQ{x_1,\dots,x_n} \type{C}$ is a type constructor \paren{see next definition}, then $\type{C\ T_1} \dots \type{T_n}$ is a type.
		\item {\upshape $\mu$-abstraction:} If $\type{T}, \type{S}$ are types and $\type{S}$ occurs in $\type{T}$, then $\muQ{\alpha}\type{T}[\type{S}\backslash\alpha]$ \paren{for a fresh variable name $\alpha$} is a type.
	\end{enumerate}
\end{definition}

\begin{definition}[Syntax: Type constructor]\label{def:typeCon}\
	Type constructors are the defined thus:
	\begin{enumerate}
		\item {\upshape Base case:} Every type $\type{T}$ is a type constructor.
		\item {\upshape Abstraction:} If $\type{C}$ is a type constructor and $\type{T}$ is a type, $\allQ{x} \type{C}[\type{T}\backslash x]$ is a type constructor.
		\item {\upshape Sum types:} If $\type{C_1}\dots,\type{C_n}$ are type constructors with $\type{C_i} = \allQ{\vec{X}_i} \type{T_i}$ \paren{$1 \leq i \leq n$}, then $\allQ{\vec{X}_1; \dots; \vec{X}_n} (\type{C_1} + \dots + \type{C_n})$ is a type constructor.
		\item {\upshape Partial application:} If $\type{T_1},\dots,\type{T_i}$ \paren{$i < n$} are types and $\allQ{x_1,\dots,x_n} \type{T}$ is a type constructor, then $\allQ{x_{i+1},\dots,x_n} \type{T}[x_1\backslash \type{T_1},\dots,x_i\backslash \type{T_i}]$ is a type constructor.
	\end{enumerate}
\end{definition}

Every type is interpreted as a set of values which are of that type; type constructors are interpreted as universally quantified templates for actual types. Their formal semantics are as follows:

\begin{definition}[Semantics: Type]\label{def:typeSem}
	Let $\type{T}$ be a type. Its interpretation function $\tint(\type{T})$ is defined thus:.
	\begin{enumerate}
		\item {\upshape Basic type:} $\R$ is interpreted as the set of real numbers. $\tint(\emptyset) = \{\}$.
		\item {\upshape Sum type:} If $\type{T_1}, \type{T_2}$ are types, then $\tint(\type{T_1 + T_2}) = \tint(\type{T_1}) \cup \tint(\type{T_2})$.
		\item {\upshape Product type:} If $\type{T_1},\dots,\type{T_n}$ are types and $\type{s}$ is a string, then
		$$
		\tint(\type{s}\ \type{T_1} \dots \type{T_n}) = \left\{
		\begin{array}{l l}
		\{\type{s}\} & \mt{if } n = 0\\
		\{\type{s}\} \times \tint(\type{T_1}) \times \dots \times \tint(\type{T_n}) & \mt{if } n \geq 1.
		\end{array}
		\right.
		$$
		
		\item {\upshape Full application:} If $\type{T_1},\dots,\type{T_n}$ are types and $\allQ{x_1,\dots,x_n} \type{C}$ is a type constructor, then
		$$
		\tint(\type{C}\ \type{T_1} \dots \type{T_n}) = \bigcup\limits_{v_1 \in\ \tint(\type{T_1})} \cdots \bigcup\limits_{v_n \in\ \tint(\type{T_n})} \left( \bigcup\limits_{C' \in\ \cint(C)} C'[x_1\backslash v_1,\dots, x_n\backslash v_n] \right).
		$$
		
		\item {\upshape $\mu$-abstraction:} If $\muQ{\alpha}\type{T}$ is a type, then $$\tint(\muQ{\alpha}\type{T}) = \tint(\type{T}) \cup \tint(\type{T}[\alpha\backslash\type{T}]) \cup \tint(\type{T}[\alpha\backslash\type{T}][\alpha\backslash\type{T}]) \cup \dots$$
		
		with $\tint(\alpha) = \{\}$.
	\end{enumerate}
\end{definition}

\begin{definition}[Semantics: Type constructor]\label{def:typeConSem}
	The partial interpretation function $\cint$ for type constructors is defined as follows: if $\type{C}$ is a type constructor containing exactly the types $\type{T_1},\dots,\type{T_n}$, then
	$$
	\cint(\type{C}) = \bigcup\limits_{v_1 \in\ \tint(\type{T_1})} \cdots \bigcup\limits_{v_n \in\ \tint(\type{T_n})} \type{C}[\type{T_1}\backslash v_1,\dots,\type{T_n}\backslash v_n].
	$$
\end{definition}

Intuitively, sum types are simply unions, product types are named Cartesian products, and full applications are instantiations of type constructors with all possible values. $\mu$-abstraction represents recursive data types such as lists or trees. Type constructors themselves are just generic types.

Whenever we want to assert that an expression has a specific type, we write:

\begin{notation}[Typed expressions]
	Let $x$ be an expression and $\type{T}$ a type. $x :: \type{T}$ asserts that $x$ has type $\type{T}$.
\end{notation}

Henceforth, by convention, we will write type variables in lower-case and concrete types in upper-case, omitting the explicit $\forall$-blocks. That is, a type like $\allQ{x,y,z} \type{C\ x\ (\N + T_1)\ y\ z}$ will simply be written as $\type{C\ x\ (\N + T_1)\ y\ z}$ and it will be clear that $\type{x}, \type{y}, \type{z}$ are type variables, while $\N, \type{T_1}$ are concrete types. A special kind of type constructor is the function arrow ($\rightarrow$) which induces the function type:

\begin{example}[Function arrow]
	If we take, say, the type $\rightarrow \type{S1}\ \type{S2}$ \paren{a product type with the product types $\type{S1}$ and $\type{S2}$ as arguments} and abstract twice, we get $\allQ{s,t} \rightarrow \type{s}\ \type{t}$. Here, $\rightarrow \type{s}\ \type{t}$ is the type constructor for unary functions from $\type{s}$ to $\type{t}$, also written infix as $\type{s} \rightarrow \type{t}$. Functions with multiple arguments, mapping $\type{t_1},\dots,\type{t_{n-1}}$ to $\type{t_n}$, can be modelled in two ways: either through n-tuples, or through nested function arrows:
	$$
	\begin{array}{l}
	\tuple{\type{t_1}, \type{t_2}, \dots \type{t_{n-1}}} \rightarrow \type{t_n};\\
	\type{t_1} \rightarrow (\type{t_2} \rightarrow \dots \rightarrow (\type{t_{n-1}} \rightarrow \type{t_n})\cdots).
	\end{array}
	$$
	
	The first method necessitates that we supply all arguments at once, whereas the second allows them to be given one after another.
\end{example}

\noindent
Function arrows allow the execution of functions in the obvious way:

\begin{definition}[Function application]
	Let $f :: \type{S} \rightarrow \type{T}$ and $x$ be an expression of type $\type{S}$. Then $f\ x$ is an expression of type $\type{T}$. Function application associates to the left, that is: $f\ x_1 \dots x_n = (\cdots((f\ x_1)\ x_2) \dots x_n)$.
\end{definition}

\noindent
We can combine type constructors, sum types, and product types into {\em algebraic data types} (ADTs).

\begin{definition}[Algebraic data type (ADT)]\label{def:ADT}
	Let $\type{s}$ be a string and $\type{C_1},\dots,\type{C_n}$ be type constructors such that $\type{C_i} = \allQ{x_1,\dots,x_n} \type{T_i}$ and $\type{T_i}$ is a named product type with type variables \paren{$1 \leq i \leq n$}. Then $\allQ{x_1,\dots,x_n} (\type{T_i} + \dots + \type{T_n})$ is an ADT. If we want to give a name to an ADT, we write it as $\type{s\ x_1 \dots x_n = T_i + \dots + T_n}$.
\end{definition}

Since an ADT is merely the sum of product types, it is itself a type constructor. If it has no type variables, it is also a type. Next, we define a couple of example ADTs, some of which we will use in the next section.

\begin{example}[$\N$, $\B$, $\Q$, $\C$,  Maybe, Either, List]
	$$
	\begin{array}{r c l}
	\N & = & \muQ{\alpha}\type{Z + S\ \alpha};\\ 
	\B & = & \type{False + True};\\
	\Q & = & \type{Rat}\ \N\ \N;\\
	\C & = & \type{Complex}\ \R\ \R;\\
	\type{Maybe\ t} & = & \type{Nothing + Just\ t};\\
	\type{Either\ l\ r} & = & \type{Left\ l + Right\ r};\\
	\type{List\ a} & = & \muQ{\alpha}\ \type{Nil + (a : \alpha)}.\\
	\end{array}
	$$
	
	$\N$ is the usual Peano definition of natural numbers, with a nullary product type $\type{Z}$ representing zero, and a unary product type $\type{S}$, which allows recursion. $\B$, $\Q$, $\C$ are the sets of Boolean number and rational/complex numbers, respectively, with $\type{False}$ and $\type{True}$ being nullary product types, and with $\type{Rat}\ \N\ \N$ and $\type{Complex}\ \R\ \R$ being binary ones. $\mathtt{Maybe}$ represents an optional value, which may or may not be present. $\mathtt{Either}$ represents a choice between two values, of which either the left or the right one is present, but not both. $\type{List\ a}$ \paren{or just $\type{[a]}$ as a shorthand} denotes a list of values of type $\type{a}$. There, $\type{Nil}$ is the nullary type constructor for an empty list and $\type{:}$ is an infix binary type constructor that stores the head and tail of a list.
\end{example}

\noindent
We also define the usual convenience functions for these types:
$$
\begin{array}{l l}
\begin{array}{r c l}
\field{isJust} & :: & \type{Maybe\ a \rightarrow Bool};\\
\field{isJust}\ m & = & \left\{
\begin{array}{l l}
\type{True} & \mt{if } m = \type{Just}\ x,\\
\type{False} & \mt{otherwise};\\
\end{array}
\right.\\
\\
\field{fromJust} & :: & \type{Maybe\ a \rightarrow a};\\
\field{fromJust}\ m & = & \left\{
\begin{array}{l l}
x & \mt{if } m = \type{Just}\ x,\\
\bot & \mt{otherwise};\\
\end{array}
\right.\\
\end{array}
&
\begin{array}{r c l}
\field{head} & :: & \type{[a] \rightarrow a};\\
\field{head}\ l & = & \left\{
\begin{array}{l l}
x & \mt{if } l = x:xs,\\
\bot & \mt{otherwise};\\
\end{array}
\right.\\
\\
\field{tail} & :: & \type{[a] \rightarrow [a]};\\
\field{tail}\ l & = & \left\{
\begin{array}{l l}
xs & \mt{if } l = x:xs,\\
\bot & \mt{otherwise}.\\
\end{array}
\right.\\
\end{array}
\end{array}
$$

Definitions~\ref{def:type}--\ref{def:ADT} specify a fragment of System F$_\omega$,\footnote{Specifically, the decidable fragment of System F$_\omega$ without higher kinds and only prenex-polymorphism. That is, type constructors can only take types as arguments and are of the form $\allQ{x_1,\dots,x_n} \type{C}$ for quantifier-free $\type{C}$. This is also called the Hindley-Milner type system. For details, see \cite{barendregt91}.} which is used to type expressions in the lambda calculus. Although System F$_\omega$ is strictly more powerful, our definitions are enough to provide a description language for the data types and functions in the rest of this work.


\subsection{Neural Systems}\label{sec:mathematicalNeuralSystem}

\begin{definition}[Neural component]
	Let $I$ be an index set and let $\type{T}$ be any type. Then, a neural component $C$ with a name from $I$ and message type $\type{T}$ is a four-tuple
	$$
	\tuple{\field{name}, \field{ft}, \field{int}, \field{proc}}
	$$
	where
	\begin{enumerate}
		\item $\field{name} :: I$ is the {\em name} of $C$,
		\item $\field{ft} :: \type{T \rightarrow \B}$ is  called the {\em filter} of $C$,
		\item $\field{int} :: \type{T} \rightarrow \type{Maybe\ T}$ is called the {\em interpreter} of $C$, and
		\item $\field{proc} :: \type{T} \rightarrow \type{T}$ is called the {\em processor} of $C$.
	\end{enumerate}
	
	Formally, the type of $C$ is $\field{Comp}_{\type{T},I}$. As a shorthand, we denote the name, filter, interpreter and processor of a given component $C$ as $\field{name}_C$, $\field{ft}_C$, $\field{int}_C$, $\field{proc}_C$, respectively.
\end{definition}

\noindent
A set of neural components, together with a set of messages, induces a {\em neural system}:

\begin{definition}[Neural system]
	Let $T$ be any type and let $I$ be an index set. Then, a \emph{neural system} with message type $T$ and component names from $I$ is a tuple
	$$
	\langle \textbf{Co}, \textbf{Me} \rangle
	$$
	where
	\begin{itemize}
		\item $\textbf{Co}$ is a set of neural components \paren{with message type $T$ and names from $I$} and
		\item $\textbf{Me}$ is a set of elements of type $T$, called the {\em set of messages}.
	\end{itemize}
\end{definition}

\subsection{Sending and Receiving Messages}\label{sec:notation}

We now give a notation for the sending and receiving of messages in a system. Here, we distinguish two aspects: first, the structural, which describes how messages {\em can} travel in a system and the operational, which describes how they {\em do} travel in some given scenario.

\subsubsection{Structural Notation}

The elements of a component statically determine which messages it can receive and send. Based on the behaviour of the filter, interpreter and processor of a component, we can express a number of properties.

\begin{definition}[Message reception]
	Let $C$ be a component and $m$ a message. $C$ can receive $m$ if and only if $\ft{C}\ m = \field{True}$ and $\int{C}\ m = \field{Just}\ m'$ for some $m'$.
	When $C$ can receive all messages in $\{m_1,\dots,m_n\}$, we write:
	$$
	\canrec{m_1,\dots,m_n}{C}.
	$$
	
	We denote the opposite statement, that $C$ cannot receive any message in $\{m_1,\dots,m_n\}$,~by:
	$$
	\cantrec{m_1,\dots,m_n}{C}.
	$$
\end{definition}

\begin{definition}[Message sending]
	Let $C$ be a component and $m, m_1,\dots,m_n$ messages. $C$ can send out a message $m$ if and only if there exists a message $m_{\mt{in}}$ such that $\proc{C}\ m_{\mt{in}} = m$.
	When $C$ can send all messages in $\{m_1,\dots,m_n\}$, we write:
	$$
	\cansend{C}{m_1,\dots,m_n}.
	$$
	
	The opposite statement, that $C$ cannot send any message in $\{m_1,\dots,m_n\}$, is denoted by:
	$$
	\cantsend{m_1,\dots,m_n}{C}.
	$$
\end{definition}

\begin{definition}[Receiving set]
	The set of components which can receive a message $m$ is denoted by
	$$
	\rec{m} \equiv \{C \in \co\ |\ \canrec{m}{C} \}.
	$$
	
	$\field{rec}$ can also be overloaded to refer to the set of components which can receive and interpret at least some message of a component $C$:
	$$
	\rec{C} \equiv \{C_i \in \co\ |\ \exists m:\ \cansend{C}{m} \wedge \canrec{m}{C_i} \}.
	$$
\end{definition}


\subsubsection{Operational Notation}

Whereas the structural notation pertained to the static properties of a neural system, the operational notation describes {\em traces}: lists of sent and received messages, and the changes they induced in the system.

\begin{definition}[Message action]
	When a component $C_i$ outputs a message $m_{out}$ that another component $C_j$ receives and interprets as message $m_{in}$, we write
	$$
	C_i \rightarrow [m_{out}, m_{in}] \rightarrow C_j.
	$$
	
	We refer to this as {\em message action}. If it is clear that the message $m$ does not change, we just write
	
	$$
	C_i \rightarrow [m] \rightarrow C_j.
	$$
\end{definition}

\begin{definition}[Trace]
	\emph{Traces} are defined inductively thus:
	\begin{enumerate}
		\item Every message action is a trace.
		\item If $T_1$ and $T_2$ are traces, $T_1;T_2$ is a trace.
	\end{enumerate}
	
	``$;$'' denotes sequential execution and is associative. Thus, the semantics of a trace $T_1;T_2;\dots;T_n$ are that $T_1$ is executed first, followed by $T_2$, and so forth, until $T_n$ is reached and the execution ends.
	For readability, $T_1;\dots;T_n$ will sometimes be written line-by-line as
	$$
	\begin{array}{l}
	T_1\\
	\vdots\\
	T_n
	\end{array}
	$$
\end{definition}

\begin{definition}[Component mutation]
	Let $f_1,f_2,\dots$ be functions $\compT{T}{I} \rightarrow \compT{T}{I}$ which preserve the names of components, $m,m'$ messages of type $T$ and $C$ a component of type $\compT{T}{I}$. When $C$ is changed into $(f_n \circ \dots \circ f_1)\ C$ by a message $m$ it receives, or changed into $(f_n \circ \dots \circ f_1)\ C$ by a message $m'$ it sends, we write, respectively:
	$$
	\begin{array}{c}
	\dots \rightarrow [m] \rightarrow \langle f_1,\dots,f_n \rangle C;\\
	C\langle f_1,\dots,f_n \rangle \rightarrow [m'] \rightarrow \dots.
	\end{array}
	$$
	
	\noindent
	If no change occurs, that is, if
	$$
	\begin{array}{r l}
	C\langle\rangle \rightarrow [m] \rightarrow \dots & \mt{or}\\
	\dots \rightarrow [m] \rightarrow \langle\rangle C
	\end{array}
	$$
	
	\noindent
	we omit the angle brackets.
	The semantics are as follows: after by sending or receiving a message, $\co$ is replaced by $(\co - \{C\}) \cup \{(f_n \circ \dots \circ f_1)\ C \}$.
\end{definition}

\begin{definition}[Plastic and non-plastic neural systems]
	If, for all messages $m$ and components $C, C'$ in a neural system:
	$$
	C\langle\rangle \rightarrow [m] \rightarrow \langle\rangle C'
	$$
	
	\noindent
	holds, we call the system \emph{non-plastic}. Otherwise, we call it \emph{plastic}.
\end{definition}


This definition intends to roughly convey the notion of neuroplasticity, as used in neuroscience: areas in the brain are changed over time through specific patterns of activity. Here, such change is modelled by the execution of functions and the replacement of $C$ in the system by $f_n \circ \dots \circ f_1 (C)$.

\begin{figure}[t]
	\centering
	\begin{subfigure}[t]{0.35\textwidth}
		\includegraphics[width=\textwidth]{Figs/c_co.png}
		\caption{Components communicating via a central organising mechanism.}
		\label{fig:centralOrganizer}
	\end{subfigure}
	\begin{subfigure}[t]{0.35\textwidth}
		\includegraphics[width=\textwidth]{Figs/c_sequence.png}
		\caption{A sequence of components.}
		\label{fig:cSequence}
	\end{subfigure}
	\caption{Two special-purpose component arrangements.}
\end{figure}

\subsection{Invariants}

Such a model does not necessitate the existence of special structures, such as central organizers or sequences of components, one activated after another,\footnote{An example of such a sequence is discussed by Sander et al.\ \cite{DBLP:journals/nn/SanderGS05} where the authors model the emotion process as a four-step pipeline of relevance, implication, coping and normative significance.} but it does not preclude them either. In fact, we can enforce certain features via first-order invariants. For example, a central organizing units for the components $C_1,\dots,C_n$ can be emulated by a component $C_{co}$ which accepts messages and transforms them into an appropriate format for the some other components.

\begin{invariant}[Central organiser]
	\label{inv:centralOrganiser}
	$$
	\begin{array}{l}
	[\forall i \in \{1\dots,n\}] [\forall m]:\\
	\quad \quad \left(\cansend{C_i}{m} \Rightarrow \rec{m} = \{C_{co}\}\right) \wedge \left( \left( \proc{C_{co}} \circ \int{C_{co}}(m)  \right) \in \bigcup\limits_{1 \leq j \leq n} \rec{C_j} \right).
	\end{array}
	$$
\end{invariant}

Figure~\ref{fig:centralOrganizer} depicts such an organiser, and Invariant~\ref{inv:centralOrganiser} gives a symbolic definition. In words, the invariant states that, for all messages $m$, if one of the peripheral components $C_1,\dots,C_n$ emits $m$, then the central component $C_{co}$ must receive it. In addition, at least one peripheral component must be able to receive each message which the central component $C_{co}$ outputs.

Similarly, sequences can be created by components $C_{1},\dots,C_{n}$, where each components reads the message of the last one. We see such an invariant depicted in Figure~\ref{fig:c_sequence} and formally described in Invariant~\ref{inv:cSequence}. The invariant simply states that the messages sent by a component $C_{i-1}$ can only be received by the next component $C_i$.

\begin{invariant}[Sequence]
	$$
	[\forall i \in \{2\dots,n\}]:\ \rec{C_{i-1}} = \{C_i\}.
	$$
\end{invariant}


	
	\chapter{Affective Architecture}\label{ch:affectiveArchitecture}.
	
	\input{4affectiveArchitecture}
	
	\chapter{Implementation}\label{ch:implementation}
	
	\input{5implementation}
	
	\chapter{Conclusion}\label{ch:conclusion}
	
	\section{Results}\label{sec:results}

results go here

\section{Future Work}\label{sec:futureWork}

In the course of the implementation and evaluation of the proof-of-concept accompanying this thesis, a number of possible improvement arose, which were not explored further but which can form the basis of fruitful future investigation. Specifically:

\begin{description}
	\item[Causality-based world simulation.] Presently, the agents create plans by taking hypothetical actions and simulating the world state as a result of these. As a consequence, the lengths of plans and the number of time-steps required to perform them correspond one-to-one.
	This schema is functional, but has apparent drawbacks when we compare it to the way in which humans plan actions: If, say, one wanted to go 100 steps in a straight line to get a glass of water, one would not consider each required step individually. Rather, one would summarize the required 100 steps as the single action ``walk in a straight line towards the glass''. Similarly, if one had to wait ten minutes for a train, one would not consider what to do during each second of the wait; one would simply resolve to ``sit there''. Clearly, not all actions or series of actions are explicated to the same degree in the minds of humans when they make plans.
	
	 It thus stands to reason that, during the planning process, one ought to consider a sort of {\em causal distance} --- that is, the number of actions which the agent regards as qualitatively distinct. As soon as we begin to group actions together and distinguish temporal from causal distance, the question during planning ceases to be ``how long will it take to achieve X?'' and becomes ``how complicated is it to achieve X?''
	 
	 \item[Goal-based planning.] Our planning scheme first selects an emotion to serve as the guiding one and then  proceeds to create hypothetical steps until the guiding emotion is either satisfied, leading the the plan's execution, or until a conflicting emotion overpowers it, leading to the plan's abortion. This is, once again, basically functional, but one could improve upon it by associating certain outcomes --- e.g. sating one's hunger or killing a Wumpus --- with certain emotions and selecting one of these as goals to reach. Agents would thus no longer seek to satisfy their dominant emotions by any means possibly, but by working towards specific goals. 
	 
	 \item[Emotional learning.] In conjunction with goal-based planning, one might also make the association of outcomes with emotions a dynamic one. Instead of outcomes being permanently associated with this or that emotion, agents would be able to learn what constitutes a ``good'' or ``bad'', or a ``pleasurable'', ``painful'' outcome. 
	 
	 \item[Inference about world-states and forgetting.] The agents' memory is merely a perfunctory fact-storage which remembers past perceptions about the world. Importantly, it does not incorporate inferences about likely changes which an agent might reasonably learn, such as the fact that plants regrow or that an agent which was last seen surrounded by 10 Wumpuses is likely dead now. The learning and application of such inferences about the likely, but not directly observed, changes in the world is an open-ended area of improvement, but carries the possibility of much-optimized behaviour.
	 
	 \item[Concept synthesis.] Although agents are able to experience individual facts about their surrounding world, they do not create larger concepts from these facts to serve as cognitive shortcuts. An agent might perceive three Wumpuses in front of it, say, but it has no concept of ``three Wumpuses'' or ``a horde of Wumpuses''. One can think of many other macro-concepts which would directly aid in the creation of efficient plans and reduce cognitive load: ``a dangerous area'', ``an aggressive agent'', ``a gathering-place for Wumpuses'', etc.
	 
	 \item[Evolution of neural nets.] Emotional reactions are currently hand-crafted; the personalities of agents customized by inserting different nets for individual emotions. One might instead allow emotions to evolve by applying genetic algorithms to the neural nets, selecting the best-performing agents in each generation and creating the agents of the next one through recombination and mutation of their parents.
\end{description}

	
	
	
	% Remove following line for the final thesis.
	%\input{intro.tex} % A short introduction to LaTeX.
	
	\backmatter
	
	% Use an optional list of figures.
	%\listoffigures % Starred version, i.e., \listoffigures*, removes the toc entry.
	
	% Use an optional list of tables.
	%\listoftables % Starred version, i.e., \listoftables*, removes the toc entry.
	
	% Use an optional list of alogrithms.
	%\listofalgorithms
	%\addcontentsline{toc}{chapter}{List of Algorithms}
	
	% Add an index.
	\printindex
	
	% Add a glossary.
	% \printglossaries
	
	% Add a bibliography.
	\bibliographystyle{plain}
	\bibliography{thesis}
	
\end{document}





%\hypersetup{linkcolor=black}
%\tableofcontents
%\hypersetup{linkcolor=DeepBlue}
%\newpage

%\begingroup
%\let\clearpage\relax
%\endgroup

\nocite{*}



\end{document}
