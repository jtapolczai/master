In this document, I will sketch a possible architecture of the human brain and a select few of its subsystems. The descriptions presented are supported by some empirical evidence, but I do not claim that they are straightforward transcriptions of neurological realities. The model is grounded substantially in evolutionary considerations, which provide the backdrop and the plausibility check for the claims presented herein.

Section~\ref{sec:preliminaries} outlines the basic considerations that lead to the model. Section~\ref{sec:schemaOfCognition} sketches the proposed model of the mind. Section~\ref{sec:mathematicalModel} presents the mathematical model. In Section~\ref{sec:selectedSubsystems}, we look at three concrete subsystems: sensory perception, counterfactual perception (imagination) and affect.

It should also be understood that everything in this document is, at best, a {\em rough} outline; it may be likened to a hexagon which approximates a circle: though (conjectured to be) basically correct, and useful, it is marred by significant incongruities with the object of its approximation.