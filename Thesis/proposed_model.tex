\documentclass[bibliography=totoc ]{scrartcl}

\usepackage{graphicx}
\usepackage{soul}
\usepackage{caption}
\usepackage{amsmath}
\usepackage{amssymb}
\usepackage{amsthm}
\usepackage{subcaption}
\usepackage[utf8]{inputenc}
\usepackage[top=1.5in, bottom=1.5in, left=1in, right=1in]{geometry}
\usepackage[usenames, dvipsnames]{xcolor}
\usepackage{framed}
\usepackage{booktabs}

% Tikz
\usepackage{tikz}
\usetikzlibrary{calc}
\usetikzlibrary{arrows.meta}

\makeatletter
\newtheoremstyle{break}
   {\topsep}{\topsep}%
   {\itshape}{}%
   {\bfseries}{}%
   { }
   {\thmname{#1}\thmnumber{\@ifnotempty{#1}{ }\@upn{#2}}%
    \thmnote{ {\bfseries(#3)}}.}% 
\makeatother
\theoremstyle{break}

\newtheorem{thm}{Theorem}
\newtheorem{definition}[thm]{Definition}
\newtheorem{invariant}[thm]{Invariant}
\newtheorem{example}[thm]{Example}
\newtheorem{notation}[thm]{Notation}

\newcommand{\mt}[1]{\textnormal{#1}}
\renewcommand{\tt}[1]{\texttt{#1}}
\colorlet{shadecolor}{yellow!60}
\newcommand{\highlight}[1]{\colorbox{yellow!70}{#1}}
\newcommand{\shade}[1]{\begin{shaded}#1\end{shaded}}

\newcommand{\N}{\mathbb{N}}
\newcommand{\Q}{\mathbb{Q}}
\newcommand{\R}{\mathbb{R}}
\newcommand{\B}{\mathbb{B}}
\newcommand{\C}{\mathbb{C}}
\newcommand{\allQ}[1]{\left[\forall #1 \right]}
\newcommand{\exQ}[1]{\left[\exists #1 \right]}

\newcommand{\co}{\mathbf{Co}}
\newcommand{\cansend}[2]{#1 \rightarrowtail \{#2\}}
\newcommand{\cantsend}[2]{#1 \stackrel{\times}{\rightarrowtail} \{#2\}}
\newcommand{\canrec}[2]{\{#1\} \rightarrowtail #2}
\newcommand{\cantrec}[2]{\{#1\} \stackrel{\times}{\rightarrowtail} #2}
\newcommand{\rec}[1]{\field{rec}(#1)}
\newcommand{\ft}[1]{\tt{ft}_{#1}}
\renewcommand{\int}[1]{\tt{int}_{#1}}
\newcommand{\proc}[1]{\tt{proc}_{#1}}
\newcommand{\sends}[3]{#1 \rightarrow [#2] \rightarrow #3}
\newcommand{\sendsm}[4]{#1 \rightarrow [#2 , #3] \rightarrow #4}
\newcommand{\sendsf}[6]{#1\langle #2 \rangle \rightarrow [#3,#4] \rightarrow \langle#5\rangle#6}

\newcommand{\compT}[2]{\tt{Comp}_{#1,#2}}

\DeclareMathOperator{\tint}{int}
\newcommand{\type}[1]{\mathtt{#1}}
\DeclareMathOperator{\cint}{cint}

% Symbol for the extended Wumpus world.
\newcommand{\wext}{\ensuremath{\mathcal{W}_{\mathrm{ext}}}}
\newcommand{\wjun}{\ensuremath{\mathcal{W}_{\mathrm{jun}}}}
\newcommand{\ssem}{\ensuremath{\mathrm{sem}}}

\newcommand{\dist}[2]{||{#1},{#2}||}

\newcommand{\tuple}[1]{\ensuremath{\langle #1 \rangle}}

\newcommand{\lp}{{\rm (}}
\newcommand{\rp}{{\rm )}}
\newcommand{\paren}[1]{\lp{#1}\rp}
% Graph Cartesian product
\newcommand{\gcp}{\Box}
% A field in a record (ADT)
\newcommand{\field}[1]{\mathtt{#1}}

\newcommand{\floor}[1]{\left\lfloor{#1}\right\rfloor}
\newcommand{\ceil}[1]{\left\lceil{#1}\right\rceil}

\newcommand{\action}[1]{\ensuremath{\mathsf{#1}}}

\newcommand{\degs}{{^\circ}}

\definecolor{PaleRed}{RGB}{202,61,58}
\definecolor{DeepRed}{RGB}{158,46,43}

\newcommand{\figDataContainer}[2]{%
\coordinate (a) at #1;
\coordinate (b) at ($(a) + (0.15,0.15)$);
\coordinate (c) at ($(b) + (0.15,0.15)$);

\shade [top color=PaleRed,bottom color=DeepRed]
($(c) + (0,0)$) -- ($(c) + (0,-2)$)
                to [out=340,in=180] ($(c) + (1,-2.2)$)
                to [out=0,in=170] ($(c) + (4,-1.5)$)
                -- ($(c) + (4,0)$);
                
\draw [very thick, opacity=0.1]
($(c) + (0,0)$) -- ($(c) + (0,-2)$)
                to [out=340,in=180] ($(c) + (1,-2.2)$)
                to [out=0,in=170] ($(c) + (4,-1.5)$)
                -- ($(c) + (4,0)$) 
                -- ($(c) + (0,0)$); 
                
\shade [top color=PaleRed,bottom color=DeepRed]
($(b) + (0,0)$) -- ($(b) + (0,-2)$)
                to [out=340,in=180] ($(b) + (1,-2.2)$)
                to [out=0,in=170] ($(b) + (4,-1.5)$)
                -- ($(b) + (4,0)$);
                
\draw [very thick, opacity=0.1]
($(b) + (0,0)$) -- ($(b) + (0,-2)$)
                to [out=340,in=180] ($(b) + (1,-2.2)$)
                to [out=0,in=170] ($(b) + (4,-1.5)$)
                -- ($(b) + (4,0)$) 
                -- ($(b) + (0,0)$); 
                
\shade [top color=PaleRed,bottom color=DeepRed]
($(a) + (0,0)$) -- ($(a) + (0,-2)$)
                to [out=340,in=180] ($(a) + (1,-2.2)$)
                to [out=0,in=170] ($(a) + (4,-1.5)$)
                -- ($(a) + (4,0)$);
                
\draw [thick, opacity=0.3]
($(a) + (0,0)$) -- ($(a) + (0,-2)$)
                to [out=340,in=180] ($(a) + (1,-2.2)$)
                to [out=0,in=170] ($(a) + (4,-1.5)$)
                -- ($(a) + (4,0)$) 
                -- ($(a) + (0,0)$); 
                
\node at ($(a) + (2,-1)$) {\color{white}{#2}};
}

\newcommand{\figProcessingComponent}[2]{%
\coordinate (a) at #1;
\shade [top color=NavyBlue,bottom color=RoyalBlue]
($(a) + (0,0)$) -- ($(a) + (0,-1)$)
                -- ($(a) + (1,-1)$)
                -- ($(a) + (1,0)$);
\draw [thick, opacity=0.4]
($(a) + (0,0)$) -- ($(a) + (0,-1)$)
                -- ($(a) + (1,-1)$)
                -- ($(a) + (1,0)$)
                -- ($(a) + (0,0)$);
                
\node at ($(a) + (0.5,-0.5)$) {\color{white}{#2}};
}

\newcommand{\figFilter}[2]{%
\coordinate (a) at #1;


\coordinate (a) at #1;

\shade [top color=LimeGreen,bottom color=OliveGreen]
($(a) + ({0.5+sin(45)/2}, {-0.5+(sin(45)/2)})$) arc [radius=0.5,start angle=45,end angle=315] -- ++($(0,{cos(45)})$);

\draw [thick, opacity=0.4]
($(a) + ({0.5+sin(45)/2}, {-0.5+(sin(45)/2)})$) arc [radius=0.5,start angle=45,end angle=315] -- ++($(0,{cos(45)})$);

\node at ($(a) + (0.5,-0.5)$) {\color{white}{#2}};
}
\newcommand{\co}{\mathbf{Co}}
\newcommand{\cansend}[2]{#1 \rightarrowtail \{#2\}}
\newcommand{\cantsend}[2]{#1 \multimap \{#2\}}
\newcommand{\canrec}[2]{\{#1\} \rightarrowtail #2}
\newcommand{\cantrec}[2]{\{#1\} \multimap #2}
\newcommand{\rec}[1]{\field{rec}(#1)}
\newcommand{\ft}[1]{\tt{ft}_{#1}}
\renewcommand{\int}[1]{\tt{int}_{#1}}
\newcommand{\proc}[1]{\tt{proc}_{#1}}
\newcommand{\sends}[3]{#1 \rightarrow [#2] \rightarrow #3}
\newcommand{\sendsm}[4]{#1 \rightarrow [#2 , #3] \rightarrow #4}
\newcommand{\sendsf}[6]{#1\langle #2 \rangle \rightarrow [#3,#4] \rightarrow \langle#5\rangle#6}

\newcommand{\compT}[2]{\tt{Comp}_{#1,#2}}

\newcommand{\field}[1]{\mathtt{#1}}

\DeclareMathOperator{\tint}{int}
\newcommand{\type}[1]{\mathtt{#1}}
\DeclareMathOperator{\cint}{cint}

\newcommand{\action}[1]{\ensuremath{\mathsf{#1}}}
\definecolor{PaleRed}{RGB}{202,61,58}
\definecolor{DeepRed}{RGB}{158,46,43}
\definecolor{PaleBlue}{RGB}{100,180,250}
\definecolor{DeepBlue}{RGB}{30,100,255}
\definecolor{DeepGreen}{RGB}{60,150,30}
\definecolor{LightGray}{RGB}{220,220,220}

\newcommand{\figDataContainer}[2]{%
\coordinate (a) at #1;
\coordinate (b) at ($(a) + (0.15,0.15)$);
\coordinate (c) at ($(b) + (0.15,0.15)$);

\shade [top color=PaleRed,bottom color=DeepRed]
($(c) + (0,0)$) -- ($(c) + (0,-2)$)
                to [out=340,in=180] ($(c) + (1,-2.2)$)
                to [out=0,in=170] ($(c) + (4,-1.5)$)
                -- ($(c) + (4,0)$);
                
\draw [very thick, opacity=0.1]
($(c) + (0,0)$) -- ($(c) + (0,-2)$)
                to [out=340,in=180] ($(c) + (1,-2.2)$)
                to [out=0,in=170] ($(c) + (4,-1.5)$)
                -- ($(c) + (4,0)$) 
                -- ($(c) + (0,0)$); 
                
\shade [top color=PaleRed,bottom color=DeepRed]
($(b) + (0,0)$) -- ($(b) + (0,-2)$)
                to [out=340,in=180] ($(b) + (1,-2.2)$)
                to [out=0,in=170] ($(b) + (4,-1.5)$)
                -- ($(b) + (4,0)$);
                
\draw [very thick, opacity=0.1]
($(b) + (0,0)$) -- ($(b) + (0,-2)$)
                to [out=340,in=180] ($(b) + (1,-2.2)$)
                to [out=0,in=170] ($(b) + (4,-1.5)$)
                -- ($(b) + (4,0)$) 
                -- ($(b) + (0,0)$); 
                
\shade [top color=PaleRed,bottom color=DeepRed]
($(a) + (0,0)$) -- ($(a) + (0,-2)$)
                to [out=340,in=180] ($(a) + (1,-2.2)$)
                to [out=0,in=170] ($(a) + (4,-1.5)$)
                -- ($(a) + (4,0)$);
                
\draw [thick, opacity=0.3]
($(a) + (0,0)$) -- ($(a) + (0,-2)$)
                to [out=340,in=180] ($(a) + (1,-2.2)$)
                to [out=0,in=170] ($(a) + (4,-1.5)$)
                -- ($(a) + (4,0)$) 
                -- ($(a) + (0,0)$); 
                
\node at ($(a) + (2,-1)$) {\color{white}{#2}};
}

\newcommand{\figProcessingComponent}[2]{%
\coordinate (a) at #1;
\shade [top color=PaleBlue,bottom color=RoyalBlue]
($(a) + (0,0)$) -- ($(a) + (0,-1)$)
                -- ($(a) + (1,-1)$)
                -- ($(a) + (1,0)$);
\draw [thick, opacity=0.4]
($(a) + (0,0)$) -- ($(a) + (0,-1)$)
                -- ($(a) + (1,-1)$)
                -- ($(a) + (1,0)$)
                -- ($(a) + (0,0)$);
                
\node at ($(a) + (0.5,-0.5)$) {\color{white}{#2}};
}

\newcommand{\figFilter}[2]{%
\coordinate (a) at #1;


\coordinate (a) at #1;

\shade [top color=LimeGreen,bottom color=OliveGreen]
($(a) + ({0.5+sin(45)/2}, {-0.5+(sin(45)/2)})$) arc [radius=0.5,start angle=45,end angle=315] -- ++($(0,{cos(45)})$);

\draw [thick, opacity=0.4]
($(a) + ({0.5+sin(45)/2}, {-0.5+(sin(45)/2)})$) arc [radius=0.5,start angle=45,end angle=315] -- ++($(0,{cos(45)})$);

\node at ($(a) + (0.5,-0.5)$) {\color{white}{#2}};
}

\newcommand{\figProcessingComponentBent}[2]{%
\coordinate (a) at #1;

\shade [top color=PaleBlue,bottom color=RoyalBlue]
($(a) + (0.1,0)$) -- ($(a) + (1.35,0)$)
                  -- ($(a) + (1.5,-0.5)$)
                  -- ($(a) + (1.35,-1)$)
                  -- ($(a) + (0.1,-1)$)
                  to [out=115,in=270] ($(a) + (0.0,-0.5)$)
                  to [out=90,in=245] ($(a) + (0.1,0)$);
                  
\draw [thick, opacity=0.4]
($(a) + (0.1,0)$) -- ($(a) + (1.35,0)$)
                  -- ($(a) + (1.5,-0.5)$)
                  -- ($(a) + (1.35,-1)$)
                  -- ($(a) + (0.1,-1)$)
                  to [out=115,in=270] ($(a) + (0.0,-0.5)$)
                  to [out=90,in=245] ($(a) + (0.1,0)$);

\node at ($(a) + (0.75,-0.5)$) {\color{white}{#2}};
}

\newcommand{\figProcessingComponentBentUp}[2]{%
\coordinate (a) at #1;

\shade [top color=PaleBlue,bottom color=RoyalBlue]
($(a) + (0,-1.4)$) -- ($(a) + (0,-0.15)$)
                  -- ($(a) + (0.5,0)$)
                  -- ($(a) + (1,-0.15)$)
                  -- ($(a) + (1,-1.4)$)
                  to [out=205,in=0] ($(a) + (0.5,-1.5)$)
                  to [out=180,in=335] ($(a) + (0,-1.4)$);
                  
\draw [thick, opacity=0.4]
($(a) + (0,-1.4)$) -- ($(a) + (0,-0.15)$)
                  -- ($(a) + (0.5,0)$)
                  -- ($(a) + (1,-0.15)$)
                  -- ($(a) + (1,-1.4)$)
                  to [out=205,in=0] ($(a) + (0.5,-1.5)$)
                  to [out=180,in=335] ($(a) + (0,-1.4)$);

\node at ($(a) + (0.5,-0.75)$) {\color{white}{#2}};
}

\newcommand{\cubeShape}[2]{%
		\coordinate (a) at #1;
	
		\fill [color=#2, opacity=0.4] (a) -- ++(0,1)
								     -- ++($({sqrt(2)/2},{sqrt(2)/2})$)
								     -- ++(1,0)
								     -- ++(0,-1)
								     -- ++($({-sqrt(2)/2},{-sqrt(2)/2})$);
		
		\draw [opacity=0.3] (a) -- ++(0,1)
								-- ++($({sqrt(2)/2},{sqrt(2)/2})$)
								-- ++(1,0)
								-- ++(0,-1)
								-- ++($({-sqrt(2)/2},{-sqrt(2)/2})$)
								-- ++(-1,0);
		\draw [opacity=0.3] (a) -- ++($({sqrt(2)/2},{sqrt(2)/2})$)
								-- ++(1,0);
		\draw [opacity=0.3] ($(a) + (0,1)$) -- ++(1,0)
							           	    -- ++($({sqrt(2)/2},{sqrt(2)/2})$);
		\draw [opacity=0.3] ($(a) + ({sqrt(2)/2},{sqrt(2)/2})$) -- ++(0,1);
		\draw [opacity=0.3] ($(a) + (1,0)$) -- ++(0,1);
}

% Simple cubes with a label

\newcommand{\figSimpleCube}[3]{%
	\coordinate (a) at #1;
	\shade [top color=#3,bottom color=#3,fill opacity=0.5]
	($(a) + (0,0)$) -- ($(a) + (0,-1)$)
	-- ($(a) + (1,-1)$)
	-- ($(a) + (1,0)$);
	%\draw [thick, opacity=0.4]
	%($(a) + (0,0)$) -- ($(a) + (0,-1)$)
	%-- ($(a) + (1,-1)$)
	%-- ($(a) + (1,0)$)
	%-- ($(a) + (0,0)$);
	
	\node at ($(a) + (0.5,-0.5)$) {\color{black}{#2}};
}

\newcommand{\figBlueCube}[2]{\figSimpleCube{#1}{#2}{PaleBlue}}
\newcommand{\figDarkGreenCube}[2]{\figSimpleCube{#1}{#2}{DeepGreen}}
\newcommand{\figRedCube}[2]{\figSimpleCube{#1}{#2}{PaleRed}}

\newcommand{\cubeLabel}[2]{\begin{tabular}{c}\textbf{``#1'':}\\\begin{tabular}{l}#2\end{tabular}\end{tabular}}
\newcommand{\stdlisting}{%
\lstset{
			mathescape,							%Damit der Math-Modus funktioniert
			language={},							%Sprache
			keywordstyle=\color{KeywordColor},	%Farbe f. Keywords
			commentstyle=\color{CommentColor},	%Farbe f. Kommentare
			stringstyle=\color{StringColor},	%Farbe f. Stringliterale
			morekeywords={procedure},			%Zus. Keywords (zu language)
			otherkeywords={},
			basicstyle=\textnormal\ttfamily\footnotesize,	%Textstil
			breaklines=true,					%Umbruch langer Zeilen
												%(sonst hängen sie über)
			escapechar=´,						%Alles zwischen 2 ´ ist
												%außerhalb der lstlistings-Umgebung
												%Kommentare im Code sind möglich mit:
												%´\comment{blabla}´
			frame=none,						%Art des Rahmens
			captionpos=t,						%Position des Titels
			tabsize=4,							%Tabs durch Spaces ersetzen
			showtabs=false,						%Bei true werden Tabs im Code mit kleinen
												%Futzeln angezeigt
			showspaces=false,					%Dasselbe mit Spaces
			showstringspaces=false				%Dasselbe mit Spaces in Stringliteralen
			numberstyle=\footnotesize,
			numbersep=5pt,
			%backgroundcolor = \color{white},
			stepnumber=1
			}%
}

\newcommand{\code}[1]{\Fcolorbox{Gray}{LightGray}{#1}}


\title{{\huge Design and Implementation of an Agent Architecture combining Emotions and Reasoning}}
\author{Janos Tapolczai}

\begin{document}

\maketitle


\begin{abstract}
Modern AI is generally divided into two schools of thought, separated by the used level of abstraction. We have, on one hand, the biologically inspired, low-level modelling which uses neural networks to imitate the workings of brains. On the other, we have a logic-oriented, high-level approach that tries design and implement ideal ways of thinking. In recent decades, the latter approach has triumphed and has displaced computational model based on neural networks. In this practically oriented thesis, I shall make the case that this was perhaps too hasty, and that, despite their many applications, purely logic- and planning-based algorithms are insufficient to model authentic intelligence. This work is composed of two parts; the first is an empirical argument for placing models of AI on evolutionary foundations, under the assumption that we can best understand the workings of a complex artefact like the animal brain by tracing its history; the second is a description of an implementation of a kind of affective agent based on these principles, serving as a proof-of-concept that one can create simple, animal-like intelligence with them.
\end{abstract}

\newpage

\hypersetup{linkcolor=black}
\tableofcontents
\hypersetup{linkcolor=DeepBlue}
\newpage

\begingroup
\let\clearpage\relax

\section{Introduction}

In this document, I will sketch a possible architecture of the human brain and a select few of its subsystems. The descriptions presented are supported by some empirical evidence, but I do not claim that they are straightforward transcriptions of neurological realities. The model is grounded substantially in evolutionary considerations, which provide the backdrop and the plausibility check for the claims presented herein.

Section~\ref{sec:preliminaries} outlines the basic considerations that lead to the model. Section~\ref{sec:schemaOfCognition} sketches the proposed model of the mind. Section~\ref{sec:mathematicalModel} presents the mathematical model. In Section~\ref{sec:selectedSubsystems}, we look at three concrete subsystems: sensory perception, counterfactual perception (imagination) and affect.

It should also be understood that everything in this document is, at best, a {\em rough} outline; it may be likened to a hexagon which approximates a circle: though (conjectured to be) basically correct, and useful, it is marred by significant incongruities with the object of its approximation.

\section{Related work}\label{sec:relatedWork}

This thesis falls into the category of cognitive architectures and the integrated approach to AI, pioneered by people like Rodney Brooks and his subsumption architecture, and \cite{brooksSubsumption}, Douglas Hofstaedter, who famously wrote about many aspects of AI in Gödel, Escher, Bach \cite{geb}, and who created the Copycat analogy-making program \cite{copycat}. 

\todo{cite others} \todo{describe their work: HCS, nouvelle AI?, 4catr, etc. - 1-2 paragraphs}.

The probably earliest example of a cognitive architecture was Allen Newell's and Herbert~A.~Simon's {\em Logic Theorist}, created in 1955 \todo{cite logic theorist}. Simon's theory of bounded rationality \cite{Gigerenzer2001} --- the idea of finding a merely satisfactory solution instead of a (provably) optimal one --- is very similar to the loop between belief generation and evaluation described in Section~\ref{sec:implementation}. In both cases, agents with limited information search heuristically for the first solution that they find acceptable. Unlike exhaustive search methods (e.g. A*), this does not guarantee the best possible results, but it is much more cost-effective and closer to the way real humans solve problems. In spirit, this is also similar to the {\em Procedural Reasoning System} of Michael Georgeff et al. \cite{pcs}, which is based on the belief-desire-intention model \cite{Rao95bdiagents, Bratman87}. Much theoretical work has been done on BDI, but it is only tangentially related to this thesis.\\

In terms of software engineering, our model has similarities, both to the Actor model, and to publish/subscribe architectures \todo{cite} --- although more as a concession to practicality and less because of a similarity to their theories. The theoretical basis of our implementation is the postulate that the components of the brain function as white boxes and that other components may listen in on their activity, so to speak. Since this is diametrically opposed to the traditional idea of the procedure/function as a black box, which nigh every programming language follows, we compromise and model the cognitive structure as a mesh of loosely coupled components communicating via passing. This description is reminiscent to the Actor model developed by Carl Hewitt et al. \cite{hewittActor}, although there are differences\footnote{Although I do not describe the implementation in the language of the Actor model, a translation into it would be quite easy. Such a translation would require using only very rudimentary features of the model, however, and as that is not the focus, I forego the task.}: in the Actor model, the topology of the network may change through the creation of new actors, and messages are always passed from one source to known targets (via addresses). In our model, on the other hand, there is no topology in a strict sense; messages are put into a global message storage and every component is free to consume any message it deems relevant. Senders do not know who will read their output, and consumers do not know the sources. This arrangement can be seen as a particularly loose variant of a publish/subscribe architecture, in which the source and the target of a message are completely unaware of each other, and in which there are no specific channels to which one may subscribe. The only criterion by which messages may be accepted or rejected is their content.

We also make use of already existing solutions --- specifically answer-set programming and the \acthex\ solver \dlvhex. The internal world simulation of our agents makes use of the non-monotonic reasoning provided by ASP and \acthex.  Answer-set programming was created by Gelfond and Lifschitz \cite{asp1}. Soon after them, Subrahmanian made the connection between ASP and planning \cite{Subrahmanian95relatingstable}. Together with Eiter and others, he later developed the \acthex\ language which allowed provided a framework for decision making in logic programming via external input and output atoms \cite{heterogeneous1, heterogeneous2, heterogeneous3}.

\section{Preliminary considerations}\label{sec:preliminaries}

\input{introduction2}

\section{Diagram notation}

\input{diagram}

\section{Schema of cognition}\label{sec:schemaOfCognition}

We can imagine the components of the mind as white boxes which inform other components by their very functioning --- however, this does not lend itself to easy implementation. Instead, we can emulate this behaviour via a \caps{message space}, from which individual components take their input and into which they put their output. A \caps{component} is then a local processing unit which continuously scans the message space, running messages through its \caps{filter}. If the filter detects a relevant message, it is then passed to the \caps{interpreter}, which parses the message into the needed format and hands it over to the \caps{processor}. The processor, after having finished, puts its output back into the message space for other other components to read. Figure~\ref{fig:global} illustrates this scheme. Note the lack of explicit hierarchical structure and central organising units.

\begin{figure}[!h]
	\centering
	\includegraphics[width=400pt]{figs/global.png}
	\caption{Global neural architecture.}
	\label{fig:global}
\end{figure}

However, as I'll show in the next section, this model is generic enough to accommodate such special-purpose structures. Figure~\ref{fig:global} shows the message-passing scheme, but it also specifies a graph in which the nodes are the components and fixed, while the edges are the accepted messages and are determined by the nodes; through their filters, components control the shape of the graph. By imposing invariants on these filters, we can have the graph take any shape we desire. In particular, we can model the kinds of structures that occur in many other cognitive models and in empirical research: central organisers, sequences of components (``pipelines''), localized messages affecting only a small part of the mind, a component reading its own messages, loops and iterative messages between two or more components et cetera.

\pagebreak

\paragraph{Messages}

We may now ask how such messages between components are structured. Here, I make two empirical claims:
\begin{enumerate}
	\item messages have a priority and
	\item they are effectively unstructured.
\end{enumerate}

\begin{figure}[!h]
	\centering
	\includegraphics[width=150pt]{figs/message.png}
	\caption{Structure of a neural message.}
	\label{fig:message}
\end{figure}

To the best of my knowledge, the veracity of either has thus far not been determined by neuroscience. For the first, Marvin Minsky's ``The Emotion Machine'' provides some circumstantial evidence \cite[p. 222]{emotionMachine}:

\begin{quote}
	Of course, when one activates two or more Critics or Selectors, this is likely to cause some conflicts, because two different resources might try to turn on a third resource both {\em on} and {\em off}. To deal with this, we could design the system to use various policies like these:
	
	\begin{enumerate}
		\item Choose the resource with the highest priority.
		\item Choose the one that is most strongly aroused.
		\item Choose the one that gives the most specific advice.
		\item Have them all compete in some ``marketplace''.
	\end{enumerate}
\end{quote}

The selection strategies Minsky lists imply that there is some mechanism in the brain to determine the urgency of a signal. While it is possible that higher brain functions like reasoning or affect make an additional, rational evaluation, sensations like intense pain, bright lights, or great sadness can likely be communicated most easily by the appropriate components causing a flood of activity which, by its very intensity, informs other components of the urgency of their messages.

The second claim --- that messages are essentially unstructured --- means that there is no common, agreed-upon format in which they are stored. In addition to the evolutionary implausibility of such a format being created, an unstructured message format is in line with the white-box nature of components: since components merely ``listen in'' on others, and since each components will have its own pattern of activity, a listener would simply have to try and make sense of this activity as best it could. The proposed structure of messages is thus shown in Figure~\ref{fig:message}: every message comprises a priority header, together with an unstructured body which, for our purposes, is simply a string of bits.

\paragraph{Filters} Before a component can respond to a message by another, such a message must be assessed for the presence of relevant information. Conceptually, this happens via a \caps{filter} in each component, which pattern-matches incoming messages and, if a certain threshold is reached, signals relevance and hands the message over the \caps{interpreter} for parsing. Figure~\ref{fig:filter} shows such a filter: it is composed of a directed graph of nodes, and a node is activated if it detects some specific content in the message. Nodes, in turn, are connected via edges of strength $\in [0,1]$. When a node is activated, it sends a charge proportional to the strength of its link to its neighbours, contributing to their activation as well. Some nodes are marked as {\em output nodes}; if enough such output nodes become activated, the message is deemed to be sufficiently relevant. This model of filters is inspired by the {\em spiking neural P Systems} of Georghe Pa\u{u}n et al. (\cite[p. 337]{membraneComputing} and \cite{spikingNeural}), in which charges sent along directed graphs of neurons are used to compute functions.

\begin{figure}[!h]
	\centering
	\includegraphics[width=168pt]{figs/filter.png}
	\caption{A pattern-matching filter for a component $C_i$.}
	\label{fig:filter}
\end{figure}

\section{Mathematical model}\label{sec:mathematicalModel}

\input{mathematicalModel}

\section{Selected subsystems}\label{sec:selectedSubsystems}

The global architecture now specified, we will introduce three related subsystems and fit them into this global framework: sensory perception --- the processing of raw sensory input into an format intelligible to other brain components ---, counterfactual perception --- the imagination, which mimics the output of the senses ---, and affect --- broadly speaking, the emotional component of cognition.

\subsection{Sensory perception}\label{sec:sensoryPerception}

The model presented herein is inspired by Marvin Minsky's ``The Emotion Machine''. Therein, Minsky proposes a layered mental structure where each successive layer operates on more and more abstract representations of the world, starting with primitive sensations and proceeding all the way to self-conscious reflection and rational planning. Figure~\ref{fig:brainLayers} shows such a layered structure.

 \begin{figure}[!h]
 	\centering
 	\includegraphics[width=300pt]{figs/emotionMachine_brainLayers.png}
 	\caption{Layered perception of the world, from \cite[p. 100]{emotionMachine}.}
 	\label{fig:brainLayers}
 \end{figure}
 
 \newpage
 
The diagram is explained thus \cite[p. 100]{emotionMachine}:

\begin{quote}
	Now suppose that your A-Brain gets some signals from the external world (via such organs as eyes, ears, nose, and skin) --- and that it also can react to these by sending signals that make your muscles move. By itself, the A-Brain is a separate animal that only reacts to external events but has no sense of what they might mean. For example, when the fingertips of two lovers come into intimate physical contact, {\em the resulting sensations, by themselves, have no particular implications}. For there is no significance in those signals themselves: their meanings to those lovers {\em lie in how they prepresent and process them in the higher levels of their minds.}
\end{quote}

If we apply this to the architecture of Section~\ref{fig:global}, we can devise a system in which each sense $S$ has an associated component $C_S$ which does two things:
\begin{enumerate}
	\item Consume the raw sensory information delivered by various organs and output processed input for higher brain functions;
	\item as a side a effect of this processing, cause  instinctive, low-level reactions in the body, such as pulling away from pain or jumping at a sudden fright.
\end{enumerate}

In Figure~\ref{fig:sensoryPerception}, a slice of just such a system is shown for visual, auditory, olfactory/gustatory and tactile sensation. The produced data can be of two kinds: one is more abstract than the input and facilitates deliberative action, and the other contains instructions for instinctive behaviour for the body.

\begin{figure}[!h]
	\centering
	\includegraphics[width=325pt]{figs/sensoryPerception.png}
	\caption{Partial structure of sensory perception - raw sensory data is processed and made available to higher functions such as the affective subsystem. The comment ``Possible side-effect: sensory experience'' signifies the fact that conscious and sub-conscious sensory experiences might occur as a side-effect of this processing. However, it is currently unknown to neuroscience whether this is indeed the case.}
	\label{fig:sensoryPerception}
\end{figure}


\subsection{Counterfactual perception and planning}\label{sec:worldSimulation}

Broadly speaking, counterfactual perception can be described as ``imagination'', and is closely related to sensory perception and world simulation. In examining the system, we might broadly classify its processes into three categories:

\begin{enumerate}
	\item Counterfactual perception --- imagining sights, sounds, etc. Such experiences have much in common with those caused by our sensory organs, yet are marked not as real. In particular, imagined experiences evoke only parts of the conscious experience that accompanies real perceptions. Research by Berthoz and Lotze et al.\ suggests that (a) the brain indeed uses similar circuitry for real and imagined experiences and that (b) imagined experiences are prevented from being confused with real ones via inhibitory signals. Lotze et al.\ write \cite{lotze1999}:
	\begin{quote}
		The results of cortical activity support the hypothesis that motor imagery and motor performance possess similar neural substrates. The differential activation in the cerebellum during EM and IM is in accordance with the assumption that the posterior cerebellum is involved in the inhibition of movement execution during imagination.
	\end{quote}
	
	From the abstract of Berthoz's paper \cite{8713551}:
	
	\begin{quotation}
		(...) experimental evidence suggesting that the brain can use the same mechanisms for the imagination and the execution of movement. In particular the fact that adaptation of the vestibulo-ocular reflex can be obtained by pure mental effort and not solely by conflicting visual and vestibular cues has been suggestive of the fact that the brain could internally simulate conflicts and use the same adaptive mechanisms used when actual sensory cues were in conflict.
	\end{quotation}
	
	\item World simulation --- the imagination of future states. Simulating worlds goes beyond the imagination of sensory experiences; it involves constructing models of worlds and simulating their behaviour. The details of this process are unknown, but we can assert that it is capable of a number of things:
	\begin{enumerate}
		\item construction of non-physical worlds, such as mathematical models,
		\item extrapolation into the future and the past
		\item simulation of the minds itself and other agents.
	\end{enumerate}
	
	
	\item Executive planning --- humans can plan both both in immediate and concrete terms (such as body movement) and in the abstract. It is likely that different circuitry is used for movement planning and for planning involving abstract reasoning, in both cases it is necessary that the brain simulate the world in some way. The simulation of the consequences of body movement is likely older than humanity and distinct from the kind of world simulation described above, but both share their function: the agent proposes as series of actions to take, inserts them into some mental world and judges the utility of those actions based on the predicted consequences.
\end{enumerate}

Needless to say, that this process in all its subtleties is immensely complex and thus I simply endeavour to sketch its possible structure only in extremely rough outlines. This sketch is shown in Figures~\ref{fig:imagination},  \ref{fig:planner}, and \ref{fig:worldSimulatorPlannerInteraction}: the world simulation is an ordinary component with a filter and interpreter which outputs, for simplicity's sake, messages marked as counterfactual. We can imagine such messages to be very much like ordinary sensory ones, with the exceptions that they have no accompanying sensation and, more importantly, that we are aware of their non-reality. The planning component receives instructions about desirable states and outputs hypothetical actions which the world simulator incorporates. The world simulator's output is in turn read by the planner, which then abandons the plan or decides to pursue it further.

\begin{figure}
	\centering
	\includegraphics[width=\textwidth]{figs/imagination.png}
	\caption{Structure of of counterfactual perception \& world simulation: messages emulating the output of sensory perception are generated, but are marked as counterfactual by unknown means.}
	\label{fig:imagination}
\end{figure}

\begin{figure}
	\centering
	\includegraphics[width=\textwidth]{figs/planner.png}
	\caption{Planner with two kinds of inputs: (1) real sensory data and (2) counterfactual data which comes from world simulation. On the basis of these inputs, possible steps are developed and sent out as commands.}
	\label{fig:planner}
\end{figure}

The planner, minimally, has to perform two functions --- first, it has to judge the desirability of various world states and second, it has to be able to devise possible steps for the agent based on some strategy. If these two functions and some desired goal(s) are given, the planner can do its work by issuing the following commands, as shown in Figure~\ref{fig:planner}:
\begin{enumerate}
	\item If some goals are not yet reached but appear possible, devise possible steps to take and have the world simulator predict their outcomes.
	\item If the goals appear impossible the necessary steps prohibitively undesirable, command the world simulator to cease its activity.
	\item If earlier proposed steps turn out to fulfil some goal, contact the agent's executive component.
\end{enumerate}

\begin{figure}
	\centering
	\includegraphics[width=\textwidth]{figs/worldSimulatorPlannerInteraction.png}
	\caption{Interaction between world simulator and planner: the planner devises possible steps and feeds them into the world simulator, which, in turn, tries to calculate their effects. The results are fed back to the planner.}
	\label{fig:worldSimulatorPlannerInteraction}
\end{figure}

\subsubsection{World simulation as rationality}
The way in which I just described the interaction between the world simulator and the planner suggests that they function as a pair of guesser and checker: the planner generates ideas on what to do and the world simulation tests their viability in some setting. Indeed, we can model rational thinking as embedded in the world simulator, especially if we make use of a plastic neural system. The proposed steps of the planner might be quite chaotic and irrational, but when given to the world simulator, it recognises them as such and returns a failure signal to the planner, causing it to abandon ``bad'' paths of cognition. A plastic planner can learn from the consistent failure of certain kinds of steps and, in time, propose them less and less often. Observed as a whole, this system of planner and simulator appears to simply deliver good plans by intuition, even though, in isolation, neither part is very clever.\footnote{I do not wish to idealize rationality too much; world simulation is only partly rational and, given faulty information about the world, will err considerably and in documented ways. Similarly, it is certainly possible for the planner to derange the world simulator by evaluating certain states as so desirable/undesirable that it will pursue even scenarios which the world simulator reports as highly unlikely.}

\paragraph{Model.} In a simplified way, we can model the process of logical deduction in a formal system $F = (A, R)$, where $A$ is a recursive set of axioms and $R$ is a recursive set of production rules of the form $(r_{\mt{from}}, r_{\mt{to}})$ s.t. $r_{\mt{from}} \rightarrow r_{\mt{to}}$ is a valid production in the system. Let
	\begin{enumerate}
		\item $W$ be a world simulator for the world of propositions $\mathcal{P}$ in $(A,R)$,
		\item $P$ a planner,
		\item $St = \{s_1,\dots,s_p\}$ a set of messages about steps to take,
		\item $Cat = \{K_1,\dots,K_q\}$ a list of message categories,
		\item $\tt{cur}$ the current state of the world simulator,
		\item $\tt{ins}\backslash 2$, $\tt{del}\backslash 1$ functions for inserting or deleting a state change into the world simulator or the planner,
		\item $t(i)$ and $b(i)$ functions which increase or decrease the likelihood of sending a message belonging to category $K_i$ and 
		\item $\bot_{i}, \top_{i}$ the failure and success signals of a message belonging to the category $K_i$.
	\end{enumerate}
	
One step of the interaction between $W$ and $P$, in a scenario where $P$ proposes steps $s_{i_1},\dots,s_{i_n}$, can then be modelled with two traces $T_{\mt{guess}}$ and $T_{\mt{check}}$:

$$
	\begin{array}{l l l}
		T_{\mt{guess}}(\tt{step}) & \equiv & \sendsf{P}{\tt{ins}(\tt{cur}, \tt{step})}{\tt{step}}{\tt{step}}{\tt{ins}(\tt{cur}, \tt{step})}{W}\\
		\\
		T_{\mt{check}}(\tt{step}) & \equiv &
		\forall K_i \in Cat: K_i(\tt{step}) \Rightarrow\\
		
		& & \hspace{1cm} \mt{if } \exists (cur, s_j) \in R\ \mt{ then } \
					\sendsf{W}{}{\top_i}{\top_i}{t(i)}{P}\\
		& & \hspace{1cm} \mt{else }\ \sendsf{W}{\tt{del}(\tt{step})}{\bot_i}{\bot_i}{\tt{del}(\tt{step}), b(i)}{P}\\
	\end{array}
$$

\medskip

Axioms can be selected by executing $T_{\mt{guess}}(\tt{ax})$ for all $\tt{ax} \in A$. We can then perform deduction via $T_{\mt{guess}};T_{\mt{check}}$, for a probabilistically selected $\tt{step} \in St$.

Intuitively, $T_{\mt{guess}}$ guesses a step to take. It does so but inserting it into the planner's world-state via $\tt{ins}$ and then sending a message to the world simulator, which also inserts it into its world state. $T_{\mt{check}}$ then checks whether the change from $\tt{cur}$ to $\tt{step}$ was legitimate. If so, it determines to which category $\tt{step}$ belongs and sends the $\top$-signal for that category back to the planner. Otherwise, it sends the corresponding $\bot$-signal. The purpose of this is to make it more or less likely, respectively, that the planner should choose the same category of step in the future. The categories, we can imagine, could be things like ``modus ponens'', ``associative reasoning'', ``appeal to consequences'' and so forth.

If we repeat this interaction (with different proposed steps $s_1,\dots,s_p$ in each iteration), we get an algorithm for logical deduction  --- that is, since $A$ and $R$ are recursive, the system will recursively enumerate all valid logical formulas, provided that we pursue each path and that the probability of selecting any valid step is $> 0$. In addition, we could add a goal function $g$ to $P$ s.t. it would accept certain states and stop. Thereby, $P$ and $W$ could be used to prove logical propositions.

\subsection{Affect}\label{sec:affect}

When discussing human affect, one can mean various things: the causation of emotion, its internal mechanisms, the expression of emotion, social communication of emotions, etc. In this document, we restrict our attention just to the internal mechanisms --- that is, to the means by which emotions are evoked in an agent and how they shape its thinking.

Furthermore, the issue will only be the causative mechanism itself; taxonomy and hierarchy of emotions are deferred to future versions of this document.

The model presented herein is adapted from Gadanho and Hallam \cite{DBLP:journals/adb/GadanhoH01}, who employed it in the context of robot learning. They constructed a system of \caps{feelings} and \caps{sensations} $\mathcal{F}$, \caps{emotions} $\mathcal{E}$, and a hormone storage $H$.

\begin{figure}[!h]
	\centering
	\includegraphics[width=200pt]{figs/gadanhoModel.png}
	\caption{Emotional model of Gadanho and Hallam \cite[p. 46]{DBLP:journals/adb/GadanhoH01}.}
	\label{fig:gadanhoModel}
\end{figure}

Figure~\ref{fig:gadanhoModel} shows this model: \caps{sensations} enter the system and are connected to the \caps{feelings}. They, in turn, determine the agent's \caps{emotions}. The emotions then feed into a \caps{hormone storage}, the contents of which influence, together with the \caps{sensations}, the agent's \caps{feelings}. In the context of their paper, this model had a very restricted application. Its purpose was to merely help guide a robot through a world, and accordingly, $\mathcal{F}$ and $\mathcal{E}$ were only defined as \cite[p. 47]{DBLP:journals/adb/GadanhoH01}:
$$
	\begin{array}{l}
		\mathcal{F} = \{ \mt{Hunger}, \mt{Pain}, \mt{Restlessness},
						 \mt{Temperature}, \mt{Eating}, \mt{Smell},
						 \mt{Eating}, \mt{Proximity} \}\\
		\mathcal{E} = \{ \mt{Happiness}, \mt{Sadness}, \mt{Fear},
						 \mt{Anger} \}
	\end{array}
$$

\begin{figure}[!h]
	\centering
	\includegraphics[width=400pt]{figs/affectiveSubsystem.png}
	\caption{Affective subsystem; specialisation of the global neural architecture. In plastic neural systems, selections may change over time.}
	\label{fig:affectiveSubsystem}
\end{figure}

\pagebreak
The main advantage of Gadanho's and Hallam's model is that (a) it is sufficiently generic to accommodate various schemas and (b) posits an internal state (the hormone storage), giving agents a certain inertia. For example, one can imagine integrating a many-dimensional model like Brazeal's \cite{breazeal2003} detailed taxonomy of emotion like Ortony's OCC model \cite{ortony1988}. The existence of an internal state is necessitated by the simple observation that our internal world is not solely dependent on momentary stimuli, but merely influenced by them. The idea of a hormone storage might be a simplistic approximation but it, too, can be refined as needed.\footnote{It might be tempting to simply replace the hormone storage with the message space, but doing so would ignore the role that neurotransmitters like dopamine and serotonin play in cognition, irrespective of the purely computational activity of brain components.} Figure~\ref{fig:sensoryPerception} shows the adapted model. The general structure was retained, but the set of sensations was replaced by the sensory processor described in Section~\ref{sec:sensoryPerception} and, instead of a single dominant emotion, competing emotions simply emit messages which are used by execute components and the world simulation.

\subsubsection{Affective subsystems}

In this section, I will develop the concept of ``emotion'' in greater detail. The process shown in Figure~\ref{fig:affectiveSubsystem} might suggest we simply have a collection of emotions and that all emotions are essentially equal, but I submit that this is not so. Instead, I propose the existence of various subsystems, each responsible for a group of emotions, and each with its own history and distinctive tasks. In the rest of this work, the following two assumptions will be made:

\begin{enumerate}
	\item {\em ``Emotion'' is not a singular phenomenon.} Specifically, this is contradicts many-dimensional models of emotions which propose one, two, three or four axes and a corresponding vector space in which every emotion is a point. Such a view implies that all emotions share a neurological template which is parametrized with coordinates to result in different experiences.
	\item {\em There exist emotions which are both different in kind and which pertain to different subsystems in the brain.} This implies that emotions cannot morally be seen as a homogeneous set $\{E_1,\dots,E_n\}$. Instead, a number of distinct subsystems are necessitated, each responsible for the causation and processing of a group of emotions. Given this, the only substantial aspect any two emotions might have in common would be our referring to both of them as ``emotion''.
\end{enumerate}

Both of these assumptions are rather concrete and thus deserve evidence. In 1999, Davidson and Irwin, using PET and fMRI scanning, found two different systems mediating approach- and avoidance related behaviors \cite[p. 13]{davidson1999}:

\begin{quote}
A large body of lesion, neuroimaging and electrophysiological data supports the view that the prefrontal cortex (PFC) is an important part of the circuitry that implements both positive and negative affect. ($\dots$)
A number of early studies that evaluated mood subsquent to brain damage suggested that patients with damage to the left hemisphere, particularly in PFC, were more likely to develop deppressive symptoms compared with patients having lesions in homologous regions of the right hemisphere. ($\dots$)
The general finding of left dorso-lateral PFC damage increasing the likelihood of deppressive symptoms has been interpreted to reflect the contribution of this cortical territory to certain features of positive affect, which, when disrupted, increases the probability of depressive symptomatology.
\end{quote}

In this, they echo earlies findings by Cacioppo et al.\ \cite{cacioppo1999}, Gray \cite{gray1994} and Lang et al.\ \cite{lang1990} that affect is lateralized, with different hemispheres being responsible for different categories of feeling. It therefore stands to reason that different emotions, being generated by different brain regions, should therefore also be different in their character.

Further, much research has been done in the area of so-called {\em basic emotions} --- a small set of emotions are acknowledged as being both elementary and characteristically distinct from each other. The Cambridge Handbook of Affective Neuroscience provides a good overview of the basic emotion theory \cite[pp. 9-10]{cambridgeAff}. Matsumoto and Eckman \cite{matsumoto2009}, for instance, identified seven basic emotions: happiness, surprise, contempt, sadness, fear, disgust, and anger.

Damasio \cite{damasio1998}, drawing upon neuroscientific findings, sketches a model of affect mainly involving the prefrontal cortex, but also the amygdala, the hypothalamus, and the anterior cingulate cortex, as seen in Figure~\ref{fig:damasioSystem}.

\begin{figure}
	\centering
	\includegraphics[width=250pt]{figs/damasioSystem.png}
	\caption{Neurological structure of affect, according to Damasio \cite{damasio1998}.}
	\label{fig:damasioSystem}
\end{figure}

In the same article, he describes how different brain regions are responsible for different kinds of emotion:

\begin{quotation}
	Equally problematic is the widespread view that the limbic system is the neural basis for all emotions. A rich body of evidence tells us that this is just not the case. Both within and around the limbic system, circuitry connection varied neural sites supports the operation of different emotion. For instance, work on aversive conditioning in rodents has shown that the amygdala is certainly involved in negative emotions such as fear [10,6]. {\em Work in humans, on the other hand, has not only confirmed the amygdala's involvement in negative emotions such as fear and anger, but also shown that the amygdala is not involved in the processing of positive emotions such as happiness, or negative emotions such as disgust.} [emphasis mine]
\end{quotation}

The last sentence of that quotation is especially revealing: it states that the neurological distinction is not simply one between positive and negative, or one between approach- or avoidance-related emotions, but that each emotion has its own profile of neurological activity and involves its own peculiar set of brain structures.

These facts make it quite clear that emotions are not simply homogeneous phenomena, being induced by a single system in the brain; rather, they are different in character and in the neural structures they involve. 

\paragraph{Structure of affect} The system depicted in Figure~\ref{fig:affectiveSubsystem} left several parts unspecified: the sensory processor $\Pr$, the emotion selectors $S_1,\dots,S_n$ and the messages sent by the chosen emotions into the message space. In the following paragraphs, I will flesh out that model in greater detail, building principally on the work of Sander, Grandjean and Scherer \cite{DBLP:journals/nn/SanderGS05}. Sander and colleagues partitioned the emotion process into four stages, as shown in Figure~\ref{fig:sanderSystem}. The first is {\em relevance}, which functions as a filter and detects the intrinsic pleasantness and the level of (emotional) attention that a stimulus demands. The processes of this stage, roughly speaking, correspond to the work of the sensory processor $\Pr$. The second stage is {\em implication}, where reasoning becomes engaged in order to determine the cause, likely outcome, and urgency of the perceived facts. At this stage, emotions like joy, anger, contentment, disgust, etc. are evoked, together with approach- and avoidance-related behaviours --- this corresponds to the emotion selectors $S_1,\dots,S_n$. Deliberate strategies come only in the next stage: {\em coping}. In it, reasoning and planning become fully engaged. The fourth stage is {\em normative significance} and deals, in essence, with moral concerns, both internal and those of other agents.

\begin{figure}
	\centering
	\includegraphics[width=440pt]{figs/sanderSystem.png}
	\caption{The four-stage emotion process according to Sander et al, consisting of relevance, implication, coping and normative significance.}
	\label{fig:sanderSystem}
\end{figure}

The system of Sander et al.\ presents the {\em vertical view} (the pipeline) of emotion; however, it does not describe the broader context into which this system is embedded. In particular, it does not address the interactions with perception, memory, and reasoning. Based on the evidence discussed above, I shall now present a {\em horizontal view} and construct a model of the hypothesized emotional subsystems and their interactions with other parts of the brain. Since no established vocabulary seems to exist in this specific are I shall first introduce a number of terms.

\begin{definition}[Evocative system]
An evocative system is a subsystem in the brain responsible for evoking consciously experienced affect within an agent based on internal or external stimuli.
\end{definition} 

Various such evocative systems can be imagined. I propose the following rough categorization:

\begin{description}
	\item[Pre-social emotions.] Certain behavioural mechanisms can be observed in non-social as well as social animals. The flight-or-flight instinct, for example, is nearly universal, as is the inclination to seek out food, shelter, and other resources. ``Instinct'' is indeed a more appropriate term in the case of most species, rather than ``emotion'', which connotes a certain richness of experience. Nonetheless, we can clearly see that, in more intelligent, social animals, emotions like anger, fear, and joy, have grown out of just these instincts. Hence the term ``pre-social emotions'': while emotion itself is quite possibly inherently social, certain emotions are rooted in instincts which are not, and an emotional animal would feel them even if it were the only one of its kind in an environment.
	
	\item[Social emotions.] A by far richer subset of emotions are the social ones. Indeed, social situations are the ones where affect can and must truly shine: the presence of other individuals the entire tribe demand a variety of affect relating to the appraisal of the agents, sympathy/antipathy, respect/contempt, the appraisal of oneself, showing dominance or submission, influencing other group members, taking action as a group, judging the behaviour of agents against norms, etc. It is also in social emotions in which it even makes sense to {\em show} emotion: facial expressions and gestures provide the signalling and mechanism needed for group coherence and coordinated action.
	
	We can identify several subsystems in the category of social emotion:
	
	\begin{enumerate}
		\item Reflective judgement about oneself in relation to the group or to abstract norms, primarily pride and shame \cite{Teroni2008}, but possibly also jealousy and humiliation (which, in contrast to shame, is attributed to external causes) \cite{fontaine2009};
		\item other-related judgement which determines whether to feel sympathy or antipathy, compassion, respect or contempt, trust or distrust for other individuals;
		\item normative judgement, which determines whether others or oneself is acting in accordance with instinctive or cultural norms.
	\end{enumerate}
	
	Other classifications are also possible. Haidt \cite{haidt2003}, for example, identifies those that are other-condemning (disgust, contempt), self-conscious (shame, embarrassment), other-suffering (compassion), other-praising (gratitude, awe). The picture is immensely complex and the neurological structure is presently not known. For the purposes of this thesis, we will therefore content ourselves with only this roughest of outlines.
	
	\item[Aesthetic emotions.] This type of emotion is perhaps the least studied in neuroscience and AI.
	
	%TODO
\end{description}

The emotions just listed can all be found in the more extensive taxonomies, chiefly among them in Ortony's OCC model \cite{ortony1988}. The taxonomies, however, tend to neglect the underlying neurology and the chronology of the development of these systems. Ortony's classification specifically is persuasive up to a point, but, despite it being fine-grained, one is left wondering about the underlying structure: which emotions are caused by the same brain regions, what structure, if any, do two given emotions share, to what degree is the classification scheme isomorphic to the actual neurology?

%TODO

\begin{definition}[Executive system]
An executive system is a subsystem in the brain which regulates attention, makes decisions, and causes deliberate motor action in an agent.
\end{definition}

Up to this point, the executive functions have not been discussed in detail, as they are not the focus of this work. Any description of affect, however, would be incomplete without them, as attentional and behavioural changes are the sole point of emotions. Very broadly speaking, we can divide executive functions into four categories:

\begin{enumerate}
	\item \textbf{sub-conscious motor control}: instinctive reaction, such as the jerking away from pain, jumping when startled, and turning towards interesting visual stimuli;
	\item \textbf{conscious motor control}: deliberate, planned action which the agent experiences as a choice;
	\item \textbf{sub-conscious mental control}: involuntary but consciously experienced changes to the mind-state of an agent which are perceived as activity rather than mere feeling. This includes like obsessing over an issue, manias, fantasies insofar as involuntary, etc.
	\item \textbf{conscious mental control}: deliberate mental changes of an agent. This includes the making of decisions, the deliberate focusing of attention, deliberate planning, deliberate strategy selection, and so forth.
\end{enumerate}

\paragraph{Pre-social behavior control} 

\paragraph{Social judgement system}

\paragraph{Social reflective system}

\paragraph{Aesthetic judgement system}

\subsection{Interaction between affect and world simulation}

Section~\ref{sec:worldSimulation} outlined what could be called {\em deliberate action} in the from of a planner-world-simulator loop. Section~\ref{sec:affect} described the structure and components of affect. These systems are of course not isolated from each other; emotional states influence both the planner's chosen heuristics and the world simulator's creation of worlds. In addition, attention, also influenced by affect, controls the allocation of cognitive resources. We now explore these relationships in further detail.

\paragraph{Planning as search} In the AI literature, search algorithms are of great importance. In this context, we can view the loop between planner and world simulator as a greedy search: the planner chooses the nodes which are to be expanded and sends them to the world simulator. It, in turn, performs the expansion by simulating the appropriate worlds. These simulated worlds are sent back to the planner for evaluation regarding desirability (i.e.\ cost). This presents an obvious problem: since greedy search is not complete, our planner-world-simulator loop can't be complete either. In fact, the situation is worse --- greedy search computes the cost of all candidates for expansion and chooses the cheapest, whereas our planner, being heuristic, might not consider certain nodes at all.

This might seem damning, but we must also consider the interaction with attention and memory. First, planned steps are committed to memory and thus, we gain access to past costs. An agent does not plan blindly, but can recall how long its plans are and what costs past planned steps entail. Given this information, we can turn the greedy algorithm into an A$^*$ search, with the qualification that the planner might not consider certain nodes. The mechanism of attention can further be used to enhance the search: if planning along a certain path takes too long, the agent might decide to abandon it altogether and start afresh with a different strategy. This failure too is stored in memory and can influence the planner in the new planning process by making the proposing of steps of the previously pursued path  unlikely.

\section{Proposed architecture}\label{sec:proposedArchitecture}

\section{Implementation}\label{sec:implementation}

Having laid the theoretical framework, we come to the practical part of this thesis --- a proof-of-concept implementation of multiple affective agents interacting with each other. This section contains the following parts: (1) the world in which act, (2) the architecture of these agents, and (3) the evolutionary changes in the agent pool from generation to generation.

\subsection{World}

The choice of world profoundly affects the implementation of the agent: its knowledge base, mechanism of perception and interaction, the required complexity of the implementation, etc. On one hand, the world should be simple enough to permit a reasonably small and effective agent which does not have to solve hard AI problems (like human-level sight) to deal with what we, in this context, might call details --- but on the other, the world should be sufficiently complex to allow the agent to shine. This is especially true in the case of an affective agent whose actions should be visibly influenced in rich and subtle ways by its emotional state. I shall first lay out the design goals and then evaluate three possible worlds for agents.

\paragraph{Design goals} The two most important criteria for prospective worlds are richness of interaction and world complexity, in that order. As said, an evaluation of affective agents is only possible if they can interact with their environment and other entities in a sufficiently complex way to allow agents with different emotional profiles to be distinguished from each other. Mechanisms of problem-solving like STRIPS \cite{fikesNilsson}, A* \cite{nilssonAStar}, ASP \cite{lifschitz}, forward-/backward-planning, etc. have been explored in the context of structurally simple worlds, generally those representable through propositional logic, cost-functions, decision tress, and the like. While these are useful, they are less appropriate in an affective scenario for the following two reasons:

\begin{enumerate}
	\item they are geared towards finding provably optimal solutions to computationally expensive but conceptually simple problems like planning or game-playing and
	\item they rely heavily on hand-crafted ontologies and domain knowledge on the part of the human programmer.
\end{enumerate}

For a world to be useful to us and to avoid these pitfalls, it should be in some sense realistic: it should permit a large number of different kinds of interactions, and it should not provide agents in it with perfect knowledge about its rules. 

I admit that I here stand in opposition with Marvin Minsky, who famously recommended the use of idealized micro-worlds to study artificial intelligence, in that same vein in which physics makes use of ideal, frictionless planes and perfect spheres. His argument certainly has merit, but I believe that emotion is too complex a phenomenon for such abstract scenarios. In too simple a setting, pure reasoning not only easily outperforms emotional behaviour, but avenues for exhibiting emotional behaviour are scarce to begin with. For this reason, I propose that, in this context, rich interactions should take precedence over idealization and simplicity.

It is of course still desirable to minimize complexity as far as possible. An overwhelmingly complex world has two obvious drawbacks: first, the required complexity of an agent scales with the complexity of the world; second, the more complex the world, the harder it is to reason about it. If there are a hundred ways to succeed, for instance, agent performance becomes quite difficult to measure.

\subsubsection{Blocks world}

Blocks worlds are the simplest type of abstract world, and many variations exist. They all have in common a number of shapes placed on top of each other in a 2-dimensional world. An agent can pick up and move a shape if and only if there are no other shapes on top of it (and if it is not already holding one). The goal generally consists of achieving some desired configuration of shapes, such as building or piecewise transporting a tower, or collecting all red triangles. 

Micro-worlds like blocks worlds have extensively studied. In this, their simplicity has been their great advantage --- That very simplicity is serious problem for us, however. Affect is inherently a subtle and social phenomenon; it is not clear how it could be believably exhibited in such an abstract and simple world. The very same properties which expedite their theoretical study make them useless for our evaluation.

%\subsubsection{Real world}

\subsubsection{Wumpus world}

The traditional Wumpus world, as described in Russell and Norvig's {\em Artificial Intelligence: A Modern Approach} \cite[p. 236]{norvig}, is a grid-based, 4x4 cave world with one agent, one monster --- the Wumpus --- and gold placed in random rooms. The agent starts at position $\langle 1,1\rangle$ and can move forward or turn 90$^\circ$ to the left or right. If it enters a room with a pit or a live Wumpus, it dies; its goal is to find and collect the gold and then move back to position $\langle 1,1\rangle$ to climb out of the cave. In addition, it has one arrow which he can fire straight ahead to defend against the Wumpus. The agent has only the following local information \cite[p. 237]{norvig}:
\begin{itemize}
	\item In the square containing the Wumpus and in the directly (not diagonally) adjacent squares, the agent will perceive a {\em Stench}.
	\item In the squares directly adjacent to a pit, the agent will perceive a {\em Breeze}.
	\item In the square where the gold is, the agent will perceive a {\em Glitter}.
	\item When an agent walks into a wall, it will perceive a {\em Bump}.
	\item When the Wumpus is killed, it emits a woeful {\em Scream} that can be perceived anywhere in the cave.
\end{itemize}

This type of world is simple enough to be amenable to rule-based reasoning, although it can contain ambiguous situations where the agent does not have enough information to make the best choice. For example, if an agent moves to position $\langle p_x,p_y \rangle$ and experiences a breeze, 1, 2, or 3 adjacent rooms may contain pits, but it cannot be safely determined which ones these are. Thus,  occasionally, the agent must choose between climbing out without the gold and risking death by pit or Wumpus.

For our purposes, this is a bit too simple, however. Caution/bravery is the only axis along which agents can be differentiated and although various complex behaviours --- such as trying one dangerous cell, then going back and trying another one to explore the world --- are possible, these do not have a clear relation to emotional states.

Let us, while staying true to the spirit of the original, now define a type of extended Wumpus world \wext\ that allows more varied interaction between agent an environment.

\begin{definition}[\wext-type world]\label{def:wext}
	Let $\type{T_v}$, $\type{T_e}$, $\type{T_g}$ be arbitrary types. Further, let $G$ be a directed graph with vertex labels of type $\type{T_v}$ and edge labels of type $\type{T_e}$, and let $\mathrm{gl}$ be an object of type $\type{T_g}$. Then the tuple \tuple{G, \mathrm{gl}} is a \wext-type world \paren{with type parameters $\type{T_v}$, $\type{T_e}$, $\type{T_g}$}. We call $G$ the {\em world frame} and $\mathrm{gl}$ the {\em world data}.
\end{definition}

We can interpret each vertex $v$ in the graph as a room with attached data $l(v)$ of type $\type{T_v}$, and each edge $e$ as an unidirectional connection between rooms with attached data (such as path costs) $l(e)$ of type $\type{T_e}$. $\mathrm{gl}$ is the global world data. Next, we specify some properties of the world frame:

\begin{definition}[World properties]
	Let $W = \tuple{G,\mathrm{gl}}$ be a \wext-world. We say that $W$ has property $X$ iff it fulfils the first-order sentence corresponding to $X$. The following properties are of importance:
	
	

	\begin{center}
		\begin{tabular}[b]{l l}
		\toprule
		\textbf{Property name} & \textbf{FO sentence}\\
		\midrule\addlinespace[0.7em]
		Reflexive & $\allQ{v \in V(G)} (v,v) \in E(G)$\\ \addlinespace[0.7em]
		Non-Euclidean &
		\begin{minipage}[t]{0.65\textwidth}
			$\allQ{\textit{ pairwise distinct } v_1,v_2,v_3 \in V(G)}$\\$\{(v_1,v_2),(v_1,v_3)\} \subseteq E(G) \Rightarrow (v_2,v_3) \notin E(G)$
		\end{minipage}\\ \addlinespace[0.7em]
		Symmetrical & $\allQ{v_1,v_2 \in V(G)} (v_1,v_2) \in E(G) \Rightarrow (v_2,v_1) \in E(G)$\\ \addlinespace[0.7em]
		Connected & $\allQ{v_1,v_2 \in V(G)}$ there exists a path from $v_1$ to $v_2$ in $G$\\ \addlinespace[0.7em]
		
		$n$-dimensionally embeddable
		 &
		\begin{minipage}[t]{0.65\textwidth}
		there exists an infinite graph $S$ such that
		\begin{enumerate}
			\item $V(G) \subseteq V(S)$,
			\item $E(G) \subseteq E(S) \cup \{ (v,v)\ |\ (v,v) \in E(G) \}$,
			\item $S$'s drawing, embedded into $\R^n$, forms a regular tiling, and
			\item $(v_1,v_2) \in E(S)$ iff the distance between $v_1$ and $v_2$ in $\R^n$ is 1.
		\end{enumerate}
		\end{minipage}
		
		\\ \addlinespace[0.5em]
		\bottomrule
		
		\end{tabular}
	\end{center}
\end{definition}

The first four properties speak for themselves. As for the fifth --- Figure~\ref{fig:2dgrid} shows an example of a 2-dimensionally embeddable frame. A frame $G$ is $n$-dimensionally embeddable if it is a fragment of an infinite, $n$-dimensional, square grid of nodes $S$, plus any loops $G$ might have. When we embed this infinite grid $S$ into $\R^n$ through an embedding, every edge corresponds to a vector of length 1 along exactly one dimension. If we additionally take $G$'s loops to correspond to null-vectors, this induces an {\em edge direction function} and a {\em position function}:

\begin{definition}[Edge direction and position]
Let $W = \tuple{G, \mathrm{gl}}$ be an $n$-dimensionally embeddable world (for some $n$) and $\epsilon$ an embedding of $W$ into $\R^n$. Then we have an {\em edge direction function} 

$$\Delta_n^\epsilon : E(G) \rightarrow \{0,x_1^+,x_1^-,x_2^+,x_2^-,\dots,x_n^+,x_n^-\}$$

with $0$ corresponding to a loop and $x_i^+$/$x_i^-$ corresponding to forward/backward movement in the $i$th dimension. We also have a {\em position function}

$$\pi^\epsilon : V(G) \rightarrow \R^n.$$

When the number of dimensions and the embedding are obvious, we omit $n$ and $\epsilon$.
Since $\pi^\epsilon$ is injective, an inverse $(\pi^\epsilon)^{-1}$ also exists. Through it, we define the {\em indexing function} of $W$:

$$
	\begin{array}{l}
		[] : n\textit{-dimensionally embeddable world} \rightarrow \R^n \rightarrow \type{Maybe}\ V(G)\\
		W[p] \equiv \left\{
			\begin{array}{l l}
				\type{Just } (\pi^\epsilon)^{-1}(p) & \textit{if } (\pi^\epsilon)^{-1}(p) \textit{ is defined}\\
				\type{Nothing} & \textit{otherwise}
			\end{array}
			\right.
	\end{array}
$$
\end{definition}

We will give agents access to $\Delta_n^\epsilon$ and $\pi^\epsilon$ (or simply $\Delta$ and $\pi$) to allow them to determine their position and direction in the world. Providing such information might seem problematic, but we thereby free ourselves from having to insert things like landmarks, wind currents, stars, and other navigational aids into the world. Given that navigation is not the focus of this thesis, this seems an appropriate simplification. Using the above properties, we can specify a subtype of \wext-type worlds:

\begin{figure}
	\centering
	\includegraphics{figs/2dgrid.png}
	\caption{An example of a 2-dimensionally embeddable world.}
	\label{fig:2dgrid}
\end{figure}

\begin{definition}[2D grid world]
	Let $W = \tuple{G,\mathrm{gl}}$ be a \wext-type world \paren{with type variables $\type{T_v}, \type{T_e}, \type{T_g}$}. If $W$ is reflexive, connected, and 2-dimensionally embeddable $W$ is a {\em 2D grid world}.
	Every 2D grid world has an associated function $\Delta_2 : E(G) \rightarrow \{0,x_1^+,x_1^-,x_2^+,x_2^- \}$.\\
	Note: every $n$-dimensionally embeddable world is also symmetrical and non-Euclidean.
\end{definition}

Grid worlds, as we have seen, are potentially infinite, n-dimensional grids, although their cells need not form a square or cube. Their shape can be irregular in that some rooms and connections may be missing, as long as the shape as a whole stays connected.

2D grid worlds are representationally the same as \wext-type worlds; they just have some structural invariants on their frames. If we additionally specialize the representation through the type parameters $\type{T_v}$, $\type{T_e}$, and $\type{T_g}$, we arrive at the type of world which will serve as the environment for our agents: the ``jungle world'' \wjun.

\begin{definition}[\wjun]
\label{def:wjun}
Let $\type{T_v}$, $\type{T_e}$, $\type{T_g}$ be the following tuples:

$$
	\begin{array}{r c l}
		\type{TV_{\mathrm{jun}}} & = & \langle \field{agents} :: [\type{Agent}],\\
		           &   &       \ \field{wumpus} :: [\type{Wumpus}],\\
		           &   & 	   \ \field{plants} :: \type{Maybe\ Plant},\\
		           &   &       \ \field{stench} :: \R,\\
		           &   &       \ \field{breeze} :: \R,\\
		           &   &	   \ \field{pit}    :: \B,\\
		           &   &	   \ \field{gold}   :: \N \rangle 
		\\
		\\
		\type{TE_{\mathrm{jun}}} & = & \langle \field{danger} :: \R,\\
				   &   &       \ \field{fatigue} :: \R \rangle
		\\
		\\
		\type{Temp} & = & \type{Freezing} + \type{Cold} + \type{Temperate} + \type{Warm} + \type{Hot}\\
		\\
		\type{TG_{\mathrm{jun}}} & = & \langle \field{time} :: \N,\\
				   &   &       \ \field{temperature} :: \type{Temp} \rangle
	\end{array}
$$

$\type{Agent}$ and $\type{Wumpus}$ are the following records:

$$
	\begin{array}{r c l}
		\type{Item} & = & \type{Gold + Fruit + Meat}\\
		\\
		\type{Agent} & = & \langle \field{name} :: \type{String},\\ 
					 & = & \ \field{direction} :: \type{X_1^+ + X_1^- + X_2^+ + X_2^-},\\
					 &   & \ \field{health} :: \R,\\
					 &   & \ \field{fatigue} :: \R,\\
					 &   & \ \field{inventory} :: [\type{\langle Item, \N \rangle}] \rangle
		\\
		\\
		\type{Wumpus} & = & \langle \field{health} :: \R,\\
					  &   & \ \field{fatigue} :: \R\rangle
	\end{array}
$$

Further, let $\mathrm{gl}$ be a value of type $\type{TG}_{\mathrm{jun}}$ and let $G$ be any 2D grid world with node labels of type $\type{TV}_{\mathrm{jun}}$ and edge labels of type $\type{TE}_{\mathrm{jun}}$. Then, $\tuple{G, \mathrm{gl}}$ is a \wjun-type jungle world.
\end{definition}

Although the field names are suggestive of the way in which a \wjun-type world works, the type, strictly speaking, only specifies the data and frame properties. We can employ such worlds in any sort of scenario, with whatever semantics we wish. Notwithstanding, our implementation will use a straightforward {\em standard semantics}, defined below.

\begin{definition}[Semantics and runs of \wjun-type worlds]
Let $\varphi$ be a function of type $\wjun \rightarrow \wjun$. $\varphi$ is called {\em semantics of \wjun-type worlds}.
Let $W$ be a \wjun-type world. The iterated application of $\varphi$ to $W$, given by the list ${[W, \varphi(W), \varphi^2(W), \varphi^3(W), \dots]}$, is called a {\em run of $W$ \paren{with semantics $\varphi$}}. $\varphi^n(W)$ is referred to as the {\em state of $W$'s simulation at time $n$ \paren{with semantics $\varphi$}}.
\end{definition}

\begin{definition}[Standards semantics of \wjun-type worlds]
\label{def:ssem}
The standard semantics for \wjun-type worlds are given by the function $\ssem :: \type{\wjun \rightarrow \wjun}$. $\ssem$ is defined as 
$$\ssem(W = \tuple{G, \mathrm{gl}}) = \tuple{G', \mathrm{gl}'}, $$
where $W'$ is identical to $W$, except for the following changes.\\

\begin{description}
	\item[Environment] For all $v \in V(G)$, perform the following:
	
	\begin{description}
		\item[Wumpus.] If there is a Wumpus in a cell $w$ at $\leq 3$ distance from $v$, increase $v$'s stench by
		$$
			\frac{
				\log_{3}(3 - \dist{v}{w}) - \field{stench}(l(v))
			}{2}
		$$
		If there is no Wumpus within distance $\leq 3$, decrease $v$'s stench by $\frac{1}{3}$, to a minimum of 0.
		
		\item[Plant.] If there is a plant on $v$ and it has no fruit, increase its growth by $\frac{1}{10}$. If its growth thereby reaches $1$, add a fruit to the plant and reset the growth to 0.
		
		\item[Pit] If there is a pit in a cell $w$ at a distance $\leq 3$ from $v$, set the breeze to
 		$$
			\log_{3}(3 - \dist{v}{w})
		$$
	\end{description}
	
	\item[Global data] The {\em daylight function} is defined as
	
	$$
			\field{cycle}(t) = 
			\left\{
				\begin{array}{l l l l}
					0 & \mt{if } & 20 & \leq |n - 25|\\
					1 & \mt{if } & 15 & \leq |n - 25| < 20\\
					2 & \mt{if } & 10 & \leq |n - 25| < 15\\
					3 & \mt{if } & 5 & \leq |n - 25| < 10\\
					4 & \mt{if } & & \ \ \ |n - 25| < 5
				\end{array}
			\right.
	$$
	
	The new global data $\mathrm{gl}'$ are given by
	
	$$
		\begin{array}{r c l}
			\mathrm{gl}' & = & \langle \field{time}(\mathrm{gl}) + 1\ \mathrm{mod}\ 50,\\
					   &   &       \ \field{cycle} \circ \field{temperature}(\mathrm{gl}') \rangle\\
			\\
			\field{cycle}(t) & = &
			\left\{
				\begin{array}{l l}
					\type{Freezing} & \mt{if }\ \field{light}(t) = 0\\
					\type{Cold} & \mt{if }\ \field{light}(t) = 1\\
					\type{Temperate} & \mt{if }\ \field{light}(t) = 2\\
					\type{Warm} & \mt{if }\ \field{light}(t) = 3\\
					\type{Hot} & \mt{if }\ \field{light}(t) = 4
				\end{array}
			\right.\\
		\end{array}
	$$
	
	\item[Wumpus behavior] Every Wumpus has three behaviors:
	
	\begin{enumerate}
		\item If the Wumpus is adjacent to a player, it performs the \action{attack} action on that player.
		
		\item If there is a player reachable with at most $\field{light} \circ \field{time} (\mathrm{gl})$ edges, move along the edge that minimizes the distance to that player (in $\R^2$). If there are multiple players, choose one at random as target. This target choice remains until the player is no longer within range.
		
		\item If there is no player within range, move in a random direction with probability
		
		$$
			0.2 \times (1 + \field{light} \circ \field{temperature} (\mathrm{gl})).
		$$
	\end{enumerate}
	
	Whenever a Wumpus travels along an edge $e$ with $\Delta(e) \neq 0$, apply $0.1$ damage with probability $\field{danger}(e)$.
	
	\item[Agent behavior] Agents always move after Wumpuses and, depending on their implementation, may choose one of the following actions:
	
	\begin{enumerate}
		\item[\action{move}] --- move along an edge $e$. If $\Delta(e) = 0$, restore $0.1$ of the agent's fatigue, otherwise reduce it by $0.05 \times \field{fatigue}(e)$. Additionally (if $\Delta(e) \neq 0$), apply $0.1$ damage with probability $\field{danger}(e)$.
		
		If an agent's fatigue is below $0.2$, it cannot choose this action.
		
		\item[\action{rotate}] --- the agent changes the direction into which it is facing to a value in ${x_1^+,x_1^-,x_2^+,x_2^-}$.
		
		\item[\action{attack}] --- move along an edge $e$ to attack an agent or wumpus.
		
		\item[\action{give}] --- give an item $i$ from the agent's inventory to another agent $a$.
		
		\item[\action{gather}] --- if there is a plant with a fruit on the agent's cell, take the fruit and put it in the agent's inventory.
		
		\item[\action{butcher}] --- if there is a dead Wumpus on the agent's cell, remove it and add an item of meat to the agent's inventory.
		
		\item[\action{collect}] --- if there is $n$ gold on the player's cell, take an amount $m$ ($1 \leq m \leq n$) of it an put it into the agent's inventory.
		
		\item[\action{eat}] --- eat a meat- or fruit-item $i$ from the agent's inventory. Restore $0.5$ health.
		
		\item[\action{gesture}] --- expresses a gesture in the form of a string $s$. All other agents on the same cell receive $s$.
	\end{enumerate}
	
\end{description}
\end{definition}

It should now be clear why \wjun\ is called a jungle world: it is a social hunter-gatherer scenario in which uncoordinated agents act and interact without any explicit performance measure. They can gather food or gold, rest, hunt wumpuses, communicate via gestures, and even develop friendships, but fundamentally, everyone is out for himself. The goal of simulating affective agents in such a world is to see which behavioural profiles are successful, how they develop over multiple generations, and how they engage each other.

\subsection{Agents}

The agents of our simulation are composed of two parts: their minds and their bodies. Their minds constitute their sensors and agents functions; their bodies, make up their actuators, although they are more than that. An agent's body can be damaged and healed, perceived by others, and it can hold items. As such, the bodies are actually part of the world. From the point of view of the agent's mind, they are external objects they happen to control.

\subsubsection{Body and percepts}

As we saw in Definitions~\ref{def:wjun} and \ref{def:ssem}, agents (1) have a body composed of a name, health, fatigue, and an inventory of items they carry, and (2) can execute one of a fixed set of actions at each step. These data function in the obvious way: the name is publicly available information other agents can use for identification, the agent is killed when its health drops to zero, fatigue determines the effectiveness when attacking and prevents movement when low, and the inventory is used to store items which the agent can use for itself or give away to others.

What we are missing is the description of the agent's percepts in the world. As in the original Wumpus world, an agent can perceive everything on its cell:
	\begin{enumerate}
		\item the list other agents,
		\item the list of (dead) Wumpuses,
		\item the plant, if present,
		\item the breeze,
		\item the stench, and
		\item the amount of gold.
	\end{enumerate}
	
In addition to this local information, the agent also has a sense of sight, modelled via an approximately $\frac{\pi}{2}$ radians cone, the length of which depends on daylight. Formally:

\begin{definition}[Sight cone]
	Let $W = \tuple{G, \mathrm{gl}}$ be a 2D grid world. Let an agent be on vertex $v \in V(G)$, facing into direction $d$. Let further $l_d$ be the line starting at $v$ and extending infinitely into direction $d$, and $l_{v,w}$ be the line from $v$ to $w$. Then, any other vertex $w \in V(G)$ falls into the agent's sight cone exactly if:
	
	\begin{enumerate}
		\item the angle between $l_{v,w}$ and $l_d$ is $\leq \frac{\pi}{4}$,
		\item $\dist{v}{w} \leq 1.5 \times (\field{light} \circ \field{time} (\mathrm{gl}) + 1)$, and
		\item there is a path $v_1, v_2, \dots, v_n$ from $v$ to $w$ in $G$ such that
		the distance between $v_i$ and the closest point along $l_{v,w}$ is $\leq \frac{\sqrt{2}}{2}$ ($1 \leq i \leq n$).
	\end{enumerate}
\end{definition}

Criterion one restricts the sight cone to $\frac{\pi}{4}$ radians; criterion two limits its length based on light conditions; criterion three demands rough line-of-sight, saying that the path in $G$ may never deviate more than one cell from the line in $\R^2$. Figure~\ref{fig:los} illustrates the working of this mechanism.

\begin{figure}
	\centering
	\includegraphics{figs/los.png}
	\caption{Sight cone of an agent at $\field{light}(t) = 2$. The cone with width $\frac{\pi}{4}$ signifies that agent's range of vision. Red vertices are perceived, black ones are not because they are blocked by holes in the world. The upper right vertex is not visible because the distance $\Delta$ between the direct line to it and the shortest edge path is too great.}
	\label{fig:los}
\end{figure}

If vertex $w$ falls into an agent's sight cone, it perceives $\pi(w)$ and the following cell data:

\begin{enumerate}
	\item the list of agents on $w$,
	\item the list of Wumpuses,
	\item the plant, it present,
	\item the pit, if present, and
	\item the amount of gold.
\end{enumerate}

The breeze and the stench, being non-visual, are not thus perceived. As we can see from criterion two in Definition~\ref{def:los} and the formulae for breeze and stench in Definition~\ref{def:ssem}, sight reaches farther, but is directed. The non-visual cues can tell an agent that it's in danger, but not from which direction that danger comes. If that agent consequently fails to look around, it may be attacked or wander into a pit.

\subsubsection{Cognition}

In this section, having sketched the underlying model and the relevant subsystems, I move on to the description of an implementation of an affective artificial agent. Its structure will necessarily be a gross simplification of any biological agent, but it will serve as a proof-of-concept.\\

\noindent
Note: instead of detailed partial descriptions, I, for now, sketch the rough outline of the proposed implementation to give a complete picture.

Outline of algorithm (provisional):

\begin{enumerate}
	\item One agent is composed of the following components:
		\begin{enumerate}
			\item Sensory perception
			\item Affect
			\item Planner and world simulator
			\item Memory
			\item Attention
		\end{enumerate}
	\item Sensory perception will be highly domain-dependent and, by necessity, simplified. Domains might be Blocks world or some custom-made game scenarios. In essence, it will deliver facts about objects and events in the world, bypassing the problem of faithfully implementing sight, hearing, smell, balance, pain, etc.
	\item At each step, sensory facts are put into the neural system's message space and are consumed by the evocative systems (PSBC, attention-refocusing, SJS, AS).
	\item The primitive parts of the PSBC can affect the executive system directly. Its higher-level parts, as well as the other three systems, work in the following way:
		\begin{enumerate}
			\item Attention-refocusing directs cognitive resources to stimuli it deems important. As a result, the priority of (certain) sensory messages is increased, increasing their influence on PSBC and the SJS.
			\item PSBC and the SJS evoke certain emotions. Thereby, they might set goals for the agent. This setting of goals engages the planner and world simulator, which try to devise a plan to meet said goal --- this is implemented via DLV-hex programs. These DLV-hex program, in turn, call the emotional systems in order to modulate both the planner and the world simulator according to the emotional state of the agent.
			\item Planned steps are committed to (short-term) memory.
		\end{enumerate}
	\item The PSBC's effects and the plans created by the agent are then translated into choices via conscious motor control, with the proviso that plans, even if deemed effective, are not executed under all circumstances: if some step is deemed too undesirable according to the emotional state of the agent at the time of execution, it might be abandoned and the planning stage starts anew. The kinds of actions that the agent executes are again highly dependent on the domain. Full physical simulation of a body would be prohibitively expensive, but simple, atomic actions like ``move left'' or ``attack'' would suffice for modelling purposes.
	\item The AS and sub-conscious motor control will not be implemented for the time being, as they are only relevant in highly sophisticated worlds.
\end{enumerate}

\section{Evaluation}

\todo[inline]{Results of the implementation.}

\endgroup

\pagebreak

\nocite{*}

\bibliographystyle{plain}
\bibliography{proposed_model}


\end{document}
