\documentclass{beamer}

% for themes, etc.
\mode<presentation>
{ \usetheme{boxes} }

\usepackage{times}  % fonts are up to you
\usepackage{graphicx}

% Tikz
\usepackage{tikz}
\usetikzlibrary{calc}
\usetikzlibrary{arrows.meta}
\usetikzlibrary{intersections}
\usetikzlibrary{positioning}
\usetikzlibrary{shapes}
\usetikzlibrary{arrows}
\usetikzlibrary{fit}

\usepackage{xcolor, colortbl}
\usepackage{multirow}

\definecolor{LightGreen}{rgb}{0.5,0.8,0.5}
\definecolor{IntenseGreen}{rgb}{0.5,1,0.5}

\definecolor{LightYellow}{rgb}{0.82,0.81,0.58}
\definecolor{IntenseYellow}{RGB}{249,240,91}

\newcommand{\tuple}[1]{\ensuremath{\langle #1 \rangle}}
\newcommand{\type}[1]{\mathtt{#1}}
\newcommand{\personality}[5]{\tuple{\type{#1},\type{#2}, \type{#3}, \type{#4}, \type{#5}}}

% these will be used later in the title page
\title{Design and Implementation of an Agent Architecture combining Emotions and Reasoning}
\author{Janos Tapolczai\\ Matrikelnummer 0825077}
\date{Wien, 09.05.2016}

\setbeamertemplate{navigation symbols}{}%remove navigation symbols
\setbeamertemplate{footline}[frame number]

%\newlist{arrowlist}{itemize}{1}
%\setlist[arrowlist]{label=$\Rightarrow$}

% note: do NOT include a \maketitle line; also note that this title
% material goes BEFORE the \begin{document}

% have this if you'd like a recurring outline
\AtBeginSection[]  % "Beamer, do the following at the start of every section"
{
   \begin{frame}<beamer> 
      \frametitle{Content} % make a frame titled "Outline"
      \tableofcontents[currentsection]  % show TOC and highlight current section
   \end{frame}
}

\tikzset{
   every overlay node/.style={
      anchor=north west,
   },
}
% Usage:
% \tikzoverlay at (-1cm,-5cm) {content};
% or
% \tikzoverlay[text width=5cm] at (-1cm,-5cm) {content};
\def\tikzoverlay{%
   \tikz[baseline,overlay]\node[every overlay node]
}%

\begin{document}
   
   % this prints title, author etc. info from above
   \begin{frame}
      \titlepage
   \end{frame}
   
   \begin{frame}{Content}
      \begin{itemize}
         \item Our Goal
         \item Methodology
         \item Design Philosophy
         \item Architecture
         \item Evaluation
         \item Conclusion
      \end{itemize}
   \end{frame}
   
   \section{Our Goal}

   \begin{frame}{Our Goal}
      \begin{itemize}
         \item We wanted to design an AI that combined emotions and reasoning.
         \item Emotions should guide the planning and serve as \emph{cognitive shortcuts}.
         
         \item In the design, we took cues from evolutionary neurobiology, as well as from Marvin Minsky and Aaron Sloman.
         
         \item We had three core questions:
         
         \begin{quotation}
            How could brains have evolved?
         \end{quotation}
         
         \begin{quotation}
            Could we approximate their functioning?
         \end{quotation}
         
         \begin{quotation}
            Would the end result be useful?
         \end{quotation}
      \end{itemize}
   \end{frame}
   
   \begin{frame}{Our Goal}
      \begin{itemize}
         \item We were not interested in logic as such; fidelity to biological systems was more important.
         
         \begin{quotation}
            It is not worth asking how to define consciousness, how to explain it, how it
            evolved, what its function is, etc. [...]\\
            Instead, we have many sub-capabilities, for which the answers are different: e.g., different kinds of perception, learning, knowledge, attention control, self-monitoring, self-control, etc.
         \end{quotation}
         
         Aaron Sloman, \texttt{comp.ai.philosophy} newsgroup
      \end{itemize}
   \end{frame}
   
%   \begin{frame}{Our Goal --- Nouvelle AI}
%         \begin{quote}\emph{
%            One aim of nouvelle AI is the relatively modest one of producing systems that display the same level of intelligence as insects.
%         }\end{quote}
%         
%        \begin{quote}\emph{
%           A central idea of nouvelle AI is that the basic building blocks of intelligence are \textbf{very simple behaviours} {\upshape[$\dots$]} More \textbf{complex behaviours} ``emerge'' from the interaction of these simple behaviours.
%         }\end{quote}
%
%         Jack Copeland. What is Artificial Intelligence?\\
%         {\footnotesize\url{http://www.alanturing.net/turing_archive/pages/Reference\%20Articles/what\_is\_AI/What\%20is\%20AI11.html}}
%   \end{frame}
   
%   \begin{frame}{Our Goal --- Evolutionary Neurobiology}
%      
%      \begin{itemize}
%         \item In what order did the components of the brain evolve?
%         \item What is their relation to each other?
%         \item How well is the brain ``designed''?
%      \end{itemize}
%      
%      \pause
%      
%      \begin{quote}\emph{
%            It is not worth asking how to define consciousness, how to explain it, how it
%            evolved, what its function is, etc., because there’s no one thing for which all
%            the answers would be the same. Instead, we have many sub-capabilities, for
%            which the answers are different: e.g., different kinds of perception, learning,
%            knowledge, attention control, self-monitoring, self-control, etc.
%         }\end{quote}
%
%         Aaron Sloman, quoted in \emph{The Emotion Machine}, p. 97.
%         \end{frame}

   \section{Methodology}
   
   \begin{frame}{Methodology}
      \begin{itemize}
         \item We implemented a \emph{toy AI} and a 2D world.
         \item Each agent was given a certain \emph{personality}.
         \item The only goal was survival.
         
         \vspace{1cm}
         
         \item We tested AI's general fitness,
         \item and we compared different personalities.
      \end{itemize}
   \end{frame}
   
   \section{Design Philosophy}
   
%   \begin{frame}{Biological Considerations}
%      \begin{itemize}
%            \item Nervous systems long predate brains.
%            \item Brains themselves evolved over hundreds of millions of years.
%      \end{itemize}
%   \end{frame}
   
%   \begin{frame}{Biological Considerations}
%      \includegraphics[width=\textwidth]{../Thesis/Figs/chordata.jpg}
%   \end{frame}

   \begin{frame}{Biological Considerations}
      \begin{itemize}
         \item Neurons evolved as a means of sending signals in an organism.
         \item Over time, neurons began to modulate the signals of other neurons.
         \item This was computation of a sort, but it was not \emph{well-designed}.
         \item Moreover, there was no long-term planning.
         \item Functionality evolved on an ad-hoc basis.\\
         
         \vspace{2mm}
         
            $\Rightarrow$ there are no well-defined components, interfaces.\\
            $\Rightarrow$ there is no calling model.\\
            $\Rightarrow$ neurons just \emph{listen in} on the activity of others.
            
         \pause
         \item Thus our working hypothesis: a \emph{White-box model of cognition}.
       \end{itemize}
   \end{frame}
   
   \begin{frame}{Biological Considerations - White-box model}
      \begin{itemize}
         \item Traditional programming languages work via \emph{black boxes}:\\
              \vspace{2mm}
               Functions are called, but their internals are unobservable.
               \vspace{2mm}
         \item We assume that components in the brain are \emph{white boxes}:\\
              \vspace{2mm}
               Any component can observe the activity of others.
      \end{itemize}
   \end{frame}
   
   \begin{frame}{Biological Considerations - White-box model}
      \begin{itemize}
         \item Of course, practical considerations apply.
         \vspace{2mm}
         \item We still use regular functions within components,
         \item but components can publish \emph{messages}.
         \item These are stored in a central \emph{message space}.
         \item Any component can read from and write into the message space.
         \vspace{2mm}
         \item Components are \emph{loosely coupled}:\\
            $\Rightarrow$ components don't know who reads their messages;\\
            $\Rightarrow$ components don't know who wrote the messages.
      \end{itemize}
   \end{frame}
   
   \section{Architecture}
   
   \begin{frame}{Architecture}
       \begin{itemize}
          \item The implementation consists of a
             \begin{itemize}
                \item \emph{world simulator} and an
                \item \emph{agent architecture}.
             \end{itemize}
       \end{itemize}
   \end{frame}
   
   \begin{frame}{Architecture --- World Simulator}
      \begin{center}
         \only<1>{\includegraphics[width=0.65\textwidth]{map.png}}%
         \only<2>{\includegraphics[width=0.65\textwidth]{map_agent.png}}%
         \only<3>{\includegraphics[width=0.65\textwidth]{map_wumpus.png}}%
         \only<4>{\includegraphics[width=0.65\textwidth]{map_plant.png}}%
         \only<5>{\includegraphics[width=0.65\textwidth]{map_fruit.png}}%
         \only<6>{\includegraphics[width=0.65\textwidth]{map_meat.png}}%
         \only<7>{\includegraphics[width=0.65\textwidth]{map_gold.png}}%
         \only<8>{\includegraphics[width=0.65\textwidth]{map_pit.png}}%
      \end{center}
%
      \begin{itemize}
         \setlength\itemsep{-1.25em}
         \item<1>Worlds are 2D grids.%
         \item<2>Agents see part of the world.%
         \item<3>Wumpuses seek and attack agents.%
         \item<4>Plants can be harvested.%
         \item<5>Fruit can be eaten.%
         \item<6>Meat can be eaten as well.%
         \item<7>Gold is just for ``trade''.%
         \item<8> Pits kill whatever falls into them.%
      \end{itemize}
   \end{frame}
   
%   \begin{frame}{Architecture --- World Simulator}
%      \begin{itemize}
%         \item Time moves in rounds.
%         \item In each round, every entity takes one action.
%            \begin{itemize}
%               \item The entity gets a slice of the world as \emph{perception} and
%               \item it has to return its desired action.
%            \end{itemize}
%         \item Actions are
%            \begin{itemize}
%               \item \emph{move},
%               \item \emph{rotate},
%               \item \emph{attack},
%               \item \emph{gather},
%               \item \emph{eat}, etc.
%            \end{itemize}
%      \end{itemize}
%   \end{frame}
   
   \begin{frame}{Architecture --- Agents}
      \begin{itemize}
            \item Based on evolutionary considerations, we designed 7 components.
            \vspace{2mm}
            \item Perception,
            \item Pre-social Behaviour Control (PSBC),
            \item Social Judgment System (SJS),
            \item Memory,
            \item Attention Control (AC),
            \item Decision Maker (DM), and
            \item Belief Generator (BG).
      \end{itemize}
      
      \tikzoverlay at (7.8cm,0.3cm) {
         \tikz node (label) at (0,0)[]{
            \includegraphics[width=3cm]{agent_components_intense.png}
         };
      };
   \end{frame}
   
%   \begin{frame}{Architecture --- Agents}
%      \begin{itemize}
%         \item Perception
%            \begin{itemize}
%               \item Complex messages from the world simulator are chopped up into atomic pieces.
%               \item The other components only have to deal with simple facts, e.g.
%                  \begin{itemize}
%                     \item ``My Health is 0.7'',
%                     \item ``Cell (x,y) has an entity'', or
%                     \item ``There is a plant on cell (x,y)''.
%                  \end{itemize}
%            \end{itemize}
%      \end{itemize}
      
%      \tikzoverlay[text width=1.8cm] at (9.7cm,3.9cm) {
%         \tikz node (label) at (0,0)[]{
%            \includegraphics[width=1.8cm]{agent_components_perception.png}
%         };
%      };
%   \end{frame}
   
   \begin{frame}{Architecture --- Agents}
      \begin{itemize}   
         \item Pre-Social Behaviour Control
         \pause
            \begin{itemize}
               \item The agent has four emotions:
                  \begin{itemize}
                     \item anger,
                     \item fear,
                     \item enthusiasm, and
                     \item contentment.   
                  \end{itemize}
%               \pause
%               \item Each emotion has an associated graph.
%               \pause
%               \item Each node responds to a specific message (e.g. low health).
%               \pause
%               \item Nodes may activate neighboring nodes.
%               \pause
%               \item The more nodes are activated, the stronger the emotion.
%               \pause
%               \vspace{3mm}
%               \item ``Pre-Social'' $\Rightarrow$ anger, fear, etc. are evolutionarily older than social emotions like trust or contempt.
            \end{itemize}
      \end{itemize}
   \end{frame}
      
   \begin{frame}{Architecture --- Agents}
      \includegraphics[width=\textwidth]{../Thesis/Figs/PSBC.png}
   \end{frame}
      
   \begin{frame}{Architecture --- Agents}
         \includegraphics[width=0.8\textwidth]{anger_filter.png}
   \end{frame}
   
   \begin{frame}{Architecture --- Agents}
      \begin{itemize}
         \item {Social Judgment System}
            \begin{itemize}
               \item Other agents evoke
                  \begin{itemize}
                     \item sympathy,
                     \item trust,
                     \item respect.
                  \end{itemize}
               \pause
               \item Every stranger has its own emotional levels.
               \item Agents can become friends (through positive interactions) or enemies.
            \end{itemize}
      \end{itemize}
      
%      \tikzoverlay[text width=1.8cm] at (9.7cm,3.9cm) {
%         \tikz node (label) at (0,0)[]{
%            \includegraphics[width=1.8cm]{agent_components_sjs.png}
%         };
%      };
   \end{frame}
   
   \begin{frame}{Architecture --- Agents}
      \begin{itemize}
         \item Memory
         \begin{itemize}
            \item We store our perceptions for later recall.
            \item Memory can also store \emph{imagined} worlds in a \emph{tree structure}.
         \end{itemize}
         \pause
         \item Attention Control
         \begin{itemize}
            \item We assist the Decision Maker by selecting important targets.
            \item ``Important'' means ''evokes the strongest emotions''.
         \end{itemize}
      \end{itemize}
      
%      \tikzoverlay[text width=2cm] at (9.5cm,3.3cm) {
%         \tikz node (label) at (0,0)[]{
%            \includegraphics[width=2cm]{agent_components_memory_ac.png}
%         };
%      };
   \end{frame}
   
   \begin{frame}{Architecture --- Agents}
      \begin{itemize}
         \item Decision Maker
            \begin{itemize}
               \item Each emotion has some actions associated with it, e.g.,
               \begin{itemize}
                  \item attacking is associated with anger,
                  \item eating is associated with enthusiasm.
               \end{itemize}
               \pause
               \item The PSBC and the AC guide the planning:
               \begin{itemize}
                  \item We select an action associated with our strongest emotion,
                  \item and the cell(s) which have the most attention.
               \end{itemize}
               \pause
               \item If our emotion is strong enough, we take a \emph{real} action,
               \item otherwise, we take a \emph{hypothetical} one.
               \vspace{2mm}
               \item The DM can also abort plans if, e.g., fear begins to override anger.
            \end{itemize}
      \end{itemize}
      
%      \tikzoverlay[text width=1.8cm] at (9.7cm,3.9cm) {
%         \tikz node (label) at (0,0)[]{
%            \includegraphics[width=1.8cm]{agent_components_dm.png}
%         };
%      };
   \end{frame}
   
   \begin{frame}{Architecture --- Agents}
      \begin{itemize}
         \item Belief Generator
         \begin{itemize}
            \item If the DM chose a \emph{hypothetical} action, we simulate its consequences.
            \item We use the actual world simulator, but we construct the world from memory.
            \item The agent is then given \emph{imagined perceptions}.
         \end{itemize}
      \end{itemize}
      
%      \tikzoverlay[text width=1.8cm] at (9.7cm,3.9cm) {
%         \tikz node (label) at (0,0)[]{
%            \includegraphics[width=1.8cm]{agent_components_bg.png}
%         };
%      };
   \end{frame}
   
   \begin{frame}{Architecture --- Agents}
      \begin{itemize}
         \item {How does this all fit together?}
      \end{itemize}
      
      \begin{center}
         \includegraphics[width=0.8\textwidth]{plan_0.png}
      \end{center}
      
      ~
   \end{frame}
   
   \begin{frame}{Architecture --- Agents}
      \begin{itemize}
         \item {How does this all fit together?}
      \end{itemize}
      
      \begin{center}
         \includegraphics[width=0.8\textwidth]{plan_1.png}
      \end{center}
      
      \emph{Perception} distributes its messages.
   \end{frame}
   
   \begin{frame}{Architecture --- Agents}
      \begin{itemize}
         \item {How does this all fit together?}
      \end{itemize}
      
      \begin{center}
         \includegraphics[width=0.8\textwidth]{plan_2.png}
      \end{center}
      
      The \emph{Decision Maker} is informed about the affective reactions.
   \end{frame}
   
   \begin{frame}{Architecture --- Agents}
      \begin{itemize}
         \item {How does this all fit together?}
      \end{itemize}
      
      \begin{center}
         \includegraphics[width=0.8\textwidth]{plan_3.png}
      \end{center}
      
      Option 1: The \emph{Belief Generator} simulates the consequences.
   \end{frame}
   
   \begin{frame}{Architecture --- Agents}
      \begin{itemize}
         \item {How does this all fit together?}
      \end{itemize}
      
      \begin{center}
         \includegraphics[width=0.8\textwidth]{plan_4.png}
      \end{center}
      
      The loop begins anew, but with \emph{imagined} perceptions.
   \end{frame}
   
   \begin{frame}{Architecture --- Agents}
      \begin{itemize}
         \item {How does this all fit together?}
      \end{itemize}
      
      \begin{center}
         \includegraphics[width=0.8\textwidth]{plan_5.png}
      \end{center}
      
      Option 2: The \emph{Decision Maker} chooses a real action.
   \end{frame}
   
   \begin{frame}{Architecture --- Agents}
      \begin{itemize}
         \item To simplify it: DM and BG are in a loop.
      \end{itemize}
      
      \begin{center}
         \includegraphics[width=0.3\textwidth]{bg_dm_loop.png}
      \end{center}
      
      \begin{enumerate}
         \item We choose a \emph{hypothetical} action,
         \item Then we simulate its consequences.
      \end{enumerate}
      
      \begin{itemize}
         \item We repeat this until the DM deems the outcome satisfactory and chooses a \emph{real} action.
      \end{itemize}
   \end{frame}
   
   \section{Evaluation}
   
   \begin{frame}{Evaluation}
      \begin{itemize}
         \item We created 32 populations with different personalities
         \item and simulated 50 rounds with each population.
         \item We combined weak/strong anger, fear, enthusiasm, contentment, as well as friendly/hostile demeanor.
         \item Throughout the simulation, we collected data: the number of
         \begin{itemize}
            \item gifts given,
            \item surviving Wumpuses,
            \item surviving agents,
            \item plants harvested, etc.
         \end{itemize}
      \end{itemize}
   \end{frame}
   
   \begin{frame}{Results}
      \begin{table}
         \centering
         \only<1>{%
         \begin{tabular}{ l | c | c }
            \emph{Personality fragment} & \emph{weak/hostile} & \emph{ strong/friendly} \\
            \hline
            Anger & 24.25 & 22.31\\
            Fear & 22.44 & 24.13\\
            Enthusiasm & 22.75 & 23.81\\
            Contentment & 23.06 & 23.50\\
            Hostility & 23.63 & 22.94\\
            \hline
         \end{tabular}}
         \only<2>{%
            \begin{tabular}{ l | c | c }
               \emph{Personality fragment} & \emph{weak/hostile} & \emph{ strong/friendly} \\
               \hline
               \rowcolor{LightGreen}Anger&\cellcolor{IntenseGreen}24.25&22.31\\
               Fear & 22.44 & 24.13\\
               Enthusiasm & 22.75 & 23.81\\
               Contentment & 23.06 & 23.50\\
               Hostility & 23.63 & 22.94\\
               \hline
            \end{tabular}}
         \only<3>{%
            \begin{tabular}{ l | c | c }
               \emph{Personality fragment} & \emph{weak/hostile} & \emph{ strong/friendly} \\
               \hline
               \rowcolor{LightGreen}Anger&\cellcolor{IntenseGreen}24.25&22.31\\
               \rowcolor{LightYellow}Fear&22.44 &\cellcolor{IntenseYellow}24.13\\
               Enthusiasm & 22.75 & 23.81\\
               Contentment & 23.06 & 23.50\\
               Hostility & 23.63 & 22.94\\
               \hline
            \end{tabular}}
         \caption{Average number of surviving agents, by personality fragment.}
         \label{tab:numAgentsAvg}
      \end{table}
      
      \begin{itemize}
         \item<3-> Weak anger and strong fear are useful!
      \end{itemize}
   \end{frame}
   
   \begin{frame}{Results}
      \begin{itemize}
         \item Other surprising results:
      \end{itemize}
      
      \begin{table}
         \centering
            \begin{tabular}{ l | c }
               \emph{Personality} & \emph{Wumpuses} \\
               \hline

					$\personality{S}{W}{W}{W}{H}$ & 0\\
               \multicolumn{2}{c}{$\dots$}\\
               $\personality{W}{W}{W}{W}{H}$ & 0\\
               $\personality{S}{W}{W}{S}{F}$ & 1\\
               $\personality{S}{W}{W}{W}{F}$ & 2\\
               \multicolumn{2}{c}{$\dots$}\\
               $\personality{S}{W}{S}{W}{F}$ & 19\\
               $\personality{S}{W}{S}{W}{H}$ & 19\\
               \hline
            \end{tabular}
            \caption{Number of meat items given as gifts after 50 rounds.}
            \label{tab:numWumpuses}
      \end{table}
      
      \begin{itemize}
         \item $\personality{S}{W}{S}{W}{H}$ means ``strong anger, weak fear, strong enthusiasm, weak contentment, hostile demeanor''.
         \item Agents with strong anger and weak fear killed everything and shared the meat among themselves.
      \end{itemize}
   \end{frame}
   
   \section{Conclusion}
   
   \begin{frame}{Conclusion}
      \begin{itemize}
         \item We designed an agent architecture based on evolutionary considerations.
         \pause
         \item Our aim was to approximate real organisms.
         \pause
         \item Agents were put into a moderately complex game world
         \pause
         \item and performed reasonably well.
         \pause
         \item Personalities differentiated themselves in interesting ways.
      \end{itemize}
   \end{frame}
   
   \begin{frame}[plain, c]
      \begin{center}
         \huge Thank you for your attention!
      \end{center}
   \end{frame}
   
%   \begin{frame}
%      \frametitle{A Geometry Proof}
%      
%      (Illustrating {\sc beamer}'s $\backslash$uncover command.)
%      \vskip 0.5in
%      
%      \begin{theorem}
%         The angles in a triangle sum to $180^{\circ}$.
%      \end{theorem}
%      
%      \pause
%      
%      Plan:  Extend AC past C to D.  Draw CE parallel to AB.
%      
%   \end{frame}
   
%   \begin{frame}
%      \begin{proof}
%         \begin{tabular}{ll}
%            % uncover makes advanced overlay
%            \uncover<1->{1. u = y} & \uncover<2->{Alternate angles of a
%               transveral.} \\ 
%            \uncover<3->{2. v = x} & \uncover<4->{Consecutive interior angles of a
%               transveral} \\ 
%            \uncover<5->{3. z+u+v = $180^{\circ}$} & \uncover<6->{ACD is a straight
%               line.} \\ 
%            \uncover<7->{4. z+y+x = $180^{\circ}$} & \uncover<8->{Substitution
%               from Steps 1 and 2.} \\
%         \end{tabular}
%      \end{proof}
%   \end{frame}
\end{document}
